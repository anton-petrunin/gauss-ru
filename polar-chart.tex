\chapter{Полугeодезичeские координаты}
\label{chap:semigeodesic}

Это вычислительная глава, в ней выводятся нескольких утверждений, обсуждавшихся выше, включая 
эквивалентное определение радиуса инъективности (\ref{prop:inj-rad}),
то, что кратчайшие являются геодезическими (\ref{prop:gamma''})
и формулу Гаусса --- Бонне (\ref{thm:gb}).
Кроме того, мы обсудим внутренние изометрии между поверхностями и докажем 
замечательную теорему Гаусса, о том, что \textit{гауссова кривизна является внутренним инвариантом}.

\section{Полярные координаты}\label{sec:Polar coordinates}

{\sloppy

Свойство экспоненциального отображения, описанное в \ref{prop:exp}, можно использовать для задания \index{полярные координаты}\emph{полярных координат} на гладкой поверхности.

}

А именно, пусть $p$ --- точка на гладкой поверхности $\Sigma$ и
$(r,\theta) \z\in \mathbb{R}_{\ge0} \times \mathbb{S}^1$ --- полярные координаты на касательной плоскости $\T_p$.
Если $\vec v\in \T_p$ имеет координаты $(r,\theta)$,
то будем говорить, что $s(r,\theta)\z=\exp_p\vec v$ --- это точка на $\Sigma$ с полярными координатами $(r,\theta)$.

Из точки $p$ в данную точку $x$ может идти много геодезических, а может их вовсе не быть.
Поэтому одна точка может иметь несколько представлений в полярных координатах, а может и не иметь их вовсе.
Однако из~\ref{prop:exp} получаем следующее.

\begin{thm}{Наблюдение}\label{obs:polar}
Пусть $s\:(r,\theta)\mapsto s(r,\theta)$ задаёт полярные координаты на гладкой поверхности~$\Sigma$ с началом в точке $p$.
Тогда существует такое $r_0>0$, что $s$ является регулярным для любой пары $(r,\theta)$ при $0<r<r_0$.

Более того, если $0\le r_1,r_2<r_0$, то $s(r_1,\theta_1) \z= s(r_2,\theta_2)$ тогда и только тогда, когда
$r_1=r_2=0$ или $r_1=r_2$ и $\theta_1\z=\theta_2+2\cdot n\cdot\pi$ для некоторого целого числа~$n$.
\end{thm}

Далее мы всегда будем предполагать, что полярные координаты на поверхности определены только при $r<r_0$,
а значит, они ведут себя обычным образом.

Следующее утверждение сыграет основную роль в строгом доказательстве того, что кратчайшие являются геодезическими, см.~\ref{sec:proof-of-gamma''}.

\begin{thm}{Лемма Гаусса}\label{lem:palar-perp}
Пусть $(r,\theta)\mapsto s(r,\theta)$ --- полярные координаты на гладкой поверхности с началом в точке $p$.
Тогда
$s_\theta\perp s_r$
для любых $r$ и~$\theta$.
\end{thm}

\parbf{Доказательство.}
Выберем $\theta \in \mathbb{S}^1$.
По определению экспоненциального отображения, кривая $\gamma(t)\z=s(t,\theta)$ --- геодезическая с единичной скоростью, исходящая из $p$.
\begin{enumerate}[(i)]
\item Поскольку у $\gamma$ единичная скорость, $|s_r|=|\gamma'|=1$, и в частности,
 \[
 \tfrac{\partial}{\partial \theta}
 \langle s_r,s_r\rangle=0.\]
\item Поскольку $\gamma$ --- геодезическая, $s_{rr}(r,\theta)=\gamma''(r)\perp\T_{\gamma(r)}$,
и следовательно, 
\[
\langle s_\theta, s_{rr}\rangle=0.\]
\end{enumerate}
Отсюда вытекает, что
\[
\begin{aligned}
\tfrac{\partial}{\partial r}
\langle s_\theta, s_r\rangle
&=
\langle s_{\theta r},s_r\rangle
+
\cancel{\langle s_\theta,s_{rr}\rangle}=
\\
&=
\tfrac12
\cdot 
\tfrac{\partial}{\partial \theta}
\langle s_r, s_r\rangle=
\\
&=0.
\end{aligned}
\eqlbl{eq:<s',s'>'=0}
\]

Далее, $s_\theta(0,\theta)=0$, ведь $s(0,\theta)=p$ для любого $\theta$.
В частности,
$\langle s_\theta, s_r\rangle=0$
при $r=0$.
Согласно \ref{eq:<s',s'>'=0}, значение 
$\langle  s_\theta, s_r\rangle$ не зависит от~$r$ при фиксированном~$\theta$.
А значит,
\[\langle s_\theta, s_r\rangle=0\]
для любых $r$ и $\theta$.
\qeds


\section{Снова кратчайшие и геодезические}
\label{sec:proof-of-gamma''}

В этом разделе мы воспользуемся полярными координатами и леммой Гаусса (\ref{lem:palar-perp}) для доказательства предложения~\ref{prop:gamma''}.


\parbf{Доказательство \ref{prop:gamma''}.}
Пусть $\gamma\:[0,\ell]\to\Sigma$ --- кратчайшая, параметризованная длиной, и $p=\gamma(0)$.
Предположим, что длина $\ell=\length\gamma$ достаточно мала, так, что $\gamma$ описывается в полярных координатах с началом в $p$;
скажем $\gamma(t)=s(r(t),\theta(t))$ для функций $t\mapsto \theta(t)$ и $t\mapsto r(t)$, где $r(0)=0$.

По правилу дифференцирования сложной функции,
\[\gamma'= s_\theta\cdot \theta'+ s_r\cdot r'
\eqlbl{eq:chain(gamma)}\]
если левая часть определена и $t>0$.
По лемме Гаусса \ref{lem:palar-perp}, $s_\theta\perp s_r$, и по определению полярных координат, $|s_r|=1$.
Следовательно, из \ref{eq:chain(gamma)} следует
\[|\gamma'(t)|\ge r'(t).\eqlbl{eq:|gamma'|=r'}\]
для любого $t>0$, где $\gamma'(t)$ определена.

Поскольку $\gamma$ параметризована длиной, 
\[|\gamma(t_2)-\gamma(t_1)|\le |t_2-t_1|.\]
В частности, $\gamma$ липшицева, и, по теореме Радемахера (\ref{thm:rademacher}), производная $\gamma'$ определена почти везде.
Из \ref{adex:integral-length:a} получаем
\begin{align*}
\length\gamma&=\int_0^\ell|\gamma'(t)|\cdot dt\ge
\\
&\ge\int_0^\ell r'(t)\cdot dt=
\\
&=r(\ell).
\end{align*}

По построению полярных координат, найдётся геодезическая длины $r(\ell)$ из $p=\gamma(0)$ в $q=\gamma(\ell)$.
Поскольку $\gamma$ --- кратчайшая, $r(\ell)=\ell$, и, более того, $r(t)=t$ для любого~$t$.
Это равенство выполняется, тогда и только тогда, когда в \ref{eq:|gamma'|=r'} достигается равенство при почти всех~$t$.
Последнее означает, что $\gamma$ --- геодезическая.

Остаётся проверить верность почти обратного.

Зафиксируем точку $p\in\Sigma$.
Пусть $\epsilon>0$, как в \ref{prop:exp}.
Предположим, что короткая геодезическая $\gamma$ (короче чем $\epsilon$) от $p$ до $q$ не минимизирует длину между своими концами.
Тогда существует кратчайшая от $p$ до $q$ отличная от $\gamma$.
Если пропараметризовать её длиной, то получим другую геодезическую.
То есть, существуют две геодезические от $p$ до $q$ длины меньше $\epsilon$.
Иными словами, найдутся такие два вектора ${\vec v},\vec w\in\T_p$, что $|{\vec v}|<\epsilon$, $|\vec w|<\epsilon$ и 
$q=\exp_p\vec v\z=\exp_p\vec w$.
Однако, по \ref{prop:exp}, экспоненциальное отображение $\T_p \to \Sigma$  инъективно в $\epsilon$-окрестности нуля --- противоречие.
\qeds


\section{Гауссова кривизна}\label{sec:jacobi-formula}

Пусть $s$ --- гладкое отображение из (возможно, неограниченного) координатного прямоугольника на плоскости $(u,v)$ в гладкую поверхность~$\Sigma$.
Отображение $s$ называется \index{полугеодезические}\emph{полугедезическим}, если для любого фиксированного $v$ отображение $u\mapsto s(u,v)$ является геодезической, параметризованной длиной, и $s_u\perp s_v$ для любых $(u,v)$.

По лемме Гаусса (\ref{lem:palar-perp}), полярные координаты на $\Sigma$ задаются полугедезическим отображением.

Пусть $\Norm=\Norm(u,v)$ --- нормаль к $\Sigma$ в точке $s(u,v)$.
Векторы $\Norm$, $\vec u\z=s_u$ и $\vec v\z=\Norm\times \vec u$ образуют ортонормированный базис для каждой пары $(u,v)$.
Напомним, что $s_v\perp \vec u$ и $s_v\perp \Norm$, ведь вектор $s_v(u,v)$ касателен к $\Sigma$ в $s(u,v)$. 
Следовательно, $s_v=b\cdot\vec v$ для некоторой гладкой функции $(u,v)\z\mapsto b(u,v)$.%
\footnote{Для фиксированного значения $v_0$ векторное поле $s_v=b\cdot\vec v$ описывает разницу между $\gamma_0$ и \textit{инфинитоземально близкой} геодезической $\gamma_1\:u\z\mapsto s(u,v_1)$.
Поля с таким свойством называются \index{поле Якоби}\emph{полями Якоби} вдоль $\gamma_0$.}



\begin{thm}{Предложение}\label{prop:jaccobi}
Пусть $(u,v)\mapsto s(u,v)$ --- полугеодезическое отображение на гладкую поверхность $\Sigma$, для которого $\Norm$, $\vec u$, $\vec v$, и $b$ определены выше.
Тогда 
\[b\cdot K+b_{uu}=0,\]
где $K=K(u,v)$ --- гауссова кривизна $\Sigma$ в точке $s(u,v)$.

Более того, 
\[
\langle\vec u_u,\vec u\rangle=
\langle\vec u_u,\vec v\rangle=
\langle\vec u_v,\vec u\rangle=0,
\quad\text{и}\quad
\langle\vec u_v,\vec v\rangle=b_u.
\]

\end{thm}

Доказательство проводится длинным, но несложным вычислением.


\parbf{Доказательство.}
Пусть $\ell=\ell(u,v)$, $m=m(u,v)$ и $n=n(u,v)$ --- компоненты матрицы, описывающей оператор формы в базисе $\vec u, \vec v$;
то есть
\[
\begin{aligned}
\Shape\vec u&=\ell\cdot \vec u+ m\cdot \vec v,
&
\Shape\vec v&=m\cdot \vec u+ n\cdot \vec v.
\end{aligned}
\eqlbl{eq:Shape(u,v)}
\]
Напомним, что (см.~\ref{sec:More curvatures})
\[K=\ell\cdot n-m^2.\]

Сначала давайте выведем предложение из следующие тождеств
\[
\begin{aligned}
\vec u_u&=\ell\cdot \Norm,
&
\vec u_v&=\phantom{-}b_u\cdot \vec v+b\cdot m\cdot\Norm,
\\
\vec v_u&=m\cdot \Norm,
&
\vec v_v&=-b_u\cdot \vec u+b\cdot n\cdot\Norm.
\end{aligned}
\eqlbl{eq:uu-vv}
\]
Действительно, 
\begin{align*}
b\cdot K&=b\cdot (\ell\cdot n-m^2)=
\\
&=\langle\vec u_u,\vec v_v\rangle-\langle\vec u_v,\vec v_u\rangle=
\tag{по \ref{eq:uu-vv}}
\\
&= 
\left(
\tfrac{\partial}{\partial v}
\langle\vec u_u,\vec v\rangle
-
\cancel{\langle\vec u_{uv},\vec v\rangle}
\right)-
\left(
\tfrac{\partial}{\partial u}
\langle\vec u_v,\vec v\rangle
-
\cancel{\langle\vec u_{uv},\vec v\rangle}
\right)=
\\
&=0-b_{uu}.
\tag{по \ref{eq:uu-vv}}
\end{align*}
Остальные тождества в предложении прямо следуют из \ref{eq:uu-vv}.
Остаётся доказать четыре тождества в \ref{eq:uu-vv}.


\parit{Вывод $\vec u_u=\ell\cdot \Norm$.}
Поскольку базис $\Norm$, $\vec u$ и $\vec v$ ортонормирован, это векторное тождество можно переписать в виде  трёх скалярных
\[
\begin{aligned}
\langle\vec u_u,\vec u\rangle&=0,
&
\langle\vec u_u,\vec v\rangle&=0,
&
\langle\vec u_u,\Norm\rangle&=\ell.
\end{aligned}
\]
Поскольку $u\mapsto s(u,v)$ --- геодезическая, $\vec u_u=s_{uu}(u,v)\perp\T_{s(u,v)}$.
Отсюда следуют первые два тождества.
Далее,
\begin{align*}
\langle\vec u_u,\Norm\rangle
&=\langle s_{uu},\Norm\rangle=
\tag{по \ref{thm:shape-chart}}
\\
&=    \langle \Shape s_u,s_u\rangle=
\tag{поскольку $\vec u=s_u$}
\\
&=\langle \Shape \vec u,\vec u\rangle=
\tag{по \ref{eq:Shape(u,v)}}
\\
&=\ell.
\end{align*}


\parit{Вывод $\vec u_v= b_u\cdot \vec v+b\cdot m\cdot\Norm$.}
Перепишем его в виде скалярных тождеств:
\[
\begin{aligned}
\langle\vec u_v,\vec u\rangle&=0,
&
\langle\vec u_v,\vec v\rangle&=b_u,
&
\langle\vec u_v,\Norm\rangle&=b\cdot m.
\end{aligned}
\eqlbl{eq:uu-vv:2}
\]

Поскольку $\langle\vec u,\vec u\rangle=1$, имеем
$0=\tfrac{\partial}{\partial v}\langle\vec u,\vec u\rangle=2\cdot\langle\vec u_v,\vec u\rangle$.
Отсюда вытекает первое тождество в \ref{eq:uu-vv:2}.
Далее,
\begin{align*}
\langle\vec u_v,\vec v\rangle&=\langle s_{vu},\vec v\rangle=
\tag{$s_v=b\cdot \vec v$}
\\
&=\langle \tfrac{\partial}{\partial u} (b\cdot\vec v),\vec v\rangle =
\\
&=b_u\cdot \langle \vec v,\vec v\rangle+b\cdot \langle \vec v_u,\vec v\rangle=
\tag*{($\langle \vec v,\vec v\rangle=1$ и $\qquad\qquad\qquad$} 
\\
&=b_u. 
\tag*{$0=\tfrac{\partial}{\partial u}\langle \vec v,\vec v\rangle=2\cdot\langle \vec v_u,\vec v\rangle$)}
\end{align*}
Получили второе тождество в \ref{eq:uu-vv:2}.
Наконец,
\begin{align*}
\langle\vec u_v,\Norm\rangle
&=\langle s_{uv},\Norm\rangle=
\tag{по \ref{thm:shape-chart}}
\\
&=\langle \Shape s_u,s_v\rangle=
\tag{$\vec u=s_u$ и $s_v=b\cdot \vec v$}
\\
&=\langle \Shape \vec u,b\cdot \vec v\rangle=
\tag{по \ref{eq:Shape(u,v)}}
\\
&=b\cdot m.
\end{align*}


\parit{Вывод $\vec v_u=m\cdot \Norm$ и $\vec v_v=-b_u\cdot \vec u+b\cdot n\cdot\Norm$.}
Напомним, что $\vec v=\Norm\times \vec u$.
Следовательно,
\[
\vec v_u=\Norm_u\times \vec u+\Norm\times \vec u_u,
\qquad\qquad
\vec v_v=\Norm_v\times \vec u+\Norm\times \vec u_v.
\eqlbl{eq:uu-vv:3+4}
\]
Выражения для $\vec u_u$ и $\vec u_v$ в \ref{eq:uu-vv} уже доказаны.
Далее,
\begin{align*}
-\Norm_u&=\Shape s_u=
&
-\Norm_v&=\Shape s_v=
\\
&=\Shape \vec u=
&
&=b\cdot\Shape \vec v=
\\
&=\ell\cdot\vec u+m\cdot\vec v,
&
&=b\cdot(m\cdot \vec u+ n\cdot \vec v),
\end{align*}
Остаётся подставить в \ref{eq:uu-vv:3+4} выражения для $\vec u_u$, $\vec u_v$, $\Norm_u$ и~$\Norm_v$. 
\qeds


Карта $(u,v)\mapsto s(u,v)$ и её локальные координаты называются \index{полугеодезические}\emph{полугеодезическими}, если отображение $(u,v)\z\mapsto s(u,v)$ является полугеодезическим.
Заметим, что функция $b=b(u,v)$ в полугеодезических координатах имеет постоянный знак.
Поэтому, обратив знак $\Norm$, можно (и должно) считать, что $b>0$;
иными словами, $b=|s_v|$.

\begin{thm}{Упражнение}\label{ex:semigeodesc-chart}
Покажите, что любую точку $p$ на гладкой поверхности $\Sigma$ можно покрыть полугеодезической картой.
\end{thm}

\begin{thm}{Упражнение}\label{ex:inj-rad}
{\sloppy
Пусть $p$ --- точка на гладкой поверхности~$\Sigma$.
Предположим, что отображение $\exp_p$ инъективно в шаре $B\z=B(0,r_0)_{\T_p}$.
Пусть полугеодезическое отображение $(r,\theta)\mapsto s(r,\theta)$ задаёт полярные координаты с началом в $p$, а функция $(r,\theta)\mapsto b(r,\theta)$ та же, что выше.

}

Докажите следующие утверждения:

\begin{subthm}{ex:inj-rad:sign}
$b(r,\theta)$ не меняет знак при $0\z\le r\z<r_0$.
\end{subthm}

\begin{subthm}{ex:inj-rad:0}
$b(r,\theta)\ne0$, если $0< r<r_0$.
\end{subthm}

\begin{subthm}{ex:inj-rad:prop:inj-rad}
Примените \ref{SHORT.ex:inj-rad:sign} и \ref{SHORT.ex:inj-rad:0}, чтобы доказать \ref{prop:inj-rad}.
\end{subthm}
 
\end{thm}


Карта $(u,v)\mapsto s(u,v)$ называется \index{ортогональная карта}\emph{ортогональной}, если $s_u\perp s_v$ для любого $(u,v)$.
Например, любая полугеодезическая карта ортогональна.

Решение следующего упражнения похоже на \ref{prop:jaccobi}.

\begin{thm}{Упражнение}\label{lem:K(orthogonal)}
Пусть $(u,v)\mapsto s(u,v)$ --- ортогональная карта гладкой поверхности~$\Sigma$.
Обозначим через $K=K(u,v)$ гауссову кривизну $\Sigma$ в точке $s(u,v)$.
Определим 
\begin{align*}
a=a(u,v)&\df|s_u|,&
b=b(u,v)&\df|s_v|,\\
\vec u=\vec u(u,v)&\df\tfrac{s_u}a,&
\vec v=\vec v(u,v)&\df\tfrac{s_v}b.
\end{align*}
Пусть $\Norm=\Norm(u,v)$ --- единичный нормальный вектор в точке $s(u,v)$.

\begin{subthm}{lem:K(orthogonal):uu-vv}
Покажите, что 
\begin{align*}
\vec u_u
&=
-\tfrac1{b}\cdot a_v
\cdot
\vec v 
+
a\cdot \ell\cdot \Norm
,
&
\vec v_u
&=
\tfrac1{b}\cdot a_v
\cdot \vec u
+
a\cdot m\cdot \Norm
\\
\vec u_v
&=
\tfrac1{a}\cdot b_u\cdot\vec v
+
b\cdot m\cdot \Norm
,
&
\vec v_v
&=
-\tfrac1{a}\cdot b_u\cdot\vec u
+
b\cdot n\cdot \Norm,
\end{align*}
где $\ell=\ell(u,v)$, $m=m(u,v)$ и $n=n(u,v)$ --- компоненты матрицы, описывающей оператор формы в базисе $\vec u, \vec v$.
\end{subthm}

\begin{subthm}{lem:K(orthogonal):K}
Покажите, что
\[K=-\frac1{a\cdot b}\cdot
\left(
\frac{\partial}{\partial u}
\left(\frac{b_u}a \right)
+
\frac{\partial}{\partial v}
\left(\frac{a_v}b\right)
\right).\]
\end{subthm}
\end{thm}

\begin{thm}{Упражнение}\label{ex:conformal}
Предположим, что $(u,v)\mapsto s(u,v)$ --- это \index{конформная карта}\emph{конформная карта};
то есть существует функция $(u,v)\mapsto b(u,v)$ такая, что $b=|s_u|=|s_v|$ и $s_u\perp s_v$ для любого $(u,v)$.
(Функция $b$ называется {}\emph{конформным множителем} карты.)

Используя \ref{lem:K(orthogonal)}, покажите, что  
\[K=-\frac{\triangle (\ln b)}{b^2},\]
где $\triangle$ --- \index{лапласиан}\emph{лапласиан}; то есть $\triangle=\tfrac{\partial^2}{\partial u^2}+\tfrac{\partial^2}{\partial v^2}$, и 
 $K=K(u,v)$ --- гауссова кривизна $\Sigma$ в точке $s(u,v)$.
\end{thm}

Полезно знать, что \textit{любую точку гладкой поверхности можно покрыть конформной картой},
соответственные локальные координаты называются \index{изотермические координаты}\emph{изотермическими}.

\section{Вращение векторного поля}

Пусть $\gamma\:[0,1]\to \Sigma$ --- простая петля на гладкой ориентированной поверхности~$\Sigma$,
и $\vec u$ --- поле единичных касательных векторов на $\Sigma$, определённое в окрестности~$\gamma$.
Обозначим через $\vec v$ поворот $\vec u$ против часовой стрелки на угол $\tfrac{\pi}2$ в касательной плоскости в каждой точке; поле $\vec v$ можно также определить как $\vec v\df\Norm\times\vec u$, где $\Norm$ --- поле нормалей к $\Sigma$.
Тогда \index{вращение}\emph{вращение} $\vec u$ вдоль $\gamma$ определяется через интеграл
\[\rot_\gamma\vec u
\df
\int_0^1\langle\vec u'(t),\vec v(t)\rangle\cdot dt.\]

\begin{thm}{Лемма}\label{lem:rotation-parallel}
Пусть $\gamma\:[0,1]\to \Sigma$ --- простая петля с базовой точкой $p$ на гладкой ориентированной поверхности~$\Sigma$, а $\vec u$ --- поле касательных единичных векторов к $\Sigma$, определённое в окрестности~$\gamma$.
Тогда параллельный перенос $\iota_\gamma\:\T_p\to\T_p$ есть поворот по часовой стрелке на угол $\rot_\gamma\vec u$.

В частности, вращения различных векторных полей вдоль $\gamma$ могут различаться только на числа кратные $2\cdot\pi$.
\end{thm}

\parbf{Доказательство.}
Как и ранее, пусть $\vec v=\Norm\times\vec u$. 
Будем обозначать через $\vec u(t)$ и $\vec v(t)$ векторы полей $\vec u$ и $\vec v$ в точке $\gamma(t)$.

Пусть $t\mapsto \vec x(t)\in \T_{\gamma(t)}$ --- это параллельное векторное поле вдоль $\gamma$ с  $\vec x(0)\z=\vec u(0)$, и $\vec y\df\Norm\times\vec x$.

Заметим, что существует непрерывная функция $t\mapsto \phi(t)$ такая, что 
$\vec u(t)$ --- это поворот $\vec x(0)$ против часовой стрелки на угол $\phi(t)$.
Так как $\vec x(0)=\vec u(0)$, можно считать, что $\phi(0)=0$.
Тогда
\begin{align*}
\vec u&=\cos\phi\cdot \vec x+\sin\phi\cdot \vec y,
\\
\vec v&=-\sin\phi\cdot \vec x+\cos\phi\cdot \vec y.
\end{align*}
Отсюда
\begin{align*}
\langle\vec u',\vec v\rangle
=\phi'\cdot\biggl(&(\sin \phi)^2\cdot \langle\vec x,\vec x\rangle+(\cos \phi)^2\cdot \langle\vec y,\vec y\rangle
\biggr)=
\phi'.
\end{align*}

Следовательно,
\begin{align*}
\rot_\gamma\vec u&=\int_0^1\langle\vec u'(t),\vec v(t)\rangle\cdot dt=
\\
&=\int_0^1\phi'(t)\cdot dt=
\\
&=\phi(1).
\end{align*}

Заметим, что 
\begin{itemize}
\item $\iota_\gamma(\vec x(0))=\vec x(1)$,

\item  $\vec x (0) = \vec u (0) = \vec u (1),$ 

\item $\vec u(1)$ --- поворот $\vec x(1)$ против часовой стрелки на угол $\phi(1)\z=\rot_\gamma\vec u$,

\end{itemize}
Отсюда вытекает, что $\vec x(1)$ --- это \textit{поворот по часовой стрелке} $\vec x(0)$ на угол $\rot_\gamma\vec u$, и результат следует.

Последнее утверждение следует из \ref{prop:pt+tgc}.
\qeds


Следующая лемма пригодится при доказательстве формулы Гаусса --- Бонне.

\begin{thm}{Лемма}\label{lem:rotation-semigeoesic}
Пусть $(u,v)\mapsto s(u,v)$ --- полугеодезическая карта на гладкой поверхности~$\Sigma$.
Предположим, что простая замкнутая кривая $\gamma$ ограничивает диск $\Delta$, который полностью покрывается отображением~$s$.
Тогда 
\[\rot_\gamma\vec u+\iint_\Delta K=0,\]
где $\vec u=s_u$, а $K$ обозначает гауссову кривизну поверхности~$\Sigma$.
\end{thm}

Вычисления ниже используют так называемую \index{формула Грина}\emph{формулу Грина}, формулирующейся следующим образом.

\textit{Пусть $D$ --- компактная область на $(u,v)$-плоскости, ограниченная кусочно гладким простым замкнутым путём $\alpha\:t\mapsto (u(t),v(t))$.
Предположим, что $\alpha$ ориентирована так, что $D$ находится слева от неё.
Тогда}
\[\iint_D (Q_u- P_v)\cdot du\cdot dv=\int_\alpha (P\cdot du+Q\cdot dv)\df \int_0^1 (P\cdot u'+Q\cdot v')\cdot dt\]
\textit{для любых двух гладких функций $P$ и $Q$, определённых на $D$.}

Формулы Грина и Гаусса --- Бонне похожи, обе связывают интеграл по области с интегралом по граничной кривой, и неудивительно, что одна помогает доказать другую.

\parbf{Доказательство.}
Пусть $\vec u$, $\vec v$ и $b$ определены, как в~\ref{sec:jacobi-formula}.
Запишем $\gamma$ в $(u,v)$-координатах: $\gamma(t)=s(u(t),v(t))$.
Тогда
\begin{align*}
\rot_\gamma \vec u&=\int_0^1\langle\vec u',\vec v\rangle\cdot dt=
\tag{поскольку $\vec u'=\vec u_u\cdot u'+\vec u_v\cdot v'$}
\\
&=\int_0^1[\langle\vec u_u,\vec v\rangle\cdot u'+\langle\vec u_v,\vec v\rangle\cdot v']\cdot dt=\tag{по \ref{prop:jaccobi}}
\\
&=\int_0^1b_u\cdot v'\cdot dt=\int_{s^{-1}\circ\gamma}b_u\cdot dv=
\tag{по формуле Грина}
\\
&=\iint_{s^{-1}(\Delta)}b_{uu}\cdot du\cdot dv=
\tag{$s_u\perp s_v \Longrightarrow \jac s=|s_u|\cdot|s_v|=b$}
\\
&=\iint_\Delta\frac{b_{uu}}{b}=
\tag{по \ref{prop:jaccobi}}
\\
&=-\iint_{\Delta}K.
\end{align*}
\qedsf

\section{Вывод формулы Гаусса --- Бонне}\label{sec:gauss--bonnet:formal}

{\sloppy

Напомним, что формула Гаусса --- Бонне записывается как $\GB(\Delta)\z=0$, где 
\[\GB(\Delta)
\df
\tgc{\partial\Delta}+\iint_\Delta K-2\cdot \pi,\]
$\Delta$ --- топологический диск на гладкой ориентированной поверхности,
ограниченный кусочно гладкой кривой $\partial \Delta$ ориентированной так, что $\Delta$ лежит от неё слева.

}

\parbf{Вывод формулы Гаусса --- Бонне (\ref{thm:gb}).}
Предположим, что $\Delta$ покрывается полугеодезической картой.
Из \ref{prop:pt+tgc}, \ref{lem:rotation-parallel} и \ref{lem:rotation-semigeoesic},
\[\GB(\Delta)
=
2\cdot n\cdot \pi,
\eqlbl{eq:gb(n)}\]
где $n=n(\Delta)$ --- целое число.

Согласно \ref{ex:semigeodesc-chart}, полугеодезической картой можно покрыть любую точку поверхности.
Таким образом, применив аддитивность $\GB$ (\ref{lem:GB-sum}) конечное число раз, получим \ref{eq:gb(n)} для любого диска $\Delta$ на~$\Sigma$.
Точнее, мы можем разрезать $\Delta$ гладкой кривой, которая проходит от границы до границы,
и повторить такое разбиение рекурсивно для полученных дисков;
см. рисунок.
\begin{figure}[!ht]
\vskip-0mm
\centering
\includegraphics{mppics/pic-1700}
\vskip-0mm
\end{figure}
После нескольких таких шагов каждый малый диск покроется полугеодезической картой.
В частности, \ref{eq:gb(n)} выполняется для каждого малого диска.
Затем, применив \ref{lem:GB-sum} несколько раз, получим \ref{eq:gb(n)} для исходного диска.

Остаётся показать, что $n=0$.
Предположим, что $\Delta$ лежит в локальной реализации нашей поверхности графиком $z = f(x,y)$.
Рассмотрим однопараметрическое семейство графов $z = t \cdot f(x,y)$;
обозначим через $\Delta_t$ соответствующий диск на $\Sigma_t$, так что $\Delta_1 = \Delta$, а $\Delta_0$ --- его проекция на плоскость $(x,y)$.
Функция $h \: t \mapsto \GB (\Delta_t )$ непрерывна.
Согласно \ref{eq:gb(n)}, $h(t)$ --- целое кратное $2 \cdot \pi$ для любого $t$.
Значит $h$ постоянна.
Следовательно,
\[\GB (\Delta ) = \GB (\Delta_0) = 0;\]
последнее равенство следует из \ref{prop:total-signed-curvature}.

Итак, мы доказали, что 
\[\GB (\Delta ) = 0
\eqlbl{eq:GB=0}\]
если $\Delta$ лежит на графике $z = f(x,y)$ в некоторой системе координат $(x,y,z)$.
Любая точка поверхности~$\Sigma$ имеет окрестность, которая покрывается таким графиком, и, применив аддитивность $\GB$ (\ref{lem:GB-sum}), как выше, получаем, что \ref{eq:GB=0} выполняется для любого диска~$\Delta$ на~$\Sigma$.
\qeds

\section{Сравнение Рауха}

Следующее предложение является частным случаем так называемой \index{теорема сравнения Рауха}\emph{теоремы сравнения Рауха}.

\begin{thm}{Предложение}\label{prop:rauch}
Пусть $p$ --- точка на гладкой поверхности $\Sigma$, и $r\le \inj(p)$.
Далее, пусть $\tilde\gamma$ --- кривая в $r$-окрестности нуля касательной плоскости $\T_p$,
и $\gamma$ --- кривая на~$\Sigma$, определяемая как
\[\gamma=\exp_p\circ\tilde\gamma,
\quad
\text{что эквивалентно}
\quad
\log_p\circ\gamma=\tilde\gamma.\]

\begin{subthm}{prop:rauch:K=<0}
Если поверхность $\Sigma$ имеет неположительную гауссову кривизну, то логарифмическое отображение $\log_p$ не увеличивает длину в $r$-окрестности точки $p$ на~$\Sigma$;
то есть
\[\length \gamma\ge \length \tilde\gamma\]
для любой кривой $\gamma$ в открытом шаре $B(p,r)_{\Sigma}$.
\end{subthm}

{\sloppy

\begin{subthm}{prop:rauch:K>=0}
Если поверхность $\Sigma$ имеет неотрицательную гауссову кривизну, то экспоненциальное отображение $\exp_p$ не увеличивает длину в $r$-окрестности нуля в $\T_p$;
то есть
\[\length \gamma\le \length \tilde\gamma\]
для любой кривой $\tilde\gamma$ в открытом шаре $B(0,r)_{\T_p}$.
\end{subthm}

}

\end{thm}

\parbf{Доказательство.}
Пусть $(r(t),\theta(t))$ --- полярные координаты $\tilde\gamma(t)$.
Тогда $(r(t),\theta(t))$ --- полярные координаты $\gamma(t)$ с началом в $p$ на~$\Sigma$;
то есть $\gamma(t)\z=s(r(t),\theta(t))$, где $s$ как в \ref{sec:Polar coordinates}.

Докажем, что $b(0,\theta)=0$ и $b_r(0,\theta)=1$ для любого $\theta$.
Можно предположить, что $\theta=0$.
Выберем стандартный базис $\vec v,\vec w$ в $\T_p$, с $\vec v$ направленным по кривой $t\mapsto s(t,0)$.
Заметим, что%
\footnote{Напомним, что $(D_{\vec w}\exp_p)(r\cdot \vec v)\df h'(0)$, где $h(t)=\exp_p(r\cdot \vec v+t\cdot \vec w)$; см.~\ref{sec:dirder}.}
\[b(r,0)=r\cdot |(D_{\vec w}\exp_p)(r\cdot \vec v)|.\]
В частности, $b(0,0)=0$.
Согласно \ref{obs:d(exp)=1}, $|(D_{\vec w}\exp_p)(0)|=1$,
и, значит, $|(D_{\vec w}\exp_p)(r\cdot \vec v)|\z\ne 0$ при малых $r$.
Отсюда следует, что функция $r\mapsto|(D_{\vec w}\exp_p)(r\cdot \vec v)|$ дифференцируема при $r=0$.
Взяв частную производную от выражения для $b(r,0)$, получаем $b_r(0,0)=1$.

Пусть $b(r,\theta)\df|s_\theta|$.
Согласно \ref{prop:jaccobi},
\[b_{rr}=-K\cdot b.\]
Если $K\ge 0$, то  для фиксированного $\theta$ функция $r\mapsto b(r,\theta)$ вогнута,
а если $K\le 0$, то $r\mapsto b(r,\theta)$ выпукла.
Поскольку $b(0,\theta)=0$ и $b_r(0,\theta)=1$,
\[
\begin{aligned}
b(r,\theta)\ge r\quad\text{если}\quad K&\le 0;
\\
b(r,\theta)\le r\quad\text{если}\quad K&\ge 0.
\end{aligned}
\eqlbl{eq:b-K}
\]

Можно считать, что $\tilde\gamma\:[a,b]\to \T_p$ параметризована длиной;
в частности, она липшицева.
Заметим, что
\begin{align*}
\length\tilde\gamma&=\int_a^b\sqrt{r'(t)^2+r(t)^2\cdot\theta'(t)^2}\cdot dt.
\shortintertext{Применив \ref{lem:palar-perp}, получим}
\length\gamma&=\int_a^b\sqrt{r'(t)^2+b(r(t),\theta(t))^2\cdot\theta'(t)^2}\cdot dt.
\end{align*}
А теперь утверждения в \ref{SHORT.prop:rauch:K=<0} и \ref{SHORT.prop:rauch:K>=0} следуют из \ref{eq:b-K}.
\qeds

\section{Внутренние изометрии}

Пусть $\Sigma$ и $\Sigma^{*}$ --- две гладкие поверхности.
Будем говорить, что отображение $f\:\Sigma\to \Sigma^{*}$ \index{сохраняющее длину}\emph{сохраняет длины}, если для любой кривой $\gamma$ на $\Sigma$ кривая $\gamma^{*}=f\circ\gamma$ на $\Sigma^{*}$ имеет ту же длину. 
Если $f$ ещё и гладкая и биекция, то она будет называться \index{внутренняя изометрия}\emph{внутренней изометрией}. 

Пример сохраняющего длину отображения можно получить, сворачивая плоскость в цилиндр при помощи отображения $s\:\mathbb{R}^2\to\mathbb{R}^3$, заданного как 
\[s(x,y)=(\cos x,\sin x,y).\]

\begin{thm}{Упражнение}\label{ex:K=0}
Пусть гауссова кривизна гладкой поверхности $\Sigma$ равна нулю.
Покажите, что $\Sigma$ является \index{локально плоская поверхность}\emph{локально плоской} поверхностью;
то есть некоторая окрестность любой точки на $\Sigma$ допускает внутреннюю изометрию на открытое множество евклидовой плоскости.  
\end{thm}

\begin{thm}{Упражнение}\label{ex:K=1}
Пусть гауссова кривизна гладкой поверхности $\Sigma$ равна 1 в каждой точке.
Покажите, что малая окрестность любой точки на $\Sigma$ допускает внутреннюю изометрию на открытое подмножество единичной сферы.
\end{thm}


\begin{thm}{Упражнение}\label{ex:deformation}
Для данного $a>0$, покажите, что существует гладкая кривая с единичной скоростью 
$\gamma(t)=(x(t),y(t))$, такая что $y(t) = a\cdot \cos t$ и $y>0$.
Найдите её интервал определения.

Пусть $\Sigma_a$ --- поверхность вращения кривой $\gamma$ вокруг оси $x$.
\begin{figure}[h!]
\vskip-0mm
\centering
\begin{lpic}[t(-0mm),b(6mm),r(0mm),l(0mm)]{asy/deformation(1.2)}
\lbl[t]{8,-.5;$a=2$}
\lbl[t]{24,3;$a=\sqrt{2}$}
\lbl[t]{41,4;$a=1$}
\lbl[t]{57,7;$a=\tfrac1{\sqrt{2}}$}
\lbl[t]{73,8;$a=\tfrac12$}
\end{lpic}
\vskip-0mm
\end{figure}
Покажите, что поверхность $\Sigma_a$ имеет единичную гауссову кривизну в каждой точке.

Воспользуйтесь этим построением и \ref{ex:K=1}, чтобы построить гладкую сохраняющую длину деформацию малого диска $\Delta$ на $\mathbb{S}^2$;
то есть однопараметрическое семейство $\Delta_t$ поверхностей с границей, для которого $\Delta_0=\Delta$ и $\Delta_t$ не конгруэнтна $\Delta_0$ для $t\ne0$.%
\footnote{На самом деле, любой диск на $\mathbb{S}^2$ допускает гладкую сохраняющую длину деформацию.
Однако если диск больше полусферы, то доказательство требует дополнительных усилий;
оно выводится из двух результатов Александра Александрова: теоремы о склеивании и теоремы о существовании выпуклой поверхности с абстрактно заданной метрикой \cite[с. 44]{pogorelov}.
}
\end{thm}

Следующее упражнение иллюстрирует заключительный шаг в доказательстве того, что \textit{любая открытая поверхность с нулевой гауссовой кривизной является цилиндрической поверхностью}.
См. обсуждение после \ref{ex:flat-plane}.

\begin{thm}{Продвинутое упражнение}\label{ex:line-cylinder} 
Пусть $(u,v)\mapsto f(u,v)$ задаёт внутреннюю изометрию из плоскости со стандартными $(u,v)$-координатами на поверхность $\Sigma$ в $\mathbb{R}^3$.
Предположим, что $f$ изометрически отображает $v$-ось на $z$-ось.
Покажите, что $\Sigma$ является цилиндрической поверхностью;
точнее, $\Sigma$ есть объединение семейства прямых, параллельных $z$-оси.
\end{thm}

\section{Замечательная теорема}

\begin{thm}{Теорема}\label{thm:remarkable}
Предположим, что $f\:\Sigma\to \Sigma^{*}$ является внутренней изометрией между двумя гладкими поверхностями; $p\in \Sigma$ и $p^{*}\z=f(p)\in \Sigma^{*}$.
Тогда 
\[K(p)_{\Sigma}=K(p^{*})_{\Sigma^{*}};\]
то есть в точке $p$ гауссова кривизна у $\Sigma$ та же, что в точке $p^{*}$ у~$\Sigma^{*}$.
\end{thm}

Напомним, что гауссова кривизна определяется как произведение главных кривизн, которые могут быть различными в точках $p$ и $p^*$; однако, согласно теореме, их произведения одинаковы.
Другими словами, гауссова кривизна является \textit{внутренним инвариантом}.
Эта теорема была доказана Карлом Фридрихом Гауссом \cite{gauss}, и справедливо названа {}\emph{замечательной} ({}\emph{Theorema Egregium}).

На самом деле, кривизну $K(p)$ можно получить из \textit{внутренних} измерений.
Например, она появляется в следующей формуле для длины окружности $c(r)$ геодезической окружности с центром в точке $p$ на поверхности: 
\[c(r)=2\cdot\pi\cdot r-\tfrac\pi3\cdot K(p)\cdot r^3+o(r^3).\]

Из теоремы следует, например, что не существует гладкого сохраняющего длину отображения, которое отправляет открытую область на единичной сфере в плоскость.%
\footnote{Гладкость существенна --- существует множество негладких отображений из сферы в плоскость сохраняющих длину; см. \cite{petrunin-yashinski} и ссылки в там.}
Это следует из того, что гауссова кривизна плоскости равна нулю, а гауссова кривизна единичной сферы равна 1.
В частности, любая географическая карта обязана иметь искажения.


\parbf{Доказательство.}
Выберем карту $(u,v)\mapsto s(u,v)$ на $\Sigma$ и пусть
$s^{*}\z=f\circ s$.
Заметим, что $s^{*}$ --- карта на $\Sigma^{*}$, и 
\begin{align*}
\langle s_u,s_u\rangle
&=
\langle s_u^{*}, s_u^{*}\rangle,
&
\langle s_u, s_v\rangle
&=
\langle s_u^{*}, s_v^{*}\rangle,
&
\langle s_v, s_v\rangle
&=
\langle s_v^{*}, s_v^{*}\rangle
\end{align*}
в любой точке $(u,v)$.
Действительно, поскольку $f$ сохраняет длины координатных линий $\gamma\:t\mapsto s(t,v)$ и  $\gamma\:t\z\mapsto s(u,t)$, мы получаем первое и третье равенства.
Теперь, поскольку $f$ сохраняет длины кривых $\gamma\:t\z\mapsto s(t,c-t)$ для любой константы~$c$, первое и третье равенства влекут второе.

Из \ref{prop:gamma''}, если $s$ --- полугеодезическая ката, то и $s^{*}$ таковая.
Остаётся применить \ref{prop:jaccobi} и \ref{ex:semigeodesc-chart}.
\qeds

