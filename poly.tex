\chapter{Ломаные}
\label{chap:poly}

Эта глава связывает кривизну кривой и углы вписанных в неё ломаных;
она поможет обзавестись правильной геометрической интуицией для кривизны.

\section{Кусочно-гладкие кривые}

\begin{wrapfigure}{o}{25 mm}
\vskip-0mm
\centering
\includegraphics{mppics/pic-54}
\end{wrapfigure}

Предположим, что $\alpha\:[a,b]\to \mathbb{R}^3$ и $\beta\:[b,c]\z\to \mathbb{R}^3$ --- такие две кривые, что $\alpha(b)\z=\beta(b)$.
Тогда их можно объединить в одну кривую $\gamma\:[a,c]\z\to \mathbb{R}^3$, определяемую как 
\[\gamma(t)=
\begin{cases}
\alpha(t)&\text{если}\quad t\le b,
\\
\beta(t)&\text{если}\quad t> b.
\end{cases}
\]
Кривая $\gamma$ называется 
\index{произведение кривых}\emph{произведением} $\alpha$ и $\beta$.
(Условие $\alpha(b)=\beta(b)$ обеспечивает непрерывность отображения $t\mapsto\gamma(t)$.)

То же определение можно применить, и если $\alpha$ и/или $\beta$ определены на полуоткрытых интервалах 
$(a,b]$ и/или $[b,c)$.

Предположение, что интервалы $[a,b]$ и $[b,c]$ продолжают друг друга, не существенно.
Достаточно чтобы конечная точка $\alpha$ совпадала с начальной точкой $\beta$.
Тогда интервалы можно подвинуть так, чтоб один продолжал другой. 

Если дополнительно $\beta(c)=\alpha(a)$, то можно рассмотреть циклическое произведение этих кривых;
так мы получим замкнутую кривую.

Если $\alpha'(b)$ и $\beta'(b)$ определены, то угол $\theta\z=\measuredangle(\alpha'(b),\beta'(b))$ называется \index{внешний угол}\emph{внешним углом} $\gamma$ при~$b$.
Если $\theta=\pi$, то говорят, что $\gamma$ имеет \index{точка!возврата}\emph{точку возврата} при~$b$.

Пространственная кривая $\gamma$ называется \index{кусочно-гладкая кривая}\emph{кусочно-гладкой}, если её можно представить как произведение конечного числа гладких кривых; если $\gamma$ замкнута, то предполагаем, что произведение циклическое.

Если $\gamma$ --- произведение гладких дуг $\gamma_1,\dots,\gamma_n$, то полная кривизна $\gamma$ определяется как сумма общих кривизн $\gamma_i$ и внешних углов;
то есть 
\[\tc\gamma=\tc{\gamma_1}+\dots+\tc{\gamma_n}+\theta_1+\dots+\theta_{n-1},\]
где $\theta_i$ --- внешний угол на стыке между $\gamma_i$ и $\gamma_{i+1}$.

Если же $\gamma$ замкнута, то её полная кривизна определяется как
\[\tc\gamma=\tc{\gamma_1}+\dots+\tc{\gamma_n}+\theta_1+\dots+\theta_{n},\]
где $\theta_n$ --- внешний угол на стыке между $\gamma_n$ и $\gamma_1$.

{

\begin{wrapfigure}{r}{23 mm}
\vskip-3mm
\centering
\includegraphics{mppics/pic-354}
\end{wrapfigure}

В частности, если $\gamma\:[a,b] \z\to \mathbb{R}^3$ --- гладкая петля, и $\hat\gamma$ соответствующая ей замкнутая кривая, то
\[\tc{\hat\gamma}\df\tc\gamma + \theta,\]
где $\theta=\measuredangle(\gamma'(a),\gamma'(b))$.

}


\section{Обобщённая теорема Фенхеля}

\begin{thm}{Теорема}\label{thm:gen-fenchel}
Пусть $\gamma$ --- замкнутая кусочно-гладкая пространственная кривая.
Тогда
\[\tc\gamma\ge2\cdot\pi.\]

\end{thm}

{\sloppy

\parbf{Доказательство.}
Пусть $\gamma$ --- циклическое произведение дуг $\gamma_1,\dots,\gamma_n$,
и $\theta_1,\dots,\theta_n$ --- её внешние углы.
Нам нужно показать, что \index{10phi@$\tc{\gamma}$ (полная кривизна)}
\[\tc{\gamma_1}+\dots+\tc{\gamma_n}+\theta_1+\dots+\theta_n\ge2\cdot\pi.\eqlbl{eq:gen-fenchel}\]

}

Рассмотрим касательную индикатрису $\tan_i$ для каждой дуги $\gamma_i$;
все они сферические дуги.

Рассуждение в доказательстве теоремы Фенхеля (\ref{thm:fenchel}), показывает, что все $\tan_1,\dots,\tan_n$ не могут лежать в открытой полусфере.

Сферическое расстояние от конечной точки $\tan_i$ до начальной точки $\tan_{i+1}$ равно внешнему углу $\theta_i$ (мы нумеруем дуги циклически, поэтому $\gamma_{n+1}=\gamma_1$).
Соединив конечную точку $\tan_i$ с начальной точкой $\tan_{i+1}$ короткой дугой большой окружности на сфере,
получаем замкнутую сферическую кривую, которая на $\theta_1+\dots+\theta_n$ длиннее общей длины $\tan_1,\dots,\tan_n$.

Применив к ней лемму о полусфере (\ref{lem:hemisphere}), получим
\[\length\tan_1+\dots+\length\tan_n+\theta_1+\dots+\theta_n\ge 2\cdot\pi.\]
Остаётся применить \ref{obs:tantrix}.
\qedsf

\begin{thm}{Лемма о хорде}\label{lem:chord}
Пусть $\gamma\:[a,b]\z\to\mathbb{R}^3$
--- гладкая дуга, и
$\ell$ --- её хорда.
Предположим, что $\gamma$ подходит к $\ell$ под углами $\alpha$ и $\beta$ в точках $\gamma(a)$ и $\gamma(b)$ соответственно;
то есть
\[\alpha=\measuredangle(\vec w,\gamma'(a))\quad\text{и}\quad \beta=\measuredangle(\vec w,\gamma'(b)),\]
где $\vec w=\gamma(b)-\gamma(a)$.
Тогда 
\[\tc\gamma\ge \alpha+\beta.\eqlbl{tc>a+b}\] 

\end{thm}

\parbf{Доказательство.}
Пропараметризуем хорду $\ell$ от $\gamma(b)$ до $\gamma(a)$, и
 пусть $\hat\gamma$ --- циклическое произведение $\gamma$ и $\ell$.
Замкнутая кривая $\hat\gamma$ имеет два внешних угла $\pi-\alpha$ и $\pi-\beta$.

\begin{wrapfigure}{r}{45 mm}
\vskip-5mm
\centering
\includegraphics{mppics/pic-53}
\vskip0mm
\end{wrapfigure}

Поскольку кривизна $\ell$ равна нулю, получаем 
\[\tc{\hat\gamma}=\tc\gamma+(\pi-\alpha)+(\pi-\beta).\]
По обобщённой теореме Фенхеля (\ref{thm:gen-fenchel}),
$\tc{\hat\gamma}\ge 2\cdot\pi$; отсюда \ref{tc>a+b}.
\qeds

\begin{thm}{Упражнение}\label{ex:chord-lemma-optimal}
Покажите, что оценка в лемме оптимальна.

Точнее, по данной паре различных точек $p, q$ и паре единичных векторов $\vec u,\vec v$ в $\mathbb{R}^3$,
постройте гладкую кривую $\gamma$, которая начинается в $p$ в направлении $\vec u$, заканчивается в $q$ в направлении $\vec v$, и при этом полная кривизна 
$\tc\gamma$ сколь угодно близка к $\measuredangle(\vec w,\vec u)+\measuredangle(\vec w,\vec v)$, где $\vec w$ --- вектор в направлении из $p$ в $q$.

\end{thm}

\section{Ломаные} 

Ломаные это частный случай кусочно-гладких кривых;
каждая дуга в их произведении является отрезком.
Поскольку кривизна отрезка равна нулю, полная кривизна ломаной есть сумма её внешних углов.

\begin{thm}{Упражнение}\label{ex:monotonic-tc}
Пусть $a$, $b$, $c$, $d$ и $x$ --- различные точки в $\mathbb{R}^3$.
Покажите, что полная кривизна ломаной $abcd$ не может превысить полную кривизну $abxcd$; то есть 
\[\tc {abcd} \le \tc {abxcd}.\]

Воспользуйтесь этим, чтобы показать, что кривизна любой замкнутой ломаной не меньше $2\cdot\pi$.
\end{thm}

\begin{thm}{Предложение}\label{prop:inscribed-total-curvature}
Пусть ломаная $p_0\dots p_n$ вписана в гладкую кривую~$\gamma$.
Тогда 
\[\tc\gamma\ge \tc{p_0\dots p_n}.\]
Более того, если $\gamma$ замкнута, то можно считать, что ломаная $p_0\dots p_n$ также замкнута.

\end{thm}



\begin{wrapfigure}[7]{o}{40 mm}
\vskip-4mm
\centering
\includegraphics{mppics/pic-55}
\vskip0mm
\end{wrapfigure}


\parbf{Доказательство.}
Будем считать, что $\gamma$ замкнута. 
Введём обозначения:
\begin{align*}
p_i&=\gamma(t_i),
&
\alpha_i&=\measuredangle(\vec w_i,\vec v_i),
\\
\vec w_i&=p_{i+1}-p_i,
& 
\beta_i&=\measuredangle(\vec w_{i-1},\vec v_i),
\\
\vec v_i&=\gamma'(t_i),
&
\theta_i&=\measuredangle(\vec w_{i-1},\vec w_i).
\end{align*}
Индексы считаем по модулю $n$;
так что $p_{n+1}\z=p_1$.

Поскольку кривизна отрезков равна нулю, 
полная кривизна ломаной $p_0\dots p_n$ равна сумме внешних углов $\theta_i$.

По неравенству треугольника для углов \ref{thm:spherical-triangle-inq}, 
\[\theta_i\le \alpha_i+\beta_i.\]
По лемме о хорде, полная кривизна дуги $\gamma$ от $p_i$ до $p_{i+1}$ не меньше $\alpha_i+\beta_{i+1}$. 
Таким образом, 
\begin{align*}
\tc{p_0\dots p_n}&=\theta_1+\dots+\theta_n\le 
\\
&\le\beta_1+\alpha_1+\dots+\beta_n+\alpha_n = 
\\
&=(\alpha_1+\beta_2)+\dots+(\alpha_n+\beta_1) \le 
\\
&\le \tc\gamma.
\end{align*}

Если $\gamma$ незамкнута, то вычисления аналогичны
\begin{align*}
\tc{p_0\dots p_n}&=\theta_1+\dots+\theta_{n-1}\le 
\\
&\le\beta_1+\alpha_1+\dots+\beta_{n-1}+\alpha_{n-1} \le
\\
&\le (\alpha_0+\beta_1)+\dots+(\alpha_{n-1}+\beta_n) \le 
\\
&\le \tc\gamma.
\end{align*}
\qedsf

\begin{thm}{Упражнение}\label{ex:sef-intersection}\label{ex:sef-intersection:>pi}  
Решите упражнение \ref{ex:length-dist:self-intersection:>pi}, используя \ref{prop:inscribed-total-curvature}.
\end{thm}

\begin{wrapfigure}{r}{30 mm}
\vskip-0mm
\centering
\includegraphics{mppics/pic-20}
\vskip0mm
\end{wrapfigure}

\begin{thm}{Упражнение}\label{ex:quadrisecant}
Замкнутая кривая $\gamma$ пересекает прямую в четырёх точках $a$, $b$, $c$ и~$d$.
Предположим, что эти точки появляются на прямой в порядке $a$, $b$, $c$, $d$, а на кривой $\gamma$ в порядке $a$, $c$, $b$,~$d$.
Покажите, что 
\[\tc\gamma\ge 4\cdot\pi.\]

\end{thm}

%Прямые, пересекающие как в упражнении, называются \index{альтерированная четырёхкратная секущая}\emph{альтерированными четырёхкратными секущими} кривой.
%Известно, что такие есть у любой нетривиального узла \cite{denne}.
%Значит, упражнение влечёт так называемую \emph{теорему Фари --- Милнора} --- \textit{полная кривизна любого узла превышает $4\cdot \pi$}; обзор \cite{petrunin-stadler} посвящён этой теореме.

\section[\texorpdfstring{А что если $\Phi(\gamma)=2\cdot \pi$?}{А что если Φ(γ)=2·π?}]{А что если $\bm{\Phi(\gamma)=2\cdot \pi}$?}

\begin{thm}{Предложение}\label{prop:fenchel=}
Случай равенства в теореме Фенхеля выполняется только для плоских выпуклых кривых;
то есть полная кривизна гладкой пространственной кривой $\gamma$ равна $2\cdot\pi$ тогда и только тогда, когда $\gamma$ является выпуклой плоской кривой.
\end{thm}

\parbf{Доказательство \ref{prop:fenchel=}.}
Достаточность содержится в \ref{cor:fenchel=convex};
остаётся доказать необходимость.

Пусть $abcd$ --- четырёхугольник вписанный в~$\gamma$.
По определению полной кривизны,
\begin{align*}
\tc{abcd}&=(\pi-
\measuredangle\hinge adb)+(\pi-
\measuredangle\hinge bac)+(\pi-
\measuredangle\hinge cbd)+(\pi-
\measuredangle\hinge dca)=
\\
&=4\cdot\pi -(
\measuredangle\hinge adb
+
\measuredangle\hinge bac
+
\measuredangle\hinge cbd
+
\measuredangle\hinge dca))
\end{align*}

По неравенству треугольника для углов (\ref{thm:spherical-triangle-inq}),
\[
\measuredangle\hinge bac
\le
\measuredangle\hinge bad
+ 
\measuredangle\hinge bdc
\quad\text{и}\quad
\measuredangle\hinge dca\le
\measuredangle\hinge dcb
+ 
\measuredangle\hinge dba.
\eqlbl{eq:spheric-triangle}
\]

\begin{wrapfigure}{r}{30 mm}
\vskip-5mm
\centering
\includegraphics{mppics/pic-56}
\vskip0mm
\end{wrapfigure}

Сумма углов в любом треугольнике равна $\pi$.
Значит из выше сказанного получаем
\begin{align*}
\tc{abcd}\ge 4\cdot \pi 
&- (\measuredangle\hinge adb+\measuredangle\hinge bad+ 
\measuredangle\hinge dba)-
\\
&-(\measuredangle\hinge cbd+\measuredangle\hinge dcb 
+\measuredangle\hinge  bdc)=
\\
=2\cdot\pi.&
\end{align*}

По \ref{prop:inscribed-total-curvature},
\[\tc{abcd}\le \tc\gamma\le 2\cdot\pi.\]
Следовательно, в \ref{eq:spheric-triangle} достигаются равенства.
Это означает, что точка $d$ принадлежит углу $abc$, 
а точка $b$ --- углу $cda$.
Последнее влечёт, что $abcd$ является выпуклым четырёхугольником на плоскости.

{\sloppy

То есть, любой четырёхугольник, вписанный в $\gamma$, выпуклый и плоский.
Таким образом, все точки $\gamma$ лежат в одной плоскости, определяемой тремя точками на~$\gamma$.
А поскольку любой четырёхугольник, вписанный в $\gamma$, выпуклый,
кривая $\gamma$ и сама выпукла. 
\qeds

}

\section{Обобщённая теорема о ДНК}\label{sec:DNA-poly}

\begin{thm}{Теорема}\label{thm:DNA-poly}
Пусть $p_1\dots p_n$ --- замкнутая ломаная в единичном шаре.
Тогда 
\[\tc{p_1\dots p_n}>\length(p_1\dots p_n).\]
\end{thm}

Согласно упражнению \ref{ex:total-curvature=}, эта теорема влечёт гладкий вариант теоремы (\ref{thm:DNA}).
То есть \ref{thm:DNA-poly} обобщает \ref{thm:DNA}.

\parbf{Доказательство.}
Будем предполагать, что $p_n=p_0$, $p_{n+1}=p_1$ и так далее.
Обозначим через $\theta_i$ внешний угол при $p_i$.

\begin{figure}[ht!]
\vskip-0mm
\centering
\includegraphics{mppics/pic-16}
\vskip0mm
\end{figure}

Пусть $o$ --- центр шара.
Рассмотрим последовательность треугольников 
\[\triangle q_0q_1s_0\cong \triangle p_0p_1o,\ \ \triangle q_1q_2s_1\cong \triangle p_1p_2o,\ \dots\]
таких, что точки $q_0,q_1,\dots$ лежат на одной прямой в том же порядке, и все точки $s_i$ лежат с одной стороны от этой прямой.

Заметим, что $s_0s_nq_nq_0$ --- параллелограмм, и, значит,
\[|s_n-s_0|=|q_n-q_0|=\length (p_1\dots p_n).\]
Следовательно, 
\[|s_0-s_1|+\dots+|s_{n-1}-s_n|\ge \length (p_1\dots p_n).\]

Далее, 
\[|q_i-s_{i-1}|=|q_i-s_i|=|p_i-o|\le 1\]
и
\[\theta_i\ge\measuredangle \hinge{q_i}{s_{i-1}}{s_i}\]
для каждого $i$.
Следовательно,
\[\theta_i>|s_{i-1}-s_i|\]
для каждого $i$.
Таким образом,
\begin{align*}
\tc {p_1\dots p_n}
&=\theta_1+\dots+\theta_n>
\\
&> |s_{0}-s_1|+\dots +|s_{n-1}-s_n|\ge 
\\
&\ge\length (p_1\dots p_n).
\end{align*}
\qedsf

Упомянем ещё одно обобщение теоремы о ДНК;
оно было получено Джеффри Лагариасом и Томасом Ричардсоном \cite{lagarias-richardso}; другое доказательство нашли Александр Назаров и Фёдор Петров \cite{nazarov-petrov}.
Оба доказательства до обидного сложны.

\begin{thm}{Теорема}
Предположим, что $\alpha$ --- замкнутая кривая, лежащая в выпуклой фигуре на плоскости, ограниченной кривой $\gamma$.
Тогда средняя кривизна $\alpha$ не меньше средней кривизны $\gamma$.

\end{thm}

\section{Обобщённая кривизна}

Следующее упражнение говорит, что неравенство в \ref{prop:inscribed-total-curvature} оптимально.

\begin{thm}{Упражнение}\label{ex:total-curvature=}
Покажите, что для любой гладкой пространственной кривой $\gamma$ выполнено равенство
\[\tc\gamma=\sup\{\tc\beta\},\]
где точная верхняя грань берётся по всем ломаным $\beta$, вписанным в $\gamma$
(если $\gamma$ замкнута, то предполагаем, что и $\beta$ замкнута).
\end{thm}

Равенство в упражнении используют в определении полной кривизны произвольной кривой~$\gamma$ --- её определяют как \textit{точную верхнюю грань полной кривизны невырожденных ломаных, вписанных в~$\gamma$.}

Эта теория была переоткрыта несколько раз; см. \cite[III §~1]{pogorelov}, \cite{aleksandrov-reshetnyak} и \cite{sullivan-curves}.
Большинство утверждений этой главы можно распространить на кривые конечной полной кривизны в этом обобщённом смысле.

\begin{thm}{Упражнение}\label{ex:tc-rectifiable}
Предположим, что полная кривизна кривой $\gamma\:[0,1]\to\mathbb{R}^3$ ограничена в обобщённом смысле;
то есть существует верхняя грань на полные кривизны ломаных, вписанных в~$\gamma$.

Покажите, что $\gamma$ спрямляема.
Постройте пример, показывающий, что обратное не верно. 
\end{thm}
