\chapter{Кручение}
\label{chap:torsion}

Эта глава в основном предоставляет практику в вычислениях.
За исключением определений в разделе~\ref{sec:frenet-frame}, она не будет использоваться в дальнейшем.

Так же как кривизна показывает, насколько кривая отклоняется от прямой линии, кручение показывает, насколько пространственная кривая отклоняется от плоской (см. \ref{ex:lancret}).

\section{Базис Френе}\label{sec:frenet-frame}

Пусть $\gamma$ --- гладкая пространственная кривая.
Будем считать, что $\gamma$ параметризована длиной,
так что вектор скорости $\tan(s)\z=\gamma'(s)$ единичный.

Предположим, что $\gamma$ имеет ненулевую кривизну в момент времени~$s$;
другими словами, $\gamma''(s)\z\ne 0$.
Тогда \index{нормальный!вектор}\emph{нормальный вектор} при $s$ определяется как
\[\norm(s)=\frac{\gamma''(s)}{|\gamma''(s)|}.\]
Заметим, что \index{10tnb@$\tan$, $\norm$, $\bi$ (базис Френе)}
\[\tan'(s)=\gamma''(s)=\kur(s)\cdot\norm(s).\]

Согласно \ref{prop:a'-pertp-a''}, $\norm(s)\perp \tan(s)$.
Поэтому векторное произведение 
\[\bi(s)=\tan(s)\times \norm(s)\]
является единичным вектором.
Более того, тройка векторов $\tan(s)$, $\norm(s)$, $\bi(s)$ образует ориентированный ортонормированный базис в $\mathbb{R}^3$;
он называется \index{базис Френе}\emph{базис Френе} кривой $\gamma$ при~$s$.
В частности, 
\[\begin{aligned}
\langle\tan,\tan\rangle&=1,
&
\langle\norm,\norm\rangle&=1,
&\langle\bi,\bi\rangle&=1,
\\
\langle\tan,\norm\rangle&=0,
&
\langle\norm,\bi\rangle&=0,
&
\langle\bi,\tan\rangle&=0.
\end{aligned}
\eqlbl{eq:orthornomal}
\]
Векторы $\tan(s)$, $\norm(s)$ и $\bi(s)$ называются \index{касательный}\emph{касательным}, \index{нормальный}\emph{нормальным} и \index{бинормальный}\emph{бинормальным} базиса Френе, соответственно.
Отметим, что базис Френе определён только если $\kur(s)\z\ne 0$.

Плоскость $\Pi_s$ через $\gamma(s)$, натянутая на $\tan(s)$ и $\norm(s)$, называется \index{соприкасающаяся!плоскость}\emph{соприкасающейся плоскостью} при $s$;
она может определяться и как плоскость через $\gamma(s)$, перпендикулярная бинормальному вектору $\bi(s)$.
Это единственная плоскость, имеющая \index{порядок касания}\emph{второй порядок касания} с $\gamma$ при $s$;
то есть $\rho(\ell)=o(\ell^2)$, где $\rho(\ell)$ --- расстояние от $\gamma(s+\ell)$ до~$\Pi_s$.


\section{Кручение}

Пусть $\gamma$ --- гладкая пространственная кривая с единичной скоростью,
и $\tan,\norm,\bi$ --- её базис Френе.
Величина \index{10tau@$\tor$ (кручение)}
\[\tor(s)=\langle \norm'(s),\bi(s)\rangle\]
называется \index{кручение}\emph{кручением} $\gamma$ при~$s$.

Отметим, что кручение $\tor(s_0)$ определено, только если $\kur(s_0)\z\ne0$.
Действительно, поскольку функция $s\mapsto \kur(s)$ является непрерывной, 
$\kur(s_0)\z\ne 0$ означает, что $\kur(s)\z\ne 0$ для всех $s$ около $s_0$.
Следовательно, базис Френе также определён на открытом интервале, содержащем~$s_0$.
Ясно, что $\tan(s)$, $\norm(s)$ и $\bi(s)$ гладко зависят от $s$ в своих областях определения.
Поэтому производная $\norm'(s_0)$ определена, а, значит, определено и кручение.

\begin{thm}{Упражнение}\label{ex:helix-torsion}
По данным $a$ и $b$,
вычислите кривизну и кручение винтовой линии
$\gamma_{a,b}(t)=(a\cdot \cos t,a\cdot\sin t, b\cdot t)$.

Выведите отсюда, что для данных $\kur>0$ и $\tor$
существует винтовая линия с постоянной кривизной $\kur$ и постоянным кручением~$\tor$.
\end{thm}



\section{Формулы Френе}

Предположим, что базис Френе $\tan(s),\norm(s),\bi(s)$ кривой $\gamma$ определён в точке~$s$.
Напомним, что 
\[\tan'=\kur\cdot \norm.
\eqlbl{eq:frenet-tau}\]
Давайте выразим оставшиеся производные $\norm'$ и $\bi'$ в базисе $\tan,\norm,\bi$.

Сначала покажем, что
\[\norm'=-\kur\cdot\tan+\tor\cdot\bi.\eqlbl{eq:frenet-nu}\]

Действительно, поскольку базис $\tan,\norm,\bi$ ортонормирован, приведённая выше формула эквивалентна следующим трём тождествам:
\[\begin{aligned}
\langle \norm',\tan\rangle&=-\kur,
&
\langle \norm',\norm\rangle&=0,
&
\langle \norm',\bi\rangle&=\tor,
\end{aligned}\eqlbl{eq:<N',?>}\]
Последнее тождество следует из определения кручения.
Второе следует из тождества $\langle \norm,\norm\rangle\z=1$ в \ref{eq:orthornomal}. 
Продифференцировав тождество $\langle\tan,\norm\rangle\z=0$ в \ref{eq:orthornomal}, получим 
\[\langle\tan',\norm\rangle+\langle\tan,\norm'\rangle=0.\]
Применив \ref{eq:frenet-tau}, получим первое тождество в \ref{eq:<N',?>}.

Дифференцируя третье тождество в \ref{eq:orthornomal}, получаем, что $\bi'\perp\bi$.
Продифференцировав остальные две тождества с $\bi$ в \ref{eq:orthornomal}, получим
\begin{align*}
\langle\bi',\tan\rangle&=-\langle\bi,\tan'\rangle=-\kur\cdot\langle\bi,\norm\rangle=0,
\\
\langle\bi',\norm\rangle&=-\langle\bi,\norm'\rangle=\tor.
\end{align*}
И поскольку базис $\tan,\norm,\bi$ ортонормирован, получаем, что
\[\bi'=-\tor\cdot\norm.\eqlbl{eq:frenet-beta}\]

Уравнения \ref{eq:frenet-tau}, \ref{eq:frenet-nu}, и \ref{eq:frenet-beta} называются \index{формулы Френе}\emph{формулами Френе}.
Все три можно записать в виде одного матричного тождества:
\[
\begin{pmatrix}
\tan'
\\
\norm'
\\
\bi'
\end{pmatrix}
=
\begin{pmatrix}
0&\kur&0
\\
-\kur&0&\tor
\\
0&-\tor&0
\end{pmatrix}
\cdot
\begin{pmatrix}
\tan
\\
\norm
\\
\bi
\end{pmatrix}.
\]

Напомним, что бинормаль $\bi$ ортогональна соприкасающейся плоскости.
Поэтому уравнение \ref{eq:frenet-beta} означает, что кручение мерит, как быстро крутится соприкасающаяся плоскость при движении вдоль~$\gamma$.

\begin{thm}{Упражнение}\label{ex:beta-from-tau+nu}
Выведите формулу \ref{eq:frenet-beta} из \ref{eq:frenet-tau} и \ref{eq:frenet-nu}, дифференцируя тождество
$\bi=\tan\times \norm$.
\end{thm}

\begin{thm}{Упражнение}\label{ex:torsion=0}
Пусть $\gamma$ --- гладкая пространственная кривая с ненулевой кривизной.
Докажите, что $\gamma$ лежит в плоскости тогда и только тогда, когда её кручение тождественно равно нулю.
\end{thm}

\begin{thm}{Упражнение}\label{ex:+B}
Пусть $\gamma_0\:[a,b]\to \mathbb{R}^3$ --- гладкая пространственная кривая, и $\tan,\norm,\bi$ --- её базис Френе.
Рассмотрим кривую $\gamma_1(t)\z\df\gamma_0(t)\z+\bi(t)$.
Докажите, что
\[\length\gamma_1\ge\length\gamma_0.\]
\end{thm}

\begin{thm}{Упражнение}\label{ex:frenet}
Пусть $\gamma$ --- гладкая пространственная кривая,
$\tan,\norm,\bi$ --- её базис Френе, и $\tor$ --- её кручение.
Выведите равенства
\[\bi=\frac{\gamma'\times\gamma''}{|\gamma'\times\gamma''|}
\quad\text{и}\quad
\tor=\frac{\langle\gamma'\times\gamma'',\gamma'''\rangle}{|\gamma'\times\gamma''|^2}.
\]

\end{thm}

\begin{thm}{Упражнение}\label{ex:moment-curve}
Найдите кривизну $\kur(t)$ и кручение $\tor(t)$ \index{кривая моментнов}\emph{кривой моментнов} $\gamma\:t\z\mapsto (t,t^2,t^3)$ в точке $\gamma(t)$.
\end{thm}

Следующее упражнение тесно связано с леммой о луке (\ref{lem:bow}).

\begin{thm}{Упражнение}\label{ex:bow-converse}
Пусть $\gamma_1,\gamma_2\:[a,b]\to\mathbb{R}^3$ --- две гладкие кривые с единичной скоростью.
Предположим, что 
\[\dist{\gamma_1(t_1)}{\gamma_1(t_2)}{}\ge \dist{\gamma_2(t_1)}{\gamma_2(t_2)}{}\]
для любых $t_1$ и $t_2$.
Покажите, что $\kur(t_0)_{\gamma_1}\le \kur(t_0)_{\gamma_2}$ для любого $t_0$.
\end{thm}


\begin{thm}{Продвинутое упражнение}\label{ex:torsion-indicatrix}
Пусть $\gamma$ --- замкнутая гладкая пространственная кривая с положительным кручением.
Докажите, что её касательная индикатриса имеет самопересечение.
\end{thm}


\section{Линии откоса}

Гладкая пространственная кривая $\gamma$ называется \index{линия откоса}\emph{линией откоса}, если она идёт под постоянным углом к фиксированному направлению.
Следующую теорему доказал Мишель Анж Ланкре~\cite{lancret}.

\begin{thm}{Теорема}\label{thm:const-slope}
Пусть $\gamma$ --- гладкая кривая с кривизной и кручением $\kur$ и $\tor$.
Предположим, что $\kur(s)>0$ для всех~$s$.
Тогда $\gamma$ --- линия откоса тогда и только тогда, когда отношение $\tfrac\tor\kur$ постоянно.
\end{thm}


\begin{thm}{Доказательство и упражнение}\label{ex:lancret}
{\sloppy
Пусть $\gamma$ --- гладкая пространственная кривая с ненулевой кривизной, $\tan,\norm,\bi$ --- её базис Френе, а $\kur$ и $\tor$ --- её кривизна и кручение.

}

\begin{subthm}{ex:lancret:a}
Предположим, что для некоторого фиксированного ненулевого вектора $\vec w$ значение $\langle \vec w,\tan\rangle$ постоянно.
Докажите, что $\langle \vec w, \norm\rangle \z=0$.
Выведите, что $\langle \vec w,\bi\rangle$ постоянно.
Докажите, что \[\tor\cdot\langle \vec w,\bi\rangle -\kur\cdot\langle \vec w,\tan\rangle =0.\]
Выведите отсюда, что отношение $\tfrac\tor\kur$ постоянно.
\end{subthm}

{\sloppy

\begin{subthm}{ex:lancret:b}
Предположим, что отношение $\tfrac\tor\kur$ не меняется.
Докажите, что вектор $\vec w\z=\tfrac\tor\kur\cdot \tan+\bi$ также не меняется.
Выведите отсюда, что $\gamma$ --- линия откоса.
\end{subthm}

}

\end{thm}

Пусть $\gamma$ --- гладкая кривая с единичной скоростью и $s_0$ --- фиксированное значение её параметра. 
Тогда кривая 
\[\alpha(s)=\gamma(s)+(s_0-s)\cdot \gamma'(s)\]
называется \index{эвольвента}\emph{эвольвеной}~$\gamma$.
Заметим, что если $\ell(s)$ обозначает касательную к $\gamma$ в точке $s$,
то $\alpha(s)\in \ell(s)$ и $\alpha'(s)\perp \ell$ для всех~$s$.

\begin{thm}{Упражнение}\label{ex:evolvent-constant-slope}
Докажите, что эвольвента линии откоса лежит на плоскости.
\end{thm}

\section{Сферические кривые}

\begin{thm}{Теорема}
Пусть $\gamma$ --- гладкая пространственная кривая с ненулевым кручением $\tor$ (и следовательно с ненулевой кривизной $\kur$).
Тогда $\gamma$ лежит в единичной сфере тогда и только тогда, когда выполняется тождество
\[\left|\frac{\kur'}{\tor}\right|=\kur\cdot\sqrt{\kur^2-1}.\]
\end{thm}

\begin{thm}{Доказательство и упражнение}\label{ex:spherical-frenet}
Предположим, что $\gamma$ --- гладкая пространственная кривая с единичной скоростью,
$\tan,\norm,\bi$ --- её базис Френе,
и $\kur$, $\tor$ --- её кривизна и кручение.

\smallskip

Предположим, что $\gamma$ сферическая; то есть $|\gamma(s)|=1$ для любого~$s$.
Покажите, что

\begin{subthm}{ex:spherical-frenet:tau} $\langle\tan,\gamma\rangle=0$ и $\langle\norm,\gamma\rangle^2+\langle\bi,\gamma\rangle^2=1$.
\end{subthm}

\begin{subthm}{ex:spherical-frenet:nu} $\langle\norm,\gamma\rangle=-\tfrac1\kur$.
\end{subthm}

\begin{subthm}{ex:spherical-frenet:beta} $\langle\bi,\gamma\rangle'=\tfrac\tor\kur$.
\end{subthm}

\begin{subthm}{ex:spherical-frenet:beta+}
Воспользовавшись \ref{SHORT.ex:spherical-frenet:beta}, докажите, что если $\gamma$ замкнута, то $\tor(s)=0$ для некоторого~$s$.
\end{subthm}

\begin{subthm}{ex:spherical-frenet:kur-tor} Предположим, что кручение $\gamma$ не равно нулю.
Используйте \ref{SHORT.ex:spherical-frenet:tau}--\ref{SHORT.ex:spherical-frenet:beta}, чтобы показать, что
\[\left|\frac{\kur'}{\tor}\right|=\kur\cdot\sqrt{\kur^2-1}.\]
\end{subthm}
Пусть теперь $\gamma$ удовлетворяет тождеству в \ref{SHORT.ex:spherical-frenet:kur-tor}.
\begin{subthm}{ex:spherical-frenet:f}
Покажите, что точка $p=\gamma+\tfrac1\kur\cdot \norm+\tfrac{\kur'}{\kur^2\cdot\tor}\cdot\bi$ не меняется.
Выведите, что $\gamma$ лежит на единичной сфере с центром в~$p$.
\end{subthm}

\end{thm}

Для кривой $\gamma$ с единичной скоростью и ненулевой кривизной и кручением при~$s$,
сфера $\Sigma_s$, проходящая через $\gamma(s)$ с центром в
\[p(s)=\gamma(s)+\tfrac1{\kur(s)}\cdot \norm(s)+\tfrac{\kur'(s)}{\kur^2(s)\cdot\tor(s)}\cdot\bi(s)\]
называется \index{соприкасающаяся!сфера}\emph{соприкасающейся сферой} кривой $\gamma$ при~$s$.
Это единственная сфера, которая имеет \index{порядок касания}\emph{касание третьего порядка} с $\gamma$ при~$s$;
то есть $\rho(\ell)=o(\ell^3)$, где $\rho(\ell)$ обозначает расстояние от $\gamma(s+\ell)$ до $\Sigma_s$.
 
\section{Основная теорема}

\begin{thm}{Теорема}\label{thm:fund-curves}
Пусть $s\mapsto \kur(s)$ и $s\mapsto \tor(s)$ --- две гладкие функции, определённые на интервале $\mathbb{I}$.
Предположим, что $\kur(s)>0$ для всех~$s$.
Тогда существует гладкая кривая $\gamma\:\mathbb{I}\to\mathbb{R}^3$ с единичной скоростью, кривизной $\kur(s)$ и кручением $\tor(s)$ при любом~$s\in \mathbb{I}$.
Более того, $\gamma$ однозначно определена с точностью до движения пространства, сохраняющего ориентацию.
\end{thm}

В доказательстве является применяется теорема о существовании и единственности решений дифференциальных уравнений (\ref{thm:ODE}).

\parbf{Доказательство.}
Выберем значение параметра $s_0$, точку $\gamma(s_0)$ и ориентированный ортонормированный базис $\tan(s_0)$, $\norm(s_0)$, $\bi(s_0)$.
Рассмотрим следующую систему дифференциальных уравнений
\[
\begin{cases}
\gamma'=\tan,
\\
\tan'=\kur\cdot\norm,
\\
\norm'=-\kur\cdot\tan+\tor\cdot\bi,
\\
\bi'=-\tor\cdot\norm
\end{cases}
\eqlbl{eq:gamma'tan'norm'bi'}
\]
с начальными данными $\gamma(s_0)$, $\tan(s_0)$, $\norm(s_0)$, $\bi(s_0)$.
(Наша система уравнений состоит из четырёх векторных уравнений, так что её можно переписать как систему из 12-и скалярных.)

Согласно \ref{thm:ODE}, эта система имеет единственное решение, которое определено на максимальном подинтервале $\mathbb{J}\subset \mathbb{I}$, содержащем $s_0$.
Покажем, что $\mathbb{J}=\mathbb{I}$.

Сначала заметим, что 
\[\begin{aligned}
\langle\tan,\tan\rangle&=1,
&
\langle\norm,\norm\rangle&=1,
&
\langle\bi,\bi\rangle&=1,
\\
\langle\tan,\norm\rangle&=0,&
\langle\tan,\norm\rangle&=0,&
\langle\bi,\tan\rangle&=0
\end{aligned}
\eqlbl{eq:111000}
\]
для любого значения параметра $s$.

Действительно, из \ref{eq:gamma'tan'norm'bi'} получаем следующую систему:
\[
\begin{cases}
\langle\tan,\tan\rangle'
&=
2\cdot\langle\tan,\tan'\rangle
=
2\cdot\kur\cdot \langle\tan,\norm\rangle,
\\
\langle\norm,\norm\rangle'
&=
2\cdot\langle\norm,\norm'\rangle
=
-
2\cdot\kur\cdot\langle\norm,\tan\rangle
+
2\cdot\tor\cdot\langle\norm,\bi\rangle,
\\
\langle\bi,\bi\rangle'
&=
2\cdot\langle\bi,\bi'\rangle
=
-2\cdot\tor\langle\bi,\norm\rangle,
\\
\langle\tan,\norm\rangle'
&=
\langle\tan',\norm\rangle
+
\langle\tan,\norm'\rangle
=
\kur\cdot\langle\norm,\norm\rangle
-
\kur\cdot\langle\tan,\tan\rangle
+
\tor\cdot\langle\tan,\bi\rangle,
\\
\langle\norm,\bi\rangle'
&=
\langle\norm',\bi\rangle+\langle\norm,\bi'\rangle
=\kur\cdot\langle\tan,\bi\rangle+\tor\cdot\langle\bi,\bi\rangle-\tor\cdot\langle\norm,\norm\rangle,
\\
\langle\bi,\tan\rangle'
&=
\langle\bi',\tan\rangle+\langle\bi,\tan'\rangle
=
-\tor\cdot \langle\norm,\tan\rangle
+\kur\cdot\langle\bi,\norm\rangle.
\end{cases}
\eqlbl{eq:<gamma'tan'norm'bi'>}
\]

Константы в \ref{eq:111000} удовлетворяют этой системе.
Более того, поскольку базис $\tan(s_0)$, $\norm(s_0)$, $\bi(s_0)$ ориентирован и ортонормирован,
\ref{eq:111000} решает нашу задачу Коши для системы \ref{eq:<gamma'tan'norm'bi'>}.

Допустим, что $\mathbb{J} \varsubsetneq \mathbb{I}$.
Тогда один из концов $\mathbb{J}$, скажем $b$, лежит во внутренней части $\mathbb{I}$.
Теорема \ref{thm:ODE} применима для $\Omega=\mathbb{R}^{12}\times \mathbb{I}$.
Значит, одно из значений $\gamma(s)$, $\tan(s)$, $\norm(s)$, $\bi(s)$
устремляется к бесконечности при $s\to b$.
Но это невозможно --- векторы $\tan(s)$, $\norm(s)$, $\bi(s)$ остаются единичными, и $|\gamma'(s)|=|\tan(s)|=1$;
так что $\gamma$ проходит лишь конечное расстояние при $s\to b$ --- противоречие.
Таким образом, $\mathbb{J}= \mathbb{I}$, и первая часть теоремы доказана.

Теперь предположим, что есть две кривые $\gamma_1$ и $\gamma_2$ с заданными функциями кривизны и кручения.
Применив движение пространства, можно добиться чтобы $\gamma_1(s_0)=\gamma_2(s_0)$ и базисы Френе кривых совпали при $s_0$.
Тогда $\gamma_1=\gamma_2$ по единственности решений системы (\ref{thm:ODE}), что завершает доказательство.
\qeds


\begin{thm}{Упражнение}\label{ex:cur+tor=helix}
Предположим, что кривая $\gamma\:\mathbb{R}\to\mathbb{R}^3$ с постоянной скоростью, кривизной и кручением.
Покажите, что $\gamma$ является винтовой линией, возможно, вырождающейся в окружность;
то есть в подходящей системе координат 
$\gamma(t)=(a\cdot \cos t,a\cdot\sin t, b\cdot t)$
для некоторых констант $a$ и~$b$.
\end{thm}


\begin{thm}{Продвинутое упражнение}\label{ex:const-dist}
Пусть $\gamma$ --- гладкая пространственная кривая такая, что расстояние $|\gamma(t)-\gamma(t+\ell)|$ зависит только от $\ell$.
Покажите, что $\gamma$ является винтовой линией, возможно, вырождающуюся в прямую или окружность.
\end{thm}




