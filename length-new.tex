\chapter{Длина}
\label{chap:length}

\section{Определения}

Напомним, что последовательность вида 
\[a=t_0 < t_1 < \cdots < t_k=b.\]
называется \index{разбиение}\emph{разбиением} интервала $[a,b]$.

\begin{thm}{Определение}\label{def:length}
Пусть $\gamma\:[a,b]\to \spc{X}$ --- кривая в метрическом пространстве.
\index{длина кривой}\emph{Длина} $\gamma$ определяется как
\begin{align*}
\length \gamma
&= 
\sup
\set{\dist{\gamma(t_0)}{\gamma(t_1)}{\spc{X}}
+\dots+
\dist{\gamma(t_{k-1})}{\gamma(t_k)}{\spc{X}}}{},
\end{align*}
где верхний предел берётся по всем разбиениям $t_0,\dots,t_k$ интервала $[a,b]$.

Длина замкнутой кривой определяется как длина соответствующей петли.
Если кривая параметризована открытым или полуоткрытым интервалом, то ее длина определяется как верхний предел длин всех её сужений на замкнутые интервалы.
\end{thm}

Кривая конечной длины называется \index{спрямляемая кривая}\emph{спрямляемой}.

\begin{thm}{Упражнение}\label{ex:integral-length-0}
Предположим, что $\gamma_1\:[a_1,b_1] \to\mathbb{R}^3$ есть репараметризация кривой $\gamma_2\:[a_2,b_2] \to\mathbb{R}^3$. 
Покажите, что
\[\length \gamma_1 = \length \gamma_2.\]
\end{thm}

\begin{wrapfigure}[4]{r}{33 mm}
\vskip-4mm
\centering
\includegraphics{mppics/pic-224}
\end{wrapfigure}

Пусть $\gamma\:[a,b]\to \mathbb{R}^3$ --- параметризованная пространственная кривая.
Выберем разбиение $a\z=t_0 < t_1 < \cdots < t_k=b$, и пусть $p_i=\gamma(t_i)$.
Тогда ломаная $p_0\dots p_k$ называется \index{вписанная ломаная}\emph{вписанной} в~$\gamma$.
Если $\gamma$ замкнута, то $p_0=p_k$, и, значит, её вписанная ломаная также замкнута.

Отметим, что длину пространственной кривой можно определить как верхний предел длин вписанных в неё ломаных.

{\sloppy

\begin{thm}{Упражнение}\label{ex:length-chain}
Рассмотрим путь $\gamma\:[0,1]\to\mathbb{R}^3$.
Предположим, что $\beta_n$ --- последовательность вписанных в $\gamma$ ломаных с вершинами $\gamma(\tfrac in)$ для $i\z\in\{0,\dots,n\}$.
Докажите, что 
\[\length\beta_n\to\length\gamma
\quad\text{при}\quad
n\to \infty.
\]
\end{thm}

}

\begin{thm}{Упражнение}\label{ex:length-image}
Рассмотрим простой путь $\gamma\:[0,1]\to\mathbb{R}^3$.
Предположим, что путь $\beta\:[0,1]\to\mathbb{R}^3$ имеет то же образ, что и $\gamma$;
то есть $\beta([0,1])=\gamma([0,1])$.
Докажите, что 
\[\length \beta\ge \length \gamma.\]

Попробуйте доказать это неравенство, ослабив предположение до $\beta([0,1])\z\supset\gamma([0,1])$.
\end{thm}

\begin{thm}{Упражнение}\label{ex:integral-length}
Пусть $\gamma\:[a,b]\to\mathbb{R}^3$ --- гладкая кривая.
Докажите, что
\vskip1mm
\begin{minipage}{.45\textwidth}
\begin{subthm}{ex:integral-length>}
$\length \gamma
\ge
\int_a^b|\gamma'(t)|\cdot dt$;
\end{subthm}
\end{minipage}
\hfill
\begin{minipage}{.45\textwidth}
\begin{subthm}{ex:integral-length<}
$\length \gamma
\le
\int_a^b|\gamma'(t)|\cdot dt$.
\end{subthm}
\end{minipage}
\vskip1mm

В частности 
\[\length \gamma
=
\int_a^b|\gamma'(t)|\cdot dt.\eqlbl{eq:length}\]

\end{thm}

\begin{thm}{Продвинутые упражнения}\label{adex:integral-length}

\begin{subthm}{adex:integral-length:a}
Докажите, что формула \ref{eq:length} справедлива для любой липшицевой кривой $\gamma\:[a,b]\z\to\mathbb{R}^3$.
\end{subthm}

\begin{subthm}{adex:integral-length:b}
Постройте такую непостоянную кривую $\gamma\:[a,b]\to\mathbb{R}^3$, что $\gamma'(t)=0$ почти всюду.
(Для такой кривой, не выполняется равенство \ref{eq:length}, хотя обе его стороны определены.)
\end{subthm}

\end{thm}

\section{Неспрямляемые кривые}

Опишем так называемую \index{снежинка Коха}\emph{снежинку Коха} ---
классический пример неспрямляемой кривой.

Начнём с равностороннего треугольника.
Разделим каждую сторону на три равные отрезка, и добавим равносторонний треугольник с основанием на среднем отрезке.
К полученному многоугольнику применим ту же операцию и так будем продолжать рекурсивно.
Снежинка Коха --- это граница объединения всех, полученных так, многоугольников.
Две итерации и получившаяся снежинка Коха показаны на рисунке.

\begin{figure}[ht!]
\centering
\includegraphics{mppics/pic-225}
\end{figure}

\begin{thm}{Упражнение}\label{ex:nonrectifiable-curve}

\begin{subthm}{ex:nonrectifiable-curve:a}
Докажите, что снежинка Коха --- простая замкнутая кривая; в частности её можно параметризовать окружностью.
\end{subthm}

\begin{subthm}{ex:nonrectifiable-curve:b}
Докажите, что снежинка Коха не спрямляема. 
\end{subthm}
\end{thm}

\section{Полунепрерывность длины}

Нижний предел последовательности $x_n$ будет обозначаться как
\[\liminf_{n\to\infty} x_n.\] 
Он определяется как наименьший частичный предел; то есть наименьший возможный предел подпоследовательности $x_n$.
Нижний предел определён для любой последовательности вещественных чисел и принимает значения в расширенной числовой прямой $[-\infty,\infty]$.

{\sloppy

\begin{thm}{Теорема}
Предположим, что последовательность кривых $\gamma_n\:[a,b]\to \spc{X}$ в метрическом пространстве $\spc{X}$ сходится поточечно к кривой $\gamma_\infty\:[a,b]\to \spc{X}$;
то есть $\gamma_n(t)\z\to\gamma_\infty(t)$
при любом $t \in [a,b]$ и $n\to\infty$. 
Тогда 
$$\liminf_{n\to\infty} \length\gamma_n \ge \length\gamma_\infty.\eqlbl{eq:semicont-length}$$
\end{thm}

}

\begin{thm}{Следствие}\label{thm:length-semicont}
Длина полунепрерывна снизу по отношению к поточечной сходимости кривых. 
\end{thm}

\parbf{Доказательство.}
Выберем разбиение $a=t_0<t_1<\dots<t_k=b$.
Пусть
\begin{align*}\Sigma_n
&\df
\dist{\gamma_n(t_0)}{\gamma_n(t_1)}{}
+\dots+
\dist{\gamma_n(t_{k-1})}{\gamma_n(t_k)}{},
\\
\Sigma_\infty
&\df
\dist{\gamma_\infty(t_0)}{\gamma_\infty(t_1)}{}
+\dots+
\dist{\gamma_\infty(t_{k-1})}{\gamma_\infty(t_k)}{}.
\end{align*}

Для каждого $i$,
\[\dist{\gamma_n(t_{i-1})}{\gamma_n(t_i)}{}
\to
\dist{\gamma_\infty(t_{i-1})}{\gamma_\infty(t_i)}{},\]
при $n\to\infty$, и, значит, $\Sigma_n\to \Sigma_\infty$.
Поскольку 
$\Sigma_n\le\length\gamma_n$
для каждого~$n$, получаем что
$$\liminf_{n\to\infty} \length\gamma_n \ge \Sigma_\infty$$
для любого разбиения.
Теперь неравенство \ref{eq:semicont-length} следует из определения длины.
\qeds


\begin{wrapfigure}{o}{20 mm}
\vskip0mm
\centering
\includegraphics{mppics/pic-6}
\end{wrapfigure}

Неравенство \ref{eq:semicont-length} может оказаться строгим.
Например, диагональ $\gamma_\infty$ единичного квадрата 
можно аппроксимировать ступенчатыми ломаными $\gamma_n$,
стороны которых параллельны сторонам квадрата ($\gamma_6$ и $\gamma_\infty$ показаны на рисунке).
В этом случае,
\[\length\gamma_\infty=\sqrt{2}\quad
\text{и}\quad \length\gamma_n=2\quad
\text{для любого}\quad n.\]

\section{Параметризация длиной}

Будем говорить, что кривая $\gamma$ \index{параметризация!длиной}\emph{параметризована длиной} или имеет \index{естественная параметризация}\index{натуральная параметризация}\emph{естественную (натуральную) параметризацию},
если 
\[t_2-t_1=\length \gamma|_{[t_1,t_2]}\]
для любых двух значений параметра $t_1<t_2$;
то есть дуга $\gamma$ от $t_1$ до $t_2$ имеет длину $t_2-t_1$.

\begin{thm}{Упражнение}\label{ex:cont-length}
Пусть $\gamma\:[a,b]\to \spc{X}$ --- спрямляемая кривая в метрическом пространстве.
Для данного $t\in [a,b]$ обозначим через $s(t)$ длину дуги $\gamma|_{[a,t]}$.
Докажите, что функция $t\mapsto s(t)$ непрерывна.

Выведите отсюда, что $\gamma$ допускает параметризацию длиной.
\end{thm}

По упражнению~\ref{ex:integral-length},
гладкая кривая $\gamma(t)=(x(t),y(t),z(t))$ параметризована длиной тогда и только тогда, когда её вектор скорости единичный в любой момент;
то есть
\[|\gamma'(t)|=\sqrt{x'(t)^2+y'(t)^2+z'(t)^2}=1\]
для всех $t$.
Поэтому гладкие кривые, параметризованные длиной, называют также кривыми с \index{кривая с единичной скоростью}\emph{единичной скоростью}.
Отметим, что гладкие параметризации с единичной скоростью всегда регулярны (см.~\ref{sec:Smooth curves}).

\begin{thm}{Предложение}\label{prop:arc-length-smooth}
Если $t\mapsto \gamma(t)$ --- гладкая кривая,
то её параметризация длиной также гладкая и регулярная.
Более того, параметр дуги $s$ для $\gamma$ можно записать в виде интеграла
\[s(t)=\int_{t_0}^t |\gamma'(\tau)|\cdot d\tau.
\eqlbl{s(t)}\]

\end{thm}

Как правило, буква $s$ будет обозначать параметр длины.

\parbf{Доказательство.}
Так как $\gamma$ гладкая, $|\gamma'(t)|>0$ для любого~$t$,
а значит, и функция $t\mapsto|\gamma'(t)|$ гладкая.

По основной теореме анализа, $s'(t)=|\gamma'(t)|$.
В частности, $t\mapsto s(t)$ --- гладкая возрастающая функция с положительной производной.

По теореме об обратной функции (\ref{thm:inverse}), функция $s^{-1}(t)$ также гладкая
и $|(\gamma\circ s^{-1})'|\equiv1$.
Следовательно, репараметризация $\gamma\circ s^{-1}$ имеет единичную скорость.
По построению, $\gamma\circ s^{-1}$ гладкая, и так как $|(\gamma\circ s^{-1})'|\equiv1$, параметризация $\gamma\circ s^{-1}$ регулярна.
\qeds

\begin{thm}{Упражнение}\label{ex:arc-length-helix}
Параметризуйте \index{винтовая линия}\emph{винтовую линию} 
\[\gamma_{a,b}(t)=(a\cdot\cos t,a\cdot \sin t, b\cdot t)\]
длиной.
\end{thm}

Нас будут интересовать свойства кривых, которые не меняются при репараметризациях.
Поэтому всегда можно предполагать, что данная гладкая кривая параметризована длиной.
Выбор этих параметризаций почти каноничен --- они отличаются только знаком и сдвигом, и, значит, с ними проще определить величины, не зависящие от параметризации.
Это наблюдение будет использовано при определении кривизны и кручения.

На практике же проще работать с исходной параметризацией.
Более того, часто невозможно найти параметр длины в явном виде.

\section{Выпуклые кривые}

Простая кривая на плоскости называется \index{выпуклая!кривая}\emph{выпуклой}, если она ограничивает выпуклую область.
Поскольку граница любой области --- замкнутое множество, любая выпуклая кривая либо замкнута, либо открыта (см. \ref{sec:proper-curves}).

\begin{thm}{Предложение}\label{prop:convex-curve}
Предположим, что замкнутая выпуклая кривая $\alpha$ лежит внутри области, ограниченной простой замкнутой кривой $\beta$ на плоскости.
Тогда
\[\length\alpha\le \length\beta.\]
\end{thm}

Достаточно показать, что периметр любого многоугольника, вписанного в $\alpha$, меньше или равен длине $\beta$.
Поскольку любой многоугольник, вписанный в $\alpha$, выпуклый, достаточно доказать следующее.

\begin{thm}{Лемма}\label{lem:perimeter}
Предположим, что выпуклый многоугольник $P$ лежит в фигуре $F$, ограниченной простой замкнутой кривой.
Тогда
\[\perim P\le \perim F,\]
где $\perim F$ обозначает периметр~$F$.
\end{thm}

\parbf{Доказательство.} 
\index{хорда}\emph{Хорда} фигуры $F$ определяется как отрезок прямой в $F$ с концами на его границе.
Предположим, что $F'$ --- это фигура, полученная из $F$ отрезанием куска по её хорде.
По неравенству треугольника,
\[\perim F'\le \perim F.\]

\begin{wrapfigure}{o}{24 mm}
\vskip-10mm
\centering
\includegraphics{mppics/pic-7}
\vskip3mm
\end{wrapfigure}

Заметим, что существует убывающая последовательность фигур 
\[F=F_0\supset F_1\supset\dots\supset F_n=P,\]
такая что $F_{i+1}$ получается из $F_{i}$ отрезанием куска по хорде.
Следовательно, 
\begin{align*}
\perim F=\perim F_0&\ge\perim F_1\ge\dots\ge\perim F_n=\perim P.
\end{align*}
\qedsf

\parit{Замечание.}
Другие доказательства получаются, применением формулы Крофтона (\ref{ex:convex-croftons}) и проецированием $F$ на $P$ как в \ref{lem:nearest-point-projection}.  

\begin{thm}{Следствие}\label{cor:convex=>rectifiable}
Любая выпуклая замкнутая кривая на плоскости спрямляема.
\end{thm}

\parbf{Доказательство.}
Любая замкнутая кривая ограничена.
Действительно, кривую можно задать некоторой петлёй $\alpha\:[0,1]\to\mathbb{R}^2$, $\alpha(t)\z=(x(t),y(t))$.
Координатные функции $t\mapsto x(t)$ и $t\mapsto y(t)$  непрерывны, и определённы на $[0,1]$.
В частности, обе функции ограничены некоторой константой~$C$.
Следовательно, $\alpha$ лежит в квадрате, определённом неравенствами $|x|\le C$ и $|y|\le C$.

Согласно~\ref{prop:convex-curve}, длина кривой не может превышать периметр этого квадрата; отсюда результат.
\qeds

Напомним, что \index{выпуклая!оболочка}\emph{выпуклая оболочка} множества $X$ --- это наименьшее выпуклое множество, содержащее $X$; иначе говоря, это пересечение всех выпуклых множеств, содержащих~$X$.

\begin{thm}{Упражнение}\label{ex:convex-hull}
{\sloppy
Пусть $\alpha$ --- простая замкнутая кривая на плоскости.
Обозначим через $K$ выпуклую оболочку $\alpha$; пусть $\beta$ --- кривая, ограничивающая~$K$.
Докажите, что
\[\length \alpha\ge \length \beta.\]

}

Попробуйте доказать, что утверждение остаётся верным для любых замкнутых кривых на плоскости, предполагая только, что $K$ имеет непустую внутренность.
\end{thm}

\section{Формулы Крофтона}
\label{sec:crofton}
\index{формула Крофтона}

Для функции $f\: \mathbb{S}^1 \to \mathbb{R}$ обозначим её среднее значение как $\overline{f(\vec u)}$; то есть
\[\overline{f(\vec u)}=\frac1{2\cdot \pi}\cdot\int_{\vec u \in\mathbb{S}^1} f(\vec u).\]
Для вектора $\vec w$ и единичного вектора $\vec u$ обозначим через $\vec w_{\vec u}$ ортогональную проекцию $\vec w$ на прямую в направлении $\vec u$;
то есть
\[\vec w_{\vec u}=\langle\vec u,\vec w\rangle\cdot\vec u.\] 

\begin{thm}{Теорема}
Для любой  кривой $\gamma$ на плоскости выполняется равенство
\[
\length\gamma
=\tfrac\pi2\cdot \overline{\length\gamma_{\vec u}}, \eqlbl{crofton-formula}
\]
где $\gamma_{\vec u}$ определяется как $\gamma_{\vec u}(t) \df (\gamma (t))_{\vec u}$.
\end{thm}

\parbf{Доказательство.}
Длина вектора ${\vec w}$ пропорциональна средней длине его проекций; то есть
\[|{\vec w}|=k\cdot \overline{|{\vec w}_{\vec u}|}\]
для некоторого $k \in \mathbb{R}$.
(Коэффициент $k$ можно найти интегрированием\footnote{Это среднее значение $|\cos x|$ при $x\in [0,2\cdot\pi]$.}, но мы найдём его другим способом.)
Пусть $\gamma\:[a,b]\to\mathbb{R}^2$ --- гладкая кривая.
Для любого $t \in [a,b]$,
\[\gamma_{\vec u}'(t)=(\gamma'(t))_{\vec u}
\quad\text{и}\quad
|\gamma'_{\vec u}(t)|=|\langle\vec u,\gamma'(t)\rangle|.\]
Из упражнения~\ref{ex:integral-length}, получаем
\begin{align*}
\length\gamma
&=\int_a^b|\gamma'(t)|\cdot dt=
\\
&=\int_a^b  k\cdot \overline{|\gamma_{\vec u}'(t)|}\cdot dt=
\\
&=k\cdot \overline{\length\gamma_{\vec u}}.
\end{align*}

Поскольку $k$ --- константа, её достаточно вычислить, предположив что $\gamma$ --- единичная окружность.
В этом случае,
\[\length \gamma=2\cdot\pi.\]
Кривая $\gamma_{\vec u}$ пробегает отрезок длины~2 туда-сюда.
Значит $\length\gamma_{\vec u}=4$ для любого $\vec u$, и 
\[\overline{\length\gamma_{\vec u}} =4.\]
Отсюда $2\cdot \pi =k\cdot 4$.
Итак, мы доказали \ref{crofton-formula} для гладких кривых.

Применяя то же рассуждение вместе с \ref{adex:integral-length}, получаем, что \ref{crofton-formula} выполняется для всех липшицевых кривых.
Далее, \ref{ex:cont-length} влечёт \ref{crofton-formula} для произвольных спрямляемых кривых,
ибо параметризация длиной превращает спрямляемую кривую в липшицеву.

Остаётся рассмотреть случай неспрямляемых кривых;
достаточно показать, что 
\[\length\gamma=\infty
\quad\Longrightarrow\quad
\overline{\length\gamma_{\vec u}}=\infty.
\]

Из определения длины следует, что
\[\length\gamma_{\vec u}+\length\gamma_{\vec v}\ge \length\gamma\]
для любой пары $(\vec u , \vec v )$ ортонормированных векторов в $\mathbb{R}^2$.
Следовательно, если $\gamma$ имеет бесконечную длину, то средняя длина $\gamma_{\vec u}$ также бесконечна.
\qeds

\begin{thm}{Упражнение}\label{ex:convex-croftons}
Рассмотрим простую замкнутую кривую $\gamma$, ограничивающую фигуру~$F$ на плоскости.
Обозначим через $s$ среднюю длину проекций $F$ на прямые.
Покажите, что $\length\gamma\ge \pi \cdot s$.
Более того, равенство выполняется тогда и только тогда, когда $\gamma$ выпукла.

Найдите решение \ref{ex:convex-hull} с использованием этого утверждения.
\end{thm}

Следующее упражнение даёт аналогичные формулы для евклидова пространства.

Как и раньше, обозначим через $\vec w_{\vec u}$ ортогональную проекцию вектора $\vec w$ на прямую в направлении $\vec u$.
Далее, обозначим через $\vec w_{\vec u}^\bot$ проекцию $\vec w$ на плоскость, ортогональную к $\vec u$;
то есть
\[\vec w_\vec u^\bot=\vec w - \vec w_{\vec u}.\]
Далее будем пользоваться обозначением 
$\overline{f(\vec u)}$ для среднего значения
функции $f$, определённой на $\mathbb{S}^2$.

\begin{thm}{Продвинутое упражнение}\label{adex:more-croftons}
{\sloppy
Покажите, что длина пространственной кривой пропорциональна 

}
\begin{subthm}{}
средней длине её проекций на все прямые; то есть
\[\length\gamma=k_1\cdot\overline{\length\gamma_{\vec u}}\]
для некоторого $k_1 \in \mathbb{R}$.
\end{subthm}
\begin{subthm}{}средней длине её проекций на все плоскости; то есть
\[\length\gamma=k_2\cdot\overline{\length\gamma_{\vec u}^\bot}\]
для некоторого $k_2 \in \mathbb{R}$.
\end{subthm}
Найдите значения $k_1$ и $k_2$.
\end{thm}



\section{Внутренняя метрика}\label{sec:Length metric}

Пусть $\spc{X}$ --- метрическое пространство.
Для двух точек $x,y$ в $\spc{X}$ обозначим через $\ell(x,y)$ точную нижнюю грань длин всех путей, соединяющих $x$ с $y$;
если же такого пути нет, считаем, что $\ell(x,y)=\infty$.

Легко видеть, что функция $\ell$ удовлетворяет всем аксиомам метрики, за исключением того, что она может равняться бесконечности.
Следовательно, если любые две точки в $\spc{X}$ можно соединить спрямляемой кривой, то $\ell$ определяет новую метрику на $\spc{X}$;
в этом случае $\ell$ называется \index{индуцированная внутренняя метрика}\emph{индуцированной внутренней метрикой} на $\spc{X}$.

Ясно, что $\ell(x,y)\ge \dist{x}{y}{}$ для любой пары точек $x,y\in \spc{X}$.
Если равенство выполняется для всех пар, то метрика $\dist{{*}}{{*}}{}$ называется \index{внутренняя метрика}\emph{внутренней}.

\begin{thm}{Упражнение}\label{ex:induced-is-length}
Пусть $(x,y)\mapsto \ell(x,y)$ --- индуцированная внутренняя метрика на метрическом пространстве $\spc{X}$.
Докажите, что $\ell$ --- внутренняя метрика.
\end{thm}

Большей частью мы рассматриваем пространства с внутренней метрикой.
Например, у евклидова пространства метрика внутренняя.

Метрика на подпространстве $A$ может быть не внутренней, даже если его объемлющее пространство $\spc{X}$ имеет внутреннюю метрику.
Расстояние в индуцированной внутренней метрике на подпространстве $A$ будет обозначаться $\dist{x}{y}A$;
то есть $\dist{x}{y}A$ --- это точная нижняя грань длин путей в $A$ от $x$ до $y$.\index{10aaa@$\lvert x-y\rvert_A$ (внутренняя метрика)}

\begin{thm}{Упражнение}\label{ex:intrinsic-convex}
Пусть $A\subset \mathbb{R}^3$ --- замкнутое подмножество.
Покажите, что $A$ выпукло тогда и только тогда, когда
\[\dist{x}{y}A=\dist{x}{y}{\mathbb{R}^3}\]
для любых $x,y\in A$
\end{thm}



\section{Сферические кривые}

Обозначим через $\mathbb{S}^2$ единичную сферу в пространстве; то есть
\[\mathbb{S}^2=\set{(x,y,z)\in\mathbb{R}^3}{x^2+y^2+z^2=1}.\]
Пространственная кривая $\gamma$ называется \index{сферическая!кривая}\emph{сферической}, если она лежит на $\mathbb{S}^2$;
то есть $|\gamma(t)|=1$ при любом $t$.

Напомним, что $\measuredangle(u,v)$ обозначает угол между векторами $u$ и~$v$.

\begin{thm}{Наблюдение}\label{obs:S2-length}
Для любых $u,v\in \mathbb{S}^2$ выполняется равенство
\[\dist{u}{v}{\mathbb{S}^2}=\measuredangle(u,v).\]

\end{thm}

\parbf{Доказательство.}
Пусть $\gamma$ --- короткая дуга большой окружности%
\footnote{Большая окружность --- это пересечение сферы с плоскостью, проходящей через её центр.}
от $u$ до $v$ в $\mathbb{S}^2$.
Тогда $\length\gamma=\measuredangle(u,v)$.
Следовательно,
\[\dist{u}{v}{\mathbb{S}^2}\le\measuredangle(u,v).\]

Остаётся доказать обратное неравенство.
Другими словами, нам нужно показать, что для любой ломаной $\beta=p_0\dots p_n$, вписанной в~$\gamma$, существует ломаная
$\beta_1=q_0\dots q_n$, вписанная в любую заданную сферическую кривую $\gamma_1$, соединяющую $u$ с $v$, такая, что 
\[\length\beta_1\ge \length \beta.\eqlbl{eq:length beta=<length beta}\]

Определим $q_i$ как первую точку на $\gamma_1$ для которой выполнено равенство $|u-p_i|=|u-q_i|$, но зададим $q_n=v$.
Ясно, что $\beta_1$ вписана в~$\gamma_1$.
По неравенству треугольника для углов (\ref{thm:spherical-triangle-inq}),
\begin{align*}
 \measuredangle(q_{i-1},q_i) &\ge  \measuredangle (u, q_i) - \measuredangle ( u , q_{i-1})  =
\\
&= \measuredangle (u,p_i) - \measuredangle (u,p_{i-1}) =
\\
& = \measuredangle(p_{i-1},p_i).
\end{align*}
По монотонности угла (\ref{lem:angle-monotonicity}),
\[|q_{i-1}-q_i|\ge|p_{i-1}-p_i|\]
и \ref{eq:length beta=<length beta} следует.
\qeds


\begin{thm}{Лемма о полусфере}\label{lem:hemisphere}
Любая замкнутая сферическая кривая длины меньше $2\cdot \pi$ лежит в открытой полусфере.
\end{thm}

Эта лемма проявится в доказательстве теоремы Фенхеля (\ref{thm:fenchel}).
Следующее доказательство принадлежит Стефани Александер.
Оно не так просто, как может показаться.
Прежде чем его читать, стоит попытаться придумать своё.

\parbf{Доказательство.}
Пусть $\gamma$ --- замкнутая кривая в $\mathbb{S}^2$ длины $2\cdot\ell$.
Допустим, что $\ell<\pi$.

{

\begin{wrapfigure}{o}{35mm}
\vskip-3mm
\centering
\includegraphics{mppics/pic-52}
\end{wrapfigure}

Разделим $\gamma$ на две дуги $\gamma_1$ и $\gamma_2$ длины~$\ell$ каждая;
пусть $p$ и $q$ --- их общие концы.
Из \ref{obs:S2-length}, получаем
\begin{align*}
\measuredangle(p,q)&\le \length \gamma_1=
\\
&= \ell<
\\
&<\pi.
\end{align*}

}

Обозначим через $z$ середину между $p$ и $q$ в $\mathbb{S}^2$;
то есть $z$ --- середина короткой дуги большой окружности от $p$ до $q$ в $\mathbb{S}^2$.
Мы утверждаем, что $\gamma$ лежит в открытой полусфере с полюсом в~$z$.
Если это не так, то $\gamma$ пересекает экватор в некоторой точке~$r$.
Не умаляя общности, можно считать, что $r$ принадлежит~$\gamma_1$.

Повернём дугу $\gamma_1$ на угол $\pi$ вокруг прямой, проходящей через $z$ и центр сферы.
Полученная дуга $\gamma_1^{*}$ вместе с $\gamma_1$ образует замкнутую кривую длины $2\cdot \ell$, проходящую через $r$ и её антиподальную точку $r^{*}$.
Снова применив \ref{obs:S2-length}, получим
\[\tfrac12\cdot\length \gamma=\ell\ge \measuredangle(r,r^{*})=\pi\]
--- противоречие.
\qeds


\begin{thm}{Упражнение}\label{ex:antipodal}
Опишите простую замкнутую сферическую кривую, которая не содержит пары антиподальных точек и при этом не лежит ни в какой полусфере.
\end{thm}


\begin{thm}{Упражнение}\label{ex:bisection-of-S2}
Предположим, что простая замкнутая сферическая кривая $\gamma$ делит $\mathbb{S}^2$ на две области равной площади.
Покажите, что 
\[\length\gamma\ge2\cdot\pi.\]
\end{thm}


\begin{thm}{Упражнение}\label{ex:flaw}
Найдите ошибку в решении следующей задачи.
Предложите правильное решение.
\end{thm}

 
\parbf{Задача.}
Предположим, что замкнутая кривая на плоскости имеет длину не более 4.
Покажите, что она лежит в единичном круге.

\parbf{Неправильное решение.}
Достаточно показать, что \index{диаметр}\emph{диаметр} нашей кривой, скажем $\gamma$, не превышает 2;
то есть
\[|p-q|\le 2\eqlbl{eq:|pq|=<2}\]
для любых двух точек $p$ и $q$ на~$\gamma$.

Длина $\gamma$ не меньше длины замкнутой вписанной ломаной, идущей от $p$ до $q$ и обратно до~$p$.
Следовательно,
\[2\cdot |p-q|\le\length \gamma\le 4;\]
откуда вытекает \ref{eq:|pq|=<2}.
\qedsf

\begin{thm}{Продвинутые упражнения} \label{adex:crofton}
Пусть ${\vec u},{\vec w}\in\mathbb{S}^2$.
Обозначим через ${\vec w}^*_{\vec u}$ ближайшую к ${\vec w}$ точку на экваторе с полюсом в ${\vec u}$;
другими словами, если ${\vec w}^\perp_{\vec u}$ --- проекция ${\vec w}$ на плоскость, перпендикулярную ${\vec u}$, то ${\vec w}^*_{\vec u}$ --- единичный вектор в направлении ${\vec w}^\perp_{\vec u}$.
Вектор ${\vec w}^*_{\vec u}$ определён при ${\vec w}\ne\pm {\vec u}$.

\begin{subthm}{adex:crofton:crofton}
Покажите, что для любой спрямляемой 
%one requires rectifiable for the right hand side to be defined for almost all U 
сферической кривой $\gamma$ выполняется
\[\length\gamma=\overline{\length\gamma^*_{\vec u}},\]
где $\overline{\length\gamma^*_{\vec u}}$ обозначает среднюю длину $\gamma^*_{\vec u}$ при изменении ${\vec u}$ на~$\mathbb{S}^2$.
\end{subthm}

\begin{subthm}{adex:crofton:hemisphere}
Найдите другое доказательство леммы о полусфере (\ref{lem:hemisphere}),
используя \ref{SHORT.adex:crofton:crofton}. 
\end{subthm}
 
\end{thm}

Часть \ref{SHORT.adex:crofton:crofton} упражнения --- это сферическая формула Крофтона.
Её можно переписать следующим образом:
\[\length\gamma=\overline n\cdot \pi,\]
где $\overline n$ --- среднее число точек пересечений $\gamma$ с экваторами.
Эквивалентность вытекает из теоремы Леви о монотонной сходимости.

