\chapter{Определения}
\label{chap:curves-def}

По следам велосипеда часто можно определить направление его движения, а также отличить следы переднего колеса от заднего, и найти расстояние между ними.

Подумайте, как всё это можно сделать.
Это заставит вас переоткрыть значительную часть дифференциальной геометрии кривых и вам станет легче читать первую часть;
см. также ссылку сразу после упражнения~\ref{ex:bike}.

\section{Прежде чем начать}

Понятие кривой имеет много вариаций.
Некоторые описываются существительными (путь, дуга и так далее), 
другие --- прилагательными (замкнутая, открытая, собственная, простая, гладкая и так далее).
Следующий рисунок даёт представление о четырёх типах кривых.

\vskip-0mm
\begin{figure}[h!]
\begin{minipage}{.48\textwidth}
\centering
\includegraphics{mppics/pic-110}
\end{minipage}\hfill
\begin{minipage}{.48\textwidth}
\centering
\includegraphics{mppics/pic-115}
\end{minipage}
\bigskip
\begin{minipage}{.48\textwidth}
\centering
\includegraphics{mppics/pic-120}
\end{minipage}\hfill
\begin{minipage}{.48\textwidth}
\centering
\includegraphics{mppics/pic-125}
\end{minipage}
\end{figure}
\vskip-0mm

Далее мы обсудим все эти определения;
главу можно пропустить, и пользоваться ей в дальнейшем как справочником.

\section{Простые кривые}

Для нас евклидова плоскость $\mathbb{R}^2$ и пространство $\mathbb{R}^3$ будут главными примерами метрических пространств; см. раздел~\ref{sec:metric-spcaes}.

Напомним, что \index{интервал}\emph{интервал} --- это связное подмножество вещественной прямой.
Непрерывная биекция $f\:X\to Y$ между подмножествами метрических пространств называется {}\emph{гомеоморфизмом}, если обратное отображение $f^{-1}\:Y\to X$ также непрерывно.  

\begin{thm}{Определение} 
Связное подмножество $\gamma$ метрического пространства называется \index{простая кривая}\emph{простой кривой}, если оно \index{локально гомеоморфно}\emph{локально} гомеоморфно интервалу;
то есть любая точка $p\in \gamma$ имеет окрестность в $\gamma$, гомеоморфную интервалу.
\end{thm}

Известно, что любую простую кривую $\gamma$ можно \index{параметризация}\emph{параметризовать} интервалом или окружностью.
То есть существует гомеоморфизм $\GG\to\gamma$, где $\GG$ --- интервал (открытый, замкнутый или полуоткрытый) или же окружность
\[\mathbb{S}^1=\set{(x,y)\in\mathbb{R}^2}{x^2+y^2=1}.\] 
Это доказано в статье Дэвида Гейла \cite{gale};
доказательство несложное, но оно отвлекло бы нас от основной темы.
%Придирчивый читатель может добавить это свойство в определение кривых.

Параметризация $\GG\to\gamma$ полностью описывает кривую.
Часто кривая и её параметризация обозначаются той же буквой;
например, можно сказать, что кривая $\gamma$ задана параметризацией $\gamma\:(a,b]\to \mathbb{R}^2$.
Однако любая простая кривая допускает различные параметризации.

\begin{thm}{Упражнение}\label{ex:9}
{\sloppy
\begin{subthm}{ex:9:compact}
Покажите, что образ любого непрерывного инъективного отображения $\gamma\:[0,1]\to\mathbb{R}^2$ является простой кривой.
\end{subthm}

}

\begin{subthm}{ex:9:9}
Постройте непрерывное инъективное отображение $\gamma\:(0,1)\z\to\mathbb{R}^2$, образ которого \textit{не} является простой кривой.
\end{subthm}

\end{thm}

\section{Параметризованные кривые}\label{sec:Parametrized curves}

Пусть $\GG$ --- окружность или интервал (открытый, замкнутый или полуоткрытый), и $\spc{X}$ --- метрическое пространство.
\index{параметризованная кривая}\emph{Параметризованная кривая} определяется как непрерывное отображение $\gamma\:\GG\to\spc{X}$. 
Для параметризованной кривой мы \textit{не} предполагаем, что отображение инъективно; другими словами, допускаются {}\emph{самопересечения}.

Про простую кривую можно всегда думать как про параметризованную.
При этом термин \index{кривая}\emph{кривая} можно использовать когда мы не хотим уточнять, параметризованная она или простая.

Если область определения $\GG$ является открытым интервалом или окружностью, то $\gamma$ называется {}\emph{кривой без концов};
в противном случае, {}\emph{кривой с концами}.
В случае, если $\GG$ --- окружность, $\gamma$ называется {}\index{замкнутая!кривая}\emph{замкнутой кривой}. 
Если $\GG$ --- замкнутый интервал $[a,b]$, то кривая называется \index{дуга}\emph{дугой}.
Если в добавок $\GG$ --- единичный интервал $[0,1]$, то кривая также называется \index{путь}\emph{путём}.

\begin{wrapfigure}{o}{15 mm}
\vskip-0mm
\centering
\includegraphics{mppics/pic-130}
\vskip-4mm
\end{wrapfigure}

Если для дуги $\gamma \: [a,b] \to \spc{X}$ выполнено условие $p\z=\gamma (a)=\gamma (b)$, то она называется \index{петля}\emph{петлёй}, и $p$ называется \index{базовая точка}\emph{базовой точкой} петли.

Пусть $\GG_1$ и $\GG_2$, оба либо интервалы, либо окружности.
Непрерывное сюръективное отображение $\tau\:\GG_1\to\GG_2$ называется \index{монотонное отображение}\emph{монотонным}, если для любого $t\in \GG_2$ множество $\tau^{-1}\{t\}$ связно.
Если $\GG_1$ и $\GG_2$ --- интервалы, то, по теореме о промежуточных значениях, монотонное отображение либо неубывающее, либо невозрастающее;
то есть наше определение совпадает со стандартным, в случае если $\GG_1$ и $\GG_2$ --- интервалы.

\begin{thm}{Упражнение}\label{ex:mono}
Приведите пример монотонного (в частности сюръективного) отображения $(0,1)\to [0,1]$.
\end{thm}

Пусть $\gamma_1\:\GG_1\to \spc{X}$ и $\gamma_2\:\GG_2\to \spc{X}$ --- две параметризованные кривые, такие что 
$\gamma_1=\gamma_2\circ\tau$ для монотонного отображения $\tau\:\GG_1\to\GG_2$.
Тогда мы говорим, что $\gamma_2$ --- \index{репараметризация}\emph{репараметризация}%
\footnote{При этом $\gamma_1$ может \textit{не} быть репараметризацией $\gamma_2$.
Другими словами, с нашим определением, отношение \textit{быть репараметризацией} не является симметричным;
в частности, это \textit{не} отношение эквивалентности.
Дело можно поправить, перейдя к минимальному отношению эквивалентности, как это сделано в \cite[2.5.1]{burago-burago-ivanov},
но мы останемся верны нашему варианту определения.}
$\gamma_1$ с помощью $\tau$.


\begin{thm}{Продвинутое упражнение}\label{aex:simple-curve}
Пусть $X$ --- подмножество плоскости.
Предположим, что две различные точки $p,q\in X$ можно соединить путём в~$X$.
Покажите, что существует простая дуга в~$X$, из $p$ в~$q$.
\end{thm}

Любую петлю (как и замкнутую кривую) можно задать {}\emph{периодической} параметризацией $\gamma\: \mathbb{R}\to \spc{X}$;
то есть такой, что $\gamma(t+\ell)=\gamma(t)$ для некоторого периода $\ell>0$ и всех~$t$.
Например, единичная окружность на плоскости описывается $2{\cdot}\pi$-периодической параметризацией $\gamma(t)\z=(\cos t,\sin t)$.

Верно и обратное: про кривую с периодической параметризацией можно думать как про замкнутую кривую или как про петлю.

\section{Гладкие кривые}\label{sec:Smooth curves}

Кривые в евклидовом пространстве или на плоскости называются соответственно {}\emph{пространственными} кривыми или {}\emph{плоскими}.

Параметризованную пространственную кривую можно описать её координатными функциями 
$\gamma(t)=(x(t),y(t),z(t))$.
Плоские кривые можно рассматривать как частный случай пространственных кривых с $z(t)\equiv 0$.

Напомним, что вещественная функция называется \index{гладкая!функция}\emph{гладкой}, если её производные всех порядков определены всюду в области определения.  
Если каждая из координатных функций $x(t), y(t)$ и $z(t)$ гладкая, то и параметризация называется \index{гладкая!параметризация}\emph{гладкой}.

Если \index{вектор скорости}\emph{вектор скорости} 
$\gamma'(t)=(x'(t),y'(t),z'(t))$
не равен нулю ни в одной точке, то параметризация $\gamma$ называется \index{регулярная!параметризация}\emph{регулярной}.

Параметризованная кривая называется {}\emph{гладкой}, если её параметризация гладкая и регулярная.
Простая пространственная кривая называется \index{гладкая!кривая}\emph{гладкой}, если она допускает регулярную гладкую параметризацию;
для замкнутой кривой предполагается, что параметризация периодическая.
Это главные объекты первой части книги.
Гладкие кривые можно было бы называть {}\emph{регулярными гладкими кривыми};
получилось бы точнее и длиннее.

Гладкая петля, может определять негладкую замкнутую кривую;
пример показан на рисунке.

\begin{wrapfigure}{o}{17 mm}
\vskip-4mm
\centering
\includegraphics{mppics/pic-51}
\bigskip
\includegraphics{mppics/pic-140}
\vskip-8mm
\end{wrapfigure}

Согласно следующему упражнению, кривые с гладкими параметризациями могут оказаться негладкими.

\begin{thm}{Упражнение}\label{ex:L-shape}
Из раздела \ref{sec:analysis} видно, что следующая функция является гладкой:
\[f(t)=
\begin{cases}
0&\text{если}\ t\le 0,
\\
\frac{t}{e^{1\!/\!t}}&\text{если}\ t> 0.
\end{cases}
\]

Покажите, что $\alpha(t)=(f(t),f(-t))$ описывает гладкую параметризацию кривой на рисунке;
это простая кривая, образованная объединением двух полуосей на плоскости.

{\sloppy

Покажите, что любая гладкая параметризация этой кривой имеет нулевой вектор скорости в начале координат.
Выведите отсюда, что эта кривая негладкая;
то есть не допускает регулярной гладкой параметризации.

}

\end{thm}

\begin{thm}{Упражнение}\label{ex:cycloid}
Опишите множество вещественных чисел $\ell$, 
при которых параметризация $\gamma_\ell (t)= (t+\ell \cdot \sin t,\ell \cdot \cos t)$, $t\in\mathbb{R}$ является

\begin{minipage}{.30\textwidth}
\begin{subthm}{ex:cycloid:smooth}
гладкой; 
\end{subthm}
\end{minipage}
\hfill
\begin{minipage}{.30\textwidth}
\begin{subthm}{ex:cycloid:regular}
регулярной;
\end{subthm}
\end{minipage}
\hfill
\begin{minipage}{.30\textwidth}
\begin{subthm}{ex:cycloid:simple}
простой.
\end{subthm}
\end{minipage}

\end{thm}

\begin{thm}{Упражнение}\label{ex:nonregular}
Постройте гладкую, но \textit{не}регулярную параметризацию кубической параболы $y=x^3$ на плоскости.
\end{thm}

\section{Неявно заданные кривые}\label{sec:implicit-curves}

Предположим, что $f\:\mathbb{R}^2\to \mathbb{R}$ --- гладкая функция; 
то есть все её частные производные везде определены.
Пусть $\gamma\subset \mathbb{R}^2$ --- её множество уровня, описанное уравнением $f(x,y)=0$.

Предположим, что $0$ является \index{регулярное значение}\emph{регулярным значением}~$f$; то есть градиент $\nabla_p f$ не обращается в ноль в любой точке $p\in \gamma$.
Другими словами, если $f(p)=0$, то   
$f_x(p)\ne 0$ или $f_y(p)\ne 0$.%
\footnote{Здесь $f_x$ является сокращённой записью для частной производной
$\tfrac{\partial f}{\partial x}$.\index{10f@$f_x$ (частная производная)}}
Если при этом множество $\gamma$ связно, то по теореме об обратной функции (\ref{thm:imlicit}), $\gamma$ --- гладкая простая кривая. 

Описанное условие достаточно, но \textit{не необходимо}.
Например, ноль \textit{не} является регулярным значением функции $f(x,y)\z=y^2$, однако уравнение $f(x,y)=0$ описывает гладкую кривую --- ось $x$.

Аналогично, предположим, что $(f,h)$ --- пара гладких функций на $\mathbb{R}^3$.
Система уравнений
\[\begin{cases}
   f(x,y,z)=0,
   \\
   h(x,y,z)=0
  \end{cases}
\]
описывает гладкую пространственную кривую, если множество решений $\gamma$ связно, и $0$ является регулярным значением отображения $F\:\mathbb{R}^3\to\mathbb{R}^2$, определённого как
\[F\:(x,y,z)\mapsto (f(x,y,z),h(x,y,z)).\]
Это означает, что градиенты $\nabla f$ и $\nabla h$ линейно независимы в любой точке $p\in \gamma$.
Другими словами, якобиан
\[\Jac_pF=
\begin{pmatrix}
f_x&f_y&f_z\\
h_x&h_y&h_z
\end{pmatrix}
\]
отображения $F\:\mathbb{R}^3\to\mathbb{R}^2$ имеет ранг 2 в любой точке $p \in \gamma$.

Если кривая $\gamma$ описана таким образом,
то мы говорим, что она \index{неявно заданная кривая}\emph{задана неявно}.

Теорема об обратной функции гарантирует существование регулярных гладких параметризаций для любой неявно заданной кривой.
Однако на практике проще работать непосредственно с неявным представлением. 

\begin{thm}{Упражнение}\label{ex:y^2=x^3}
Рассмотрим множество на плоскости, описанное уравнением
$y^2=x^3$.
Является ли оно простой кривой?
Является ли оно гладкой кривой?
\end{thm}

\begin{thm}{Упражнение}\label{ex:viviani}
Опишите множество вещественных чисел $\ell$, 
при которых система уравнений
\[\begin{cases}
x^2+y^2+z^2&=1
\\
x^2+\ell\cdot x+y^2&=0
\end{cases}\]
описывает гладкую кривую.
\end{thm}

\section{Собственные, замкнутые и открытые}\label{sec:proper-curves}

{\sloppy

Параметризованная кривая $\gamma$ в метрическом пространстве $\spc{X}$ называется \index{собственная!кривая}\emph{собственной}, если для любого компактного множества $K \z\subset \spc{X}$ его прообраз $\gamma^{-1}(K)$ компактен.

}

Вот пример несобственной кривой, определенной на всей вещественной прямой: $\gamma(t)=(e^t,0,0)$.
Действительно, множество $(-\infty,0]$ некомпактно, но это прообраз замкнутого единичного шара с центом в начале координат.

\begin{thm}{Упражнение}\label{ex:open-curve}
Покажите, что кривая $\gamma\:\mathbb{R}\to\mathbb{R}^3$ собственная тогда и только тогда, когда $|\gamma(t)|\z\to\infty$ при $t\to\pm\infty$.
\end{thm}

Напомним, что замкнутый интервал компактен, и замкнутые подмножества компактного множества также компактны
(см. \ref{sec:topology}).
Поскольку окружности и замкнутые конечные интервалы компактны, замкнутые кривые и дуги являются собственными кривыми.

Простая кривая называется собственной, если она допускает собственную параметризацию.

\begin{thm}{Упражнение}\label{ex:proper-closed}
Покажите, что простая пространственная кривая является собственной тогда и только тогда, когда она образует замкнутое множество.
\end{thm}

Собственная простая кривая называется \index{открытая!кривая}\emph{открытой}, если она не замкнута и не имеет концов.
Таким образом, любая простая собственная кривая без концов является либо замкнутой, либо открытой.
Обратите внимание, что термины \textit{открытая кривая} и \textit{замкнутая кривая} не имеют отношения к открытым и замкнутым множествам.

\begin{thm}{Упражнение}\label{ex:proper-curve}
Используя теорему Жордана (\ref{thm:jordan}), покажите, что любая простая открытая кривая на плоскости делит её на две связные компоненты.
\end{thm}
