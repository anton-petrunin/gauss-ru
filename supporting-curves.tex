\chapter{Опорные кривые}
\label{chap:supporting-curves}


Если одна кривая касается другой, оставаясь на той же её стороне, то кривизну одной можно оценить через кривизну другой.
Это будет доказано и использовано при изучении глобального поведения кривых на плоскости.

\section{Сонаправленность}

Пусть $\gamma_1$ и $\gamma_2$ --- гладкие кривые на плоскости.
Напомним, что кривые $\gamma_1$ и $\gamma_2$ касаются при $t_1$ и $t_2$,
если $\gamma_1(t_1)=\gamma_2(t_2)$
и у них общая касательная при $t_1$ и $t_2$ соответственно.
В этом случае точка $p\z=\gamma_1(t_1)=\gamma_2(t_2)$ называется \index{точка!касания}\emph{точкой касания} кривых.
Заметим, что векторы скорости $\gamma_1'(t_1)$ и $\gamma_2'(t_2)$ параллельны.
\begin{figure}[!ht]
\vskip-0mm
\centering
\includegraphics{mppics/pic-85}
\vskip-0mm
\end{figure}
Касание называется \index{сонаправленные и противонаправленные!кривые}\emph{сонаправленым} или {}\emph{противонаправленым} если вектора $\gamma_1'(t_1)$ и $\gamma_2'(t_2)$ сонаправлено или противоположны соответственно.

Обращение параметризации у одной из кривых превращает сонаправленное касание в противонаправленное и наоборот.
Так что всегда можно считать данное касание сонаправленным.

\pagebreak%???

\section{Опорные кривые}

\begin{wrapfigure}[8]{o}{43 mm}
\vskip-4mm
\centering
\includegraphics{mppics/pic-86}
\vskip0mm
\end{wrapfigure}

Пусть $\gamma_1$ и $\gamma_2$ --- две гладкие плоские кривые с общей точкой 
\[p=\gamma_1(t_1)=\gamma_2(t_2),\] 
и при этом $p$ не является концом ни $\gamma_1$, ни $\gamma_2$.
Предположим, что для некоторого $\epsilon>0$, дуга $\gamma_2|_{[t_2-\epsilon, t_2+\epsilon]}$ лежит в замкнутой области $R$ с дугой $\gamma_1|_{[t_1-\epsilon, t_1+\epsilon]}$ на её границе.
Тогда мы говорим, что $\gamma_1$ --- \index{опорная кривая}\emph{локальная опорная} к $\gamma_2$ при параметрах $t_1$ и $t_2$.
При этом, если кривые на рисунке параметризованы по стрелкам, то $\gamma_1$ подпирает $\gamma_2$ \emph{справа} в точке $p$ (а также $\gamma_2$ подпирает $\gamma_1$ \emph{слева} в~$p$).

Предположим, что $\gamma_1$ простая, и она разрезает плоскость на две замкнутые области: слева и справа от~$\gamma_1$.
Мы говорим, что $\gamma_1$ \index{опорная кривая}\emph{глобально подпирает} $\gamma_2$ в точке $p=\gamma_2(t_2)$, 
если $\gamma_2$ лежит в одной из этих замкнутых областей, и при этом $p$ лежит на~$\gamma_1$.

Далее, предположим, что $\gamma_2$ --- простая замкнутая кривая на плоскости.
По теореме Жордана (\ref{thm:jordan}), $\gamma_2$ разрезает плоскость на два замкнутые области: одна из них ограниченна, другая --- нет.
Мы говорим, что точка $p$ лежит {}\emph{внутри} (соответственно, {}\emph{снаружи}) $\gamma_2$, если $p$ лежит в ограниченной области (соответственно, неограниченной области).
Если $\gamma_1$ подпирает $\gamma_2$ и лежит внутри $\gamma_2$ (снаружи $\gamma_2$), то мы говорим, что $\gamma_1$ подпирает $\gamma_2$ \index{опорная кривая}\emph{изнутри} (соответственно, {}\emph{снаружи})


Если $p=\gamma_1(t_1)=\gamma_2(t_2)$ и кривые $\gamma_1$ и $\gamma_2$ не касаются друг друга при $t_1$ и $t_2$, то в момент времени $t_2$ кривая $\gamma_2$ переходит с одной стороны $\gamma_1$ на другую.
В этом случае $\gamma_1$ не может подпирать $\gamma_2$ при $t_1$ и $t_2$, то есть верно следующее.

\begin{thm}{Наблюдение и определение}
Пусть $\gamma_1$ и $\gamma_2$ --- две гладкие кривые на плоскости.
Предположим, что $\gamma_1$ локально подпирает $\gamma_2$ при $t_1$ и $t_2$.
Тогда $\gamma_1$ и $\gamma_2$ касаются при $t_1$ и $t_2$.

В этом случае, если кривые сонаправлены, и область $R$ в определении опорных кривых лежит справа (слева) от дуги $\gamma_1$, то мы говорим, что 
$\gamma_1$ подпирает $\gamma_2$ слева (соответственно справа).
\end{thm}

Мы говорим, что гладкая плоская кривая $\gamma$ имеет \index{вершина кривой}\emph{вершину} при $s$, 
если $\skur'(s)_\gamma=0$, то есть функция ориентированной кривизны имеет критическую точку в $s$.
Можно также сказать, что точка $p=\gamma(s)$ есть вершина $\gamma$,
и если $\gamma$ простая, то это не приводит к неоднозначности.

\begin{thm}{Упражнение}\label{ex:vertex-support}
Докажите, что точка $p\z=\gamma(s)$ гладкой плоской кривой $\gamma$ является её вершиной если
соприкасающаяся окружность $\sigma_s$ при $s$ локально подпирает $\gamma$ при $\gamma(s)$.
\end{thm}

\section{Признак опорной}

\begin{thm}{Предложение}\label{prop:supporting-circline}
Пусть $\gamma_1$ и $\gamma_2$ --- гладкие кривые на плоскости.

Предположим, что $\gamma_1$ локально подпирает $\gamma_2$ слева (справа) при $t_1$ и $t_2$.
Тогда
\[\skur_1(t_1)\ge \skur_2(t_2)\quad(\text{соответственно}\quad \skur_1(t_1)\le \skur_2(t_2)),\]
где $\skur_1$ и $\skur_2$ --- ориентированные кривизны $\gamma_1$ и $\gamma_2$, соответственно.

При этом строгое неравенство влечёт обратное.
А именно, если $\gamma_1$ и $\gamma_2$ касаются и сонаправлены при $t_1$ и $t_2$, то
\[\skur_1(t_1)> \skur_2(t_2)\quad(\text{соответственно}\quad \skur_1(t_1)< \skur_2(t_2)),\]
влечёт, что $\gamma_1$ локально подпирает $\gamma_2$ слева (справа) при $t_1$ и $t_2$.

\end{thm}


\parbf{Доказательство.}
Не умаляя общности, можно считать, что $t_1\z=t_2\z=0$, общая точка $\gamma_1(0)=\gamma_2(0)$ --- начало координат, а векторы скорости $\gamma'_1(0)$, $\gamma'_2(0)$ направлены по оси $x$.
Тогда малые дуги $\gamma_1|_{[-\epsilon,+\epsilon]}$ и $\gamma_2|_{[-\epsilon,+\epsilon]}$ задаются графиками 
$y=f_1(x)$ и $y=f_2(x)$ гладких функций $f_1$ и $f_2$, таких что $f_i(0)=0$ и $f_i'(0)=0$.
Из \ref{ex:curvature-graph}, $f_1''(0)=\skur_1(0)$ и $f_2''(0)=\skur_2(0)$.

Очевидно, что $\gamma_1$ подпирает $\gamma_2$ слева (справа), если 
\[f_1(x)\ge f_2(x)\quad(\text{соответственно}\quad f_1(x)\le f_2(x))\]
для всех достаточно малых~$x$.
Остаётся применить  к функции $f_1-f_2$ признак экстремума по второй производной.

Доказательство обратного аналогично.
\qeds


\begin{thm}{Продвинутое упражнение}\label{ex:support}
Предположим, что две гладкие простые плоские кривые с единичной скоростью $\gamma_0$ и $\gamma_1$ касаются и сонаправлены в точке $p\z=\gamma_0(0)\z=\gamma_1(0)$ и при этом $\skur_0(s)\le\skur_1(s)$ для любого~$s$.
Покажите, что $\gamma_0$ локально подпирает $\gamma_1$ справа в точке~$p$.

Приведите пример двух простых кривых $\gamma_0$ и $\gamma_1$, удовлетворяющих указанному условию, таких что $\gamma_0$ замкнута, но не глобальная опорная к $\gamma_1$ в точке $p$.
\end{thm}

Напомним (см. \ref{thm:DNA}), что среднее значение кривизны замкнутой гладкой кривой в единичном круге не меньше~1.
В частности, на ней должна быть точка, с кривизной хотя бы~1.
Следующее упражнение говорит, что то же верно и для петель.

\begin{thm}{Упражнение}\label{ex:in-circle}
Предположим, что гладкая петля $\gamma$ лежит в единичном круге на плоскости.
Покажите, что $\gamma$ имеет кривизну не меньше~1 в какой то точке.
\end{thm}


\begin{thm}{Упражнение}\label{ex:between-parallels-1}
{\sloppy
Пусть замкнутая гладкая кривая $\gamma$ лежит между двумя параллельными прямыми на плоскости, расстояние между которыми равно 2.
Покажите, что существует точка на $\gamma$ с кривизной хотя бы~1.

}

Попробуйте доказать то же самое для гладкой плоской петли.
\end{thm}

\begin{thm}{Упражнение}\label{ex:in-triangle}
Пусть замкнутая гладкая плоская кривая $\gamma$ лежит внутри треугольника $\triangle$ со вписанным радиусом~1; то есть единичная окружность касается всех трёх сторон $\triangle$. 
Покажите, что существует точка на $\gamma$ с кривизной хотя бы~$1$.
\end{thm}

Отметим, что три упражнения выше являются частными случаями упражнения \ref{ex:moon-rad},
однако постарайтесь найти прямые решения.

{

\begin{wrapfigure}{r}{32 mm}
\vskip-4mm
\centering
\includegraphics{mppics/pic-70}
\vskip0mm
\end{wrapfigure}

\begin{thm}{Упражнение}\label{ex:lens}
Пусть $F$ --- фигура на плоскости, ограниченная двумя дугами окружностей $\sigma_1$ и $\sigma_2$ с ориентированной кривизной 1, которые проходят от $x$ до~$y$.
Предположим, что $\sigma_1$ короче, чем~$\sigma_2$,
а гладкая дуга $\gamma$ лежит в $F$ и имеет обе конечные точки на $\sigma_1$.
Покажите, что кривизна $\gamma$ не меньше 1 при каком-то значении параметра.

\end{thm}

}

\section{Выпуклые кривые}

Напомним, что плоская кривая называется \index{выпуклая!кривая}\emph{выпуклой}, если она ограничивает выпуклую область.

\begin{thm}{Предложение}\label{prop:convex}
Пусть простая замкнутая гладкая кривая $\gamma$ ограничивает компактное множество~$F$ на плоскости.
Для выпуклости $F$ необходимо и достаточно чтобы ориентированная кривизна $\gamma$ не меняла знак.
\end{thm}


\begin{thm}{Лемма о линзе}\label{lem:lens}
Пусть $\gamma$ --- гладкая простая плоская кривая, идущая от $x$ до~$y$.
Пусть $\gamma$ проходит строго с правой стороны (левой стороны) от ориентированной прямой $xy$, и только её конечные точки $x$ и $y$ лежат на прямой.
Тогда на $\gamma$ найдётся точка с положительной (соответственно, отрицательной) ориентированной кривизной.
\end{thm}

{

\begin{wrapfigure}{o}{35 mm}
\vskip-0mm
\centering
\includegraphics{mppics/pic-22}
\vskip0mm
\end{wrapfigure}

Лемма не выполняется для кривых с самопересечениями.
Например, кривая $\gamma$ на рисунке всегда поворачивает направо, 
поэтому её ориентированная кривизна везде отрицательна, однако она находится с правой стороны от прямой $xy$.


}


\begin{wrapfigure}[6]{i}{50 mm}
\vskip-0mm
\centering
\includegraphics{mppics/pic-24}
\end{wrapfigure}

\parbf{Доказательство.}
Выберем точки $p$ и $q$ на прямой $xy$, 
таким образом, чтобы точки $p, x, y, q$ появлялись на прямой в этом порядке.
Можно предположить, что $p$ и $q$ лежат достаточно далеко от $x$ и $y$, так что полукруг с диаметром $pq$ содержит~$\gamma$.

Рассмотрим наименьший круговой сегмент с хордой $[p,q]$, который содержит~$\gamma$.
Его дуга $\sigma$ подпирает $\gamma$ в некоторой точке $w=\gamma(t_0)$.

Давайте параметризуем $\sigma$ от $p$ до~$q$.
Заметим, что $\gamma$ и $\sigma$ касаются и сонаправлены в~$w$.
В противном случае, дуга $\gamma$ от $w$ до $y$ была бы заперта в криволинейном треугольнике $xwp$, ограниченным отрезком $[p,x]$ и дугами кривых $\sigma$ и $\gamma$.
Но это невозможно, поскольку $y$ не принадлежит этому треугольнику.

\begin{wrapfigure}{o}{50 mm}
\vskip-4mm
\centering
\includegraphics{mppics/pic-23}
\bigskip
\includegraphics{mppics/pic-230}
\end{wrapfigure}

Из этого следует, что $\sigma$ подпирает $\gamma$ справа при $t_0$.
Согласно \ref{prop:supporting-circline},
\[\skur(w)_\gamma\ge \skur_\sigma >0.\]
\qedsf

\parit{Замечание.}
Вместо того, чтобы брать минимальный круговой сегмент, можно было бы выбрать точку $w$ на $\gamma$, лежащую на максимальном расстоянии до прямой $xy$.
То же рассуждение показывает, что кривизна в $w$ неотрицательна, что немного слабее требуемой положительной кривизны.


\parbf{Доказательство \ref{prop:convex}; необходимость.}
Если $F$ выпукла, то каждая касательная прямая к $\gamma$ подпирает~$\gamma$.
При движении по $\gamma$, фигура $F$ должна оставаться с одной стороны от её касательной.
Можно предположить, что каждая касательная прямая подпирает $\gamma$ с одной стороны, скажем, справа.
Поскольку прямая имеет нулевую кривизну, по признаку опорной (\ref{prop:supporting-circline}) $\skur\ge 0$ в каждой точке.

\begin{wrapfigure}{r}{35 mm}
\vskip-3mm
\centering
\includegraphics{mppics/pic-68}
\vskip0mm
\end{wrapfigure}

\parit{Достаточность.}
Обозначим через $K$ выпуклую оболочку~$F$.
Если $F$ не выпукла, то $F$ является собственным подмножеством~$K$.
Следовательно, граница $\partial K$ содержит отрезок прямой, который не является частью $\partial F$.
Другими словами, существует прямая, опорная к $\gamma$ в двух точках, скажем $x$ и $y$.
Эти точки делят $\gamma$ на две дуги $\gamma_1$ и $\gamma_2$, обе отличные от отрезка прямой $[x,y]$.

Одна из дуг $\gamma_1$ или $\gamma_2$ параметризована от $x$ до $y$, а другая от $y$ до~$x$.
Применив лемму о линзе, получаем, что на $\gamma$ есть точки с ориентированными кривизнами противоположных знаков.
Возможно при этом нужно будет перейти к меньшим дугам так, чтобы только их конечные точки лежали на прямой. 
\qeds


\begin{thm}{Упражнение}\label{ex:convex small}
Пусть $\gamma$ --- гладкая простая замкнутая плоская кривая диаметра больше~2.
Покажите, что на $\gamma$ найдётся точка с кривизной меньше~1.
\end{thm}

\begin{wrapfigure}{r}{45 mm}
\vskip-6mm
\centering
\includegraphics{mppics/pic-713}
\vskip0mm
\end{wrapfigure}

\begin{thm}{Упражнение}\label{ex:convex-lens}
Пусть $\gamma$ --- простая гладкая плоская дуга с концами $p$ и~$q$.
Предположим, что $\gamma$ имеет неотрицательную ориентированную кривизну, и $|p-q|$ равно диаметру~$\gamma$.
Покажите, что дуга $\gamma$ и её хорда $[p,q]$ ограничивают выпуклую фигуру на плоскости.
\end{thm}

\begin{thm}{Упражнение}\label{ex:diameter-of-simple-curve}
Покажите, что любая простая гладкая плоская кривая $\gamma$ с кривизной не менее 1 имеет диаметр не более 2.

Попробуйте доказать, что $\gamma$ лежит в единичном круге.
\end{thm}



\section{О луне в луже}

Следующая теорема уточняет результат Владимира Ионина и Германа Пестова \cite{ionin-pestov},\index{теорема Ионина --- Пестова}
а для выпуклых кривых он был известен ранее \cite[\S 24]{blaschke}.


\begin{wrapfigure}{r}{18 mm}
\vskip-6mm
\centering
\includegraphics{mppics/pic-67}
\vskip-2mm
\end{wrapfigure}

\begin{thm}{Теорема}\label{thm:moon-orginal}
Предположим, что простая гладкая петля с кривизной не больше 1 ограничивает фигуру $F$ на плоскости.
Тогда $F$ содержит единичный круг.
\end{thm}

Это простой, но содержательный пример теорем типа {}\emph{от локального к глобальному}.
То есть, исходя из некоторых локальных свойств (в данном случае оценки на кривизну) мы доказываем какое-то глобальное свойство (в данном случае существование единичного круга, окружённого кривой).

{

\begin{wrapfigure}{r}{33 mm}
\vskip-0mm
\centering
\includegraphics{mppics/pic-62}
\vskip0mm
\end{wrapfigure}

Можно было бы попробовать, начать с какого-то круга в $F$ раздувать его в надежде достичь единичного радиуса.
Однако, пример на рисунке показывает, что это не всегда приводит к решению --- 
иногда, чтобы достичь единичного радиуса придётся сначала сдуть круг.

\begin{thm}{Основная лемма}\label{thm:moon}
Пусть $\gamma$ --- простая гладкая петля на плоскости.
Тогда соприкасающаяся к ней окружность в некоторой точке (отличной от базовой) глобально подпирает $\gamma$ изнутри.
\end{thm}

}

Сначала покажем, что теорема следует из леммы.

\parbf{Доказательство \ref{thm:moon-orginal}, используя \ref{thm:moon}.}
Поскольку кривизна $\gamma$ не превышает 1, каждая соприкасающаяся окружность имеет радиус не менее~1.
По основной лемме, одна из соприкасающихся окружностей, скажем $\sigma$, подпирает $\gamma$ изнутри.
В частности, $\sigma$ лежит внутри $\gamma$, отсюда результат.
\qeds


\parbf{Доказательство \ref{thm:moon}.}
Обозначим через $F$ замкнутую область, окружённую $\gamma$.
Можно считать, что $F$ находится слева от~$\gamma$.
Рассуждая от противного, допустим, что соприкасающаяся окружность в каждой точке $p\in \gamma$ не лежит в~$F$.

\begin{figure}[!ht]
\vskip-0mm
\centering
\includegraphics{mppics/pic-32}
\vskip-2mm
\end{figure}

Для данной точки $p\in\gamma$, рассмотрим максимальную окружность, которая полностью лежит в $F$ и касается $\gamma$ в~$p$.
Эту окружность, скажем $\sigma$, будем называть {}\emph{вписанной} в $F$ при~$p$.
Её кривизна $\skur_\sigma$ обязана быть больше, чем $\skur(p)_\gamma$.
Действительно, из \ref{prop:supporting-circline}, $\skur_\sigma\ge \skur(p)_\gamma$, ведь $\sigma$ подпирает $\gamma$ слева.
В случае равенства $\sigma$ является соприкасающейся окружностью при~$p$,
а это невозможно в силу нашего предположения.

Следовательно, $\sigma$ обязана коснуться $\gamma$ в другой точке.
Иначе её можно было бы подраздуть, оставляя внутри~$F$.
Действительно, поскольку $\skur_\sigma> \skur(p)_\gamma$, 
по \ref{prop:supporting-circline} найдётся окрестность $U$ точки $p$ так, что после небольшого раздутия $\sigma$, пересечение $U\cap \sigma$ всё ещё в~$F$.
С другой стороны, если $\sigma$ не касается $\gamma$ в другой точке, то после некоторого (возможно меньшего) раздутия $\sigma$ дополнение $\sigma\setminus U$ всё ещё в~$F$.
То есть, немного увеличенная $\sigma$ всё ещё лежит в $F$ --- противоречие.

Выберем точку $p_1$ на петле $\gamma$, отличную от базовой точки. 
Пусть $\sigma_1$ --- вписанная окружность при $p_1$,
и $\gamma_1$ --- дуга $\gamma$ от $p_1$ до первой точки $q_1$ на $\sigma_1$.
Обозначим через $\hat\sigma_1$ и $\check\sigma_1$ две дуги $\sigma_1$ от $p_1$ до $q_1$ так, что циклическое произведение $\hat\sigma_1$ и $\gamma_1$ окружает~$\check\sigma_1$. 

Пусть $p_2$ --- середина $\gamma_1$.
Обозначим через $\sigma_2$ вписанную окружность в точке $p_2$.

Окружность $\sigma_2$ не может пересечь $\hat\sigma_1$,
ведь если $\sigma_2$ пересекает $\hat\sigma_1$ в какой-то точке $s$, то она должна иметь ещё две общие точки с $\check\sigma_1$, скажем $x$ и $y$ --- по одной для каждой дуги $\sigma_2$ от $p_2$ до~$s$.
Следовательно, $\sigma_1=\sigma_2$, ведь у них три общие точки: $s$, $x$ и~$y$. 
С другой стороны, по построению, $p_2\in \sigma_2$ и $p_2\notin \sigma_1$ --- противоречие.

\begin{wrapfigure}{r}{32 mm}
\vskip-0mm
\centering
\includegraphics{mppics/pic-64}
\caption*{Два овала изображают окружности.}
\vskip-2mm
\end{wrapfigure}

Напомним, что $\sigma_2$ должна касаться $\gamma$ в другой точке.
Из вышесказанного вытекает, что $\sigma_2$ может касаться только~$\gamma_1$. 
Следовательно, можно выбрать дугу $\gamma_2\subset \gamma_1$, которая идёт от $p_2$ до первой точки $q_2$ на $\sigma_2$.
Поскольку $p_2$ --- середина $\gamma_1$, 
\[\length \gamma_2< \tfrac12\cdot\length\gamma_1.\eqlbl{eq:length<length/2}\]

Повторяя это построение рекурсивно,
получаем бесконечную последовательность дуг $\gamma_1\z\supset \gamma_2\z\supset\dots$;
по \ref{eq:length<length/2} также получим, что 
\[\length\gamma_n\to0\quad\text{когда}\quad n\to\infty.\] 
Следовательно, пересечение $\gamma_1\cap\gamma_2\cap\dots$
содержит единственную точку, скажем $p_\infty$.

Пусть $\sigma_\infty$ --- вписанная окружность при $p_\infty$; она должна касаться $\gamma$ в другой точке, скажем $q_\infty$.
То же рассуждение, что и выше, показывает, что $q_\infty\in\gamma_n$ для любого~$n$.
Следовательно, $q_\infty =p_\infty$ --- противоречие.
\qeds


\begin{thm}{Упражнение}\label{ex:moon-rad}
Пусть замкнутая гладкая кривая $\gamma$ лежит в фигуре $F$, ограниченной простой замкнутой плоской кривой.
Пусть $R$ --- это максимальный радиус кругов, которые лежат в~$F$.
Покажите, что кривизна $\gamma$ не менее $\tfrac1R$ при некотором значении параметра.
\end{thm}





\section{Теорема о четырёх вершинах}
\index{теорема о четырёх вершинах}

{

\begin{wrapfigure}{r}{20 mm}
\vskip-8mm
\centering
\includegraphics{mppics/pic-26}
\vskip0mm
\end{wrapfigure}

Напомним, что вершина гладкой кривой определяется как критическая точка её ориентированной кривизны.
В частности, точка локального минимума (или максимума) кривизны является вершиной,
а у окружности все точки вершины.

\begin{thm}{Теорема}\label{thm:4-vert}
Любая гладкая простая замкнутая плоская кривая имеет хотя бы четыре вершины.
\end{thm}

}

Ясно, что любая замкнутая гладкая кривая имеет как минимум две вершины --- точки, в которых достигаются минимум и максимум кривизны.
На рисунке отмечены вершины двух кривых;
первая имеет одно самопересечение и ровно две вершины;
у второй четыре вершины и нет самопересечений.

Изначально теорема была доказана Сьямадасом Мухопадхьяей \cite{mukhopadhyaya} для выпуклых кривых.
Одно из самых красивых доказательств предложил Роберт Оссерман \cite{osserman}.
Мы докажем следующее более сильное утверждение, используя основную лемму предыдущего раздела.
Больше по теме можно узнать в нашей статье \cite{petrunin-zamora:moon} и ссылок там.

{

\begin{wrapfigure}[7]{r}{33 mm}
\vskip-6mm
\centering
\includegraphics{mppics/pic-63}
\vskip0mm
\end{wrapfigure}

\begin{thm}{Теорема}\label{thm:4-vert-supporting}
На любой гладкой простой замкнутой плоской кривой $\gamma$ можно найти четыре точки, соприкасающиеся окружности в которых подпирают $\gamma$; две из них изнутри и две --- снаружи.
\end{thm}


\parbf{Доказательство \ref{thm:4-vert}, используя \ref{thm:4-vert-supporting}.}
Достаточно доказать следующее: \textit{если соприкасающаяся окружность $\sigma$ в точке $p$ подпирает $\gamma$ в $p$, то $p$ --- вершина~$\gamma$}.

}

Если это не так, то кривизна на малой дуге вокруг $p$ монотонна.
По лемме о спирали (\ref{lem:spiral}), соприкасающиеся окружности на этой дуге образуют монотонное семейство.
В частности, кривая $\gamma$ проходит сквозь $\sigma$ в точке~$p$, и, значит, $\sigma$ не подпирает $\gamma$.
\qeds

\parbf{Доказательство \ref{thm:4-vert-supporting}.}
По основной лемме (\ref{thm:moon}), соприкасающийся окружность при некоторой точке $p\in\gamma$ подпирает $\gamma$ изнутри.
Кривую $\gamma$ можно рассмотреть как петлю с базовой точкой~$p$.
Значит, по основной лемме, найдётся ещё одна точки $q\in\gamma$ с тем же свойством.

Это даёт пару соприкасающихся окружностей, опорных к $\gamma$ изнутри;
остаётся раздобыть ещё две.

Для получения соприкасающихся кругов, подпирающих $\gamma$ снаружи, можно повторить доказательство основной леммы, взяв вместо вписанного круга окруность (или прямую) с максимальной ориентированной кривизны, опорную к кривой снаружи, предполагая, что $\gamma$ ориентирована так, что область слева от неё ограничена.\qeds

\parbf{Другой конец доказательства.} Применим к $\gamma$ инверсию относительно окружности с центром внутри~$\gamma$. Тогда у полученной кривой, скажем $\gamma_1$, также можно найти пару соприкасающихся окружностей опорных к $\gamma_1$ изнутри.
Согласно \ref{ex:inverse}, инверсии этих окружностей соприкасающихся к~$\gamma$.
При этом область, лежащая внутри $\gamma$, отображается в область снаружи $\gamma_1$, и наоборот.
Следовательно, эти две новые окружности подпирают $\gamma$ снаружи.\qeds


{

\begin{wrapfigure}{r}{20 mm}
\vskip-0mm
\centering
\includegraphics{mppics/pic-725}
\vskip0mm
\end{wrapfigure}

\begin{thm}{Упражнение}\label{ex:2-squares}
Предположим, что гладкая простая замкнутая плоская кривая \(\gamma\) лежит в квадрате со стороной 2 и окружает квадрат с диагональю~2.
Докажите, что \(\gamma\) содержит точку с кривизной равной~1.
\end{thm}

}

\begin{thm}{Упражнение}\label{ex:moon-area}
Предположим, что гладкая простая замкнутая плоская кривая ограничивает область площадью $a$.
Докажите, что на ней найдётся точка с кривизной $\sqrt{\pi/a}$.
\end{thm}


\begin{wrapfigure}[5]{r}{25 mm}
\vskip-7mm
\centering
\includegraphics{mppics/pic-65}
\vskip0mm
\end{wrapfigure}

\begin{thm}{Продвинутое упражнение}\label{ex:curve-crosses-circle}
{\sloppy
Предположим, что простая замкнутая гладкая плоская кривая $\gamma$  пересекает окружность $\sigma$ в точках $p_1,\dots,p_{2{\cdot} n}$, и эти точки появляются в том же порядке и на $\gamma$ и на $\sigma$.
Докажите, что у $\gamma$ есть как минимум $2\cdot n$ вершин.

}

Постройте пример простой замкнутой гладкой плоской кривой $\gamma$ у которой только 4 вершины, и при этом она пересекает данную окружность в произвольно большом числе точек.
\end{thm}

{\sloppy

\begin{thm}{Продвинутое упражнение}\label{ex:berk}
Пусть $\gamma$ --- простая гладкая плоская кривая с кривизной, ограниченной~1.
Докажите, что $\gamma$ окружает два непересекающихся открытых единичных диска тогда и только тогда, когда её диаметр не менее 4;
то есть $\dist{p}{q}{}\ge 4$ для некоторых точек $p,q\in\gamma$.
\end{thm}

}

%\begin{thm}{Продвинутое упражнение}\label{ex:order} Show that the points $a,b,c,d$ guaranteed by \ref{thm:4-vert-supporting} can be chosen so that they appear in the same order on the curve and the osculating circles at $a$ and $c$ support the curve from the inside and $b$ and $d$ support the curve from the outside. \end{thm}

Следующее упражнение является версией теоремы о четырёх вершинах для пространственных кривых без параллельных касательных; такие кривые существуют по \ref{ex:no-parallel-tangents}.

\begin{thm}{Продвинутое упражнение}\label{ex:4x0-torsion}
Пусть $\gamma$ --- замкнутая гладкая пространственная кривая, у которой нет пары точек с параллельными касательными.
Допустим, что кривизна $\gamma$ не обращается в ноль ни в одной точке.
Докажите, что у $\gamma$ есть как минимум четыре точки с нулевым кручением.
\end{thm}

