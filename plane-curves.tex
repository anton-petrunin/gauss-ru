\chapter{Кривизна со знаком}\label{chap:signed-curvature}

Кривая на плоскости может поворачивать влево или вправо.
Это позволяет ввести знак кривизны плоских кривых.
Если вести машину вдоль плоской кривой, то ориентированная кривизна полностью задаёт положение руля в данный момент времени.


\section{Определения}\label{sec:def(skur)}

Пусть $\gamma$ --- гладкая плоская кривая с единичной скоростью.
Как обычно, $\tan(s)=\gamma'(s)$ --- её единичный касательный вектор при~$s$.

Повернём $\tan(s)$ на угол $\tfrac\pi 2$ против часовой стрелки; 
обозначим полученный вектор через $\norm(s)$.
Пара $\tan(s),\norm(s)$ образует ориентированный ортонормированный базис плоскости.
Он аналогичен базису Френе (см. \ref{sec:frenet-frame}) и также будет назваться \index{базис Френе}\emph{базисом Френе}.

Напомним, что $\gamma''(s)\perp \gamma'(s)$ (см. \ref{prop:a'-pertp-a''}).
Следовательно, \index{10k@$\skur$ (ориентированная кривизна)}
\[\tan'(s)=\skur(s)\cdot \norm(s).\eqlbl{eq:tau'}\]
для некоторого числа $\skur(s)$.
Величина $\skur(s)$ называется \index{ориентированная кривизна}\index{кривизна! ориентированная}\emph{ориентированной кривизной} $\gamma$ при~$s$;
её также можно обозначать через $\skur(s)_\gamma$.

Заметим, что 
\[\kur(s)=|\skur(s)|;\]
то есть, с точностью до знака, ориентированная кривизна $\skur(s)$ совпадает с обычной кривизной $\kur(s)$ (см. \ref{sec:curvature}).
Знак ориентированной кривизны указывает направление поворота: $\skur (s)>0$ если в момент времени $s$ кривая уходит влево.
Если захочется подчеркнуть, что мы работаем с обычной кривизной, 
то можно сказать \index{кривизна!без знака}\emph{кривизна без знака}.

Ориентированная кривизна меняет знак как при обращении параметризации кривой, так и при обращении ориентации плоскости.

Поскольку $\tan(s),\norm(s)$ --- ортонормированный базис, 
\begin{align*}
\langle\tan,\tan\rangle&=1,
&
\langle\norm,\norm\rangle&=1, 
&
\langle\tan,\norm\rangle&=0.
\end{align*}
Продифференцировав эти тождества, получим 
\begin{align*}
\langle\tan',\tan\rangle&=0,
&
\langle\norm',\norm\rangle&=0,
&
\langle\tan',\norm\rangle+\langle\tan,\norm'\rangle&=0.
\end{align*}
Согласно \ref{eq:tau'}, $\langle\tan',\norm\rangle=\skur$. 
Следовательно, $\langle\tan,\norm'\rangle=-\skur$, и 
\[\norm'(s)=-\skur(s)\cdot \tan(s).\eqlbl{eq:nu'}\]
Уравнения \ref{eq:tau'} и \ref{eq:nu'} называются \index{формулы Френе}\emph{формулами Френе} для плоских кривых. 
Их можно записать одним матричным уравнением:
\[
\begin{pmatrix}
\tan'
\\
\norm'
\end{pmatrix}
=
\begin{pmatrix}
0&\skur
\\
-\skur&0
\end{pmatrix}
\cdot
\begin{pmatrix}
\tan
\\
\norm
\end{pmatrix}.
\]

\begin{thm}{Упражнение}\label{ex:bike}
Пусть $\tan$ --- касательная индикатриса гладкой кривой $\gamma_0\:[a,b]\to\mathbb{R}^2$.
Рассмотрим кривую $\gamma_1\:[a,b]\to\mathbb{R}^2$, определённую как $\gamma_1(t)\z\df\gamma_0(t)+\tan(t)$.
Докажите, что

\begin{minipage}{.47\textwidth}
\begin{subthm}{ex:bike:length}$\length\gamma_0\le \length\gamma_1$;
\end{subthm}
\end{minipage}
\hfill
\begin{minipage}{.47\textwidth}
\begin{subthm}{ex:bike:tc}$\tc{\gamma_0}\le \length\gamma_1$.
\end{subthm}
\end{minipage}

\end{thm}

Кривые $\gamma_0$ и $\gamma_1$ выше описывают следы шин идеализированного велосипеда с единичным расстоянием между колёсами.
Из упражнения следует, что след  переднего колеса обычно длинней.
Больше по теме можно найти в обзоре Роберта Фута, Марка Леви и Сергея Табачникова \cite{foote-levi-tabachnikov}.


\section{Основная теорема}

\begin{thm}{Теорема}\label{thm:fund-curves-2D}
Для любой гладкой функции $s\mapsto \skur(s)$, определённой на интервале $\mathbb{I}$,
существует гладкая кривая $\gamma\:\mathbb{I}\to\mathbb{R}^2$ с единичной скоростью и ориентированной кривизной $\skur(s)$ при любом $s$.
Более того, $\gamma$ определяется однозначно с точностью до движения плоскости сохраняющего ориентацию.
\end{thm}

Теорема является частным своего трёхмерного аналога (\ref{thm:fund-curves}), но мы проведём прямое доказательство.


\parbf{Доказательство.}
Выберем $s_0\in\mathbb{I}$.
Рассмотрим функцию
\[\theta(s)
=
\int_{s_0}^s\skur(t)\cdot dt.\]
По основной теореме анализа, $\theta'(s)\z=\skur(s)$ для всех~$s$.

Пусть 
$\tan(s)=(\,\cos(\theta(s)),\,\sin(\theta(s))\,)$,
и $\norm(s)=(\,-\sin(\theta(s)),\,\cos(\theta(s))\,)$;
то есть $\norm(s)$ --- поворот $\tan(s)$ на угол $\tfrac\pi2$ против часовой стрелки.
Рассмотрим кривую 
\[\gamma(s)=\int_{s_0}^s\tan(s)\cdot ds.\]
Так как $|\gamma'|=|\tan|=1$, кривая $\gamma$ имеет единичную скорость, и $\tan,\norm$ --- её базис Френе. 

Заметим, что
\begin{align*}
\gamma''(s)&=\tan'(s)=
\\
&=(\,\cos(\theta(s))',\,\sin(\theta(s))'\,)=
\\
&=\theta'(s)\cdot (\,-\sin(\theta(s)),\,\cos(\theta(s))\,)=
\\
&=\skur(s)\cdot \norm(s).
\end{align*}
То есть, ориентированная кривизна кривой $\gamma$ при~$s$ равна $\skur(s)$. 

Существование доказано; осталась единственность.

Предположим, что $\gamma_1$ и $\gamma_2$ --- две кривые, которые удовлетворяют условиям теоремы.
Применив движение плоскости, можно добиться, чтобы совпали точки $\gamma_1(s_0)$ и $\gamma_2(s_0)$, а также базисы Френе обеих кривых при $s_0$.
Пусть $\tan_1,\norm_1$ и $\tan_2,\norm_2$ --- базисы Френе кривых $\gamma_1$ и $\gamma_2$, соответственно.
Обе тройки $\gamma_i,\tan_i,\norm_i$ удовлетворяют следующей системе обыкновенных дифференциальных уравнений 
\[
\begin{cases}
\gamma_i'=\tan_i,
\\
\tan_i'=\skur\cdot\norm_i,
\\
\norm_i'=-\skur\cdot\tan_i.
\end{cases}
\]
Более того, у обеих троек те же начальные значения в $s_0$,
а значит, $\gamma_1=\gamma_2$ по единственности решений задачи Коши (\ref{thm:ODE}).
\qeds

Пусть $\gamma\:\mathbb{I}\z\to\mathbb{R}^2$ --- кривая с единичной скоростью.
Функция $\theta\:\mathbb{I}\z\to\mathbb{R}$ называется \index{непрерывный аргумент}\emph{непрерывным аргументом} $\gamma$, если она непрерывна~и
\[\gamma'(s)=(\,\cos (\theta(s)),\,\sin(\theta(s))\,)\]
для любого $s$.
Из доказательства теоремы видно следующее.

\begin{thm}{Следствие}\label{cor:2D-angle}
Для любой гладкой кривой с единичной скоростью $\gamma\:\mathbb{I}\to\mathbb{R}^2$ существует непрерывный аргумент $\theta\:\mathbb{I}\to\mathbb{R}$.
Более того 
\[\theta'(s)=\skur(s),\]
где $\skur$ --- ориентированная кривизна~$\gamma$.
\end{thm}

\section{Полная ориентированная кривизна}\label{sec:Total signed curvature}

Пусть $\gamma\:\mathbb{I}\to\mathbb{R}^2$ --- гладкая плоская кривая с единичной скоростью.
\index{кривизна}\index{полная!ориентированная кривизна}\emph{Полная ориентированная кривизна} $\gamma$, будет обозначаться $\tgc\gamma$; она определяется как интеграл \index{10psi@$\tgc\gamma$ (полная ориентированная кривизна)}
\[\tgc\gamma
=
\int_\mathbb{I} \skur(s)\cdot ds,\eqlbl{eq:tsc-k}\]
где $\skur$ --- ориентированная кривизна~$\gamma$.

Если $\mathbb{I}=[a,b]$, то 
\[\tgc\gamma=\theta(b)-\theta(a),\eqlbl{eq:tsc-theta}\]
где $\theta$ --- непрерывный аргумент $\gamma$ (см. \ref{cor:2D-angle}).

Если $\gamma$ --- кусочно-гладкая, то её полная ориентированная кривизна определяется как сумма полных ориентированных кривизн её дуг плюс сумма \textit{ориентированных} внешних углов на стыках;
эти углы положительны, когда $\gamma$ поворачивает влево, отрицательны, когда $\gamma$ поворачивает вправо, и равны 0, когда $\gamma$ идёт прямо.
Ориентированный угол не определён, в точке возврата.

Другими словами, если $\gamma$ --- произведение гладких дуг $\gamma_1,\dots,\gamma_n$, то 
\[\tgc\gamma=\tgc{\gamma_1}+\dots+\tgc{\gamma_n}+\theta_1+\dots+\theta_{n-1},\]
где $\theta_i$ --- внешний ориентированный угол на стыке между $\gamma_i$ и $\gamma_{i+1}$.
Если $\gamma$ замкнута, то произведение циклическое, и
\[\tgc\gamma=\tgc{\gamma_1}+\dots+\tgc{\gamma_n}+\theta_1+\dots+\theta_{n},\]
где $\theta_n$ --- внешний ориентированный угол на стыке между $\gamma_n$ и $\gamma_1$.

Поскольку $\left|\int \skur(s)\cdot ds\right|\le \int|\skur(s)|\cdot ds$, неравенство
\[|\tgc\gamma|\le \tc\gamma\eqlbl{eq:tsc-tc}\] 
выполняется для любой гладкой кривой $\gamma$ на плоскости;
то есть полная ориентированная кривизна $\tgc{}$ не превышает полную кривизну $\tc{}$ по абсолютной величине.
При этом равенство достигается тогда и только тогда, когда ориентированная кривизна не меняет знак.


\begin{thm}{Упражнение}\label{ex:trochoids}
Трохоида --- это кривая, описываемая точкой, закреплённой на колесе, катящемся по прямой линии.
\begin{figure}[!ht]
\centering
\begin{lpic}[t(-0mm),b(0mm),r(0mm),l(0mm)]{asy/trochoids}
\lbl[l]{4,0;{\footnotesize $-\tfrac32$}}
\lbl[l]{4,6;{\footnotesize $-1$}}
\lbl[tl]{10,15;{\footnotesize $-\tfrac12$}}
\lbl[t]{22,17;{\footnotesize $0$}}
\lbl[r]{3,23.6;{\footnotesize $\tfrac12$}}
\lbl[r]{3,29.4;{\footnotesize $1$}}
\lbl[r]{3,35.2;{\footnotesize $\tfrac32$}}
\end{lpic}
\end{figure}
Семейство \index{трохоида}\emph{трохоид} $\gamma_a\:[0,2\cdot\pi]\to \mathbb{R}^2$ (см. рисунок) можно параметризовать как
\[\gamma_a(t)=(t+a\cdot \sin t, a\cdot \cos t).\]
\begin{enumerate}[(a)]
\item Для данного $a\in \mathbb{R}$ найдите $\tgc{\gamma_a}$, если она определена.
\item Для данного $a\in \mathbb{R}$ найдите $\tc{\gamma_a}$.
\end{enumerate}
\end{thm}

\begin{thm}{Предложение}\label{prop:total-signed-curvature}
Полная ориентированная кривизна любой простой замкнутой гладкой плоской кривой $\gamma$ равна $\pm2\cdot\pi$; это значение равно $+2\cdot\pi$, если область, ограниченная $\gamma$, находится слева от неё, и $-2\cdot\pi$ в противном случае.

Более того, то же выполняется и для любой простой замкнутой кусочно-гладкой кривой $\gamma$ на плоскости, если её полная ориентированная кривизна определена.
\end{thm}

Предложение даёт дифференциально-геометрический аналог теоремы о сумме внутренних углов многоугольника (\ref{thm:sum=(n-2)pi}), которую мы будем использовать в доказательстве.
Более концептуальное доказательство нашёл Хайнц Хопф \cite{hopf1935}, \cite[с. 42]{hopf1989}.

\parbf{Доказательство.}
Не умаляя общности, можно предположить, что $\gamma$ ориентирована так, что область, ограниченная $\gamma$, находится слева от неё.
Также можно предположить, что $\gamma$ имеет единичную скорость.

Рассмотрим замкнутую ломаную $p_1\dots p_n$, вписанную в~$\gamma$.
Можно считать, что дуги между вершинами настолько малы, что ломаная проста, и каждая дуга $\gamma_i$ от $p_i$ до $p_{i+1}$ имеет малую полную кривизну; скажем, $\tc{\gamma_i}<\pi$ при любом~$i$.


{

\begin{wrapfigure}[13]{o}{41 mm}
\vskip-4mm
\centering
\includegraphics{mppics/pic-59}
\vskip0mm
\end{wrapfigure}

Пусть $p_i=\gamma(t_i)$;
как обычно, будем считать индексы по модулю $n$; в частности, $p_{n+1}\z=p_1$.
Тогда
\begin{align*}
\vec w_i&=p_{i+1}-p_i,& \vec v_i&=\gamma'(t_i),
\\
\alpha_i&=\measuredangle(\vec v_i,\vec w_i),&\beta_i&=\measuredangle(\vec w_{i-1},\vec v_i),
\end{align*}
здесь $\alpha_i,\beta_i\in(-\pi,\pi)$ --- ориентированные углы: $\alpha_i$ положительный, если $\vec w_i$ направлен левее~$\vec v_i$.

}

Согласно \ref{eq:tsc-theta}, величина
\[\tgc{\gamma_i}-\alpha_i-\beta_{i+1}\eqlbl{eq:Psi-alpha-beta}\]
 кратна $2\cdot\pi$.
Поскольку $\tc{\gamma_i}<\pi$, по лемме о хорде (\ref{lem:chord}), $|\alpha_i|\z+|\beta_i|<\pi$.
Из \ref{eq:tsc-tc}, получаем $|\tgc{\gamma_i}|\z\le\tc{\gamma_i}$;
следовательно, величина в \ref{eq:Psi-alpha-beta} равна нулю.
То есть
\[\tgc{\gamma_i}=\alpha_i+\beta_{i+1}\]
для каждого $i$ имеем


Заметим, что 
\[\delta_i=\pi-\alpha_i-\beta_i\eqlbl{eq:delta=pi-alpha-beta}\] 
--- внутренний угол ломаной при $p_i$;
$\delta_i\in (0,2\cdot\pi)$ для каждого~$i$.
Напомним, что сумма внутренних углов $n$-угольника равна $(n-2)\cdot \pi$ (\ref{thm:sum=(n-2)pi}); то есть
\[\delta_1+\dots+\delta_n=(n-2)\cdot \pi.\]
Следовательно, 
\[
\begin{aligned}
\tgc\gamma&=\tgc{\gamma_1}+\dots+\tgc{\gamma_n}=
\\
&=(\alpha_1+\beta_2)+\dots+(\alpha_n+\beta_1)=
\\
&=(\beta_1+\alpha_1)+\dots+(\beta_n+\alpha_n)=
\\
&=(\pi-\delta_1)+\dots+(\pi-\delta_n)=
\\
&=n\cdot\pi-(n-2)\cdot \pi=
\\
&=2\cdot\pi.
\end{aligned}\eqlbl{eq:delta=pi-alpha-beta-sum}\]

{\sloppy

Случай кусочно-гладких кривых аналогичен;
потребуется дополнительно подразбить дуги циклического произведения чтобы выполнялось вышеуказанное условие.
Вместо \ref{eq:delta=pi-alpha-beta}, придётся пользоваться уравнением
\[\delta_i=\pi-\alpha_i-\beta_i-\theta_i,\]
где $\theta_i$ --- ориентированный внешний угол кривой $\gamma$ при $p_i$;
он равен нулю, если кривая $\gamma$ гладкая в $p_i$.
Вычисления в \ref{eq:delta=pi-alpha-beta-sum} превращаются в следующее:
\begin{align*}
\tgc\gamma&=\tgc{\gamma_1}+\dots+\tgc{\gamma_n}+\theta_1+\dots+\theta_n=
\\
&=(\alpha_1+\beta_2)+\dots+(\alpha_n+\beta_1)+\theta_1+\dots+\theta_n=
\\
&=(\beta_1+\alpha_1+\theta_1)+\dots+(\beta_n+\alpha_n+\theta_n)=
\\
&=(\pi-\delta_1)+\dots+(\pi-\delta_n)=
\\
&=n\cdot\pi-(n-2)\cdot \pi=
\\
&=2\cdot\pi.
\end{align*}
\qedsf

}

\begin{thm}{Упражнение}\label{ex:zero-tsc}
Нарисуйте такую гладкую замкнутую кривую $\gamma$ на плоскости, что

\begin{subthm}{ex:zero-tsc:0}
$\tgc\gamma=0$;
\end{subthm}
 
\begin{subthm}{ex:zero-tsc:5}
$\tgc\gamma=\tc\gamma=10\cdot\pi$;
\end{subthm}

\begin{subthm}{ex:zero-tsc:2-4}
$\tgc\gamma=2\cdot\pi$ и $\tc\gamma=4\cdot\pi$.
\end{subthm}

\end{thm}

\begin{thm}{Упражнение}\label{ex:length'}
Пусть $\gamma\:[a,b]\to\mathbb{R}^2$ --- гладкая кривая на плоскости, и $\tan,\norm$ --- её базис Френе.
Для данного параметра $\ell$, рассмотрим
кривую $\gamma_\ell(t)\z=\gamma(t)\z+\ell\cdot\norm(t)$; она называется \index{параллельная кривая}\emph{параллельной} к $\gamma$  кривой с параметром $\ell$.

\begin{subthm}{ex:length':reg}
Покажите, что параметризация $\gamma_\ell$ регулярна, если $\ell\cdot \skur(t)_\gamma\ne 1$ для всех $t$.
\end{subthm}
 
\begin{subthm}{ex:length':formula}
Рассмотрим функцию $L(\ell)=\length\gamma_\ell$.
Покажите, что 
\[L(\ell)=L(0)-\ell\cdot\tgc\gamma\eqlbl{eq:length(parallel-curve)}\]
при всех $\ell$, достаточно близких к $0$. 
\end{subthm}

\begin{subthm}{ex:length':antiformula}
Приведите пример, показывающий, что формула \ref{eq:length(parallel-curve)} может не выполняется при некоторых значениях~$\ell$. 
\end{subthm}

\end{thm}

\section{Соприкасающаяся окружность}

\begin{thm}{Предложение}\label{prop:circle}
Для данной точки $p\in\mathbb{R}^2$,
единичного вектора $\tan$ 
и вещественного числа $\skur$ существует единственная гладкая кривая $\sigma\:\mathbb{R}\to\mathbb{R}^2$ с единичной скоростью, которая начинается в $p$ в направлении $\tan$ и имеет постоянную ориентированную кривизну $\skur$.

Более того, если $\skur=0$, то это прямая $\sigma(s)=p+s\cdot \tan$;
если $\skur\ne 0$, то $\sigma$ описывает окружность радиуса $\tfrac1{|\skur|}$ с центром в $p+\tfrac1\skur\cdot \norm$, где $\tan,\norm$ --- это ориентированный ортонормальный базис.
\end{thm}

\parbf{Доказательство.}
Выберем систему координат с началом отсчёта в $p$, чтобы $\tan$ шёл по оси $x$.
В этом случае $\norm$ идёт по оси $y$.

Заметим, что
\begin{align*}\theta(s)&=\int_{0}^s\skur\cdot dt=\skur\cdot s
\end{align*}
является непрерывным аргументом $\gamma$, см. \ref{cor:2D-angle}.
Следовательно,
\[\sigma'(s)=(\,\cos(\skur\cdot s),\,\sin(\skur\cdot s)\,).\]
Остаётся проинтегрировать последнее равенство.
Если $\skur=0$, то получаем $\sigma(s)=(s,0)$,
что описывает прямую $\sigma(s)=p+s\cdot \tan$.

Если $\skur\ne 0$, то получаем
\[\sigma(s)=(\,\tfrac1\skur\cdot\sin(\skur\cdot s),\, \tfrac1\skur\cdot(1-\cos(\skur\cdot s))\,);\]
это окружность радиуса $r=\tfrac1{|\skur|}$ и центром в $(0,\tfrac1\skur)=p+\tfrac1\skur\cdot\norm$.
\qeds

\begin{thm}{Определение}
Пусть $\gamma$ --- гладкая плоская кривая с единичной скоростью,
и $\skur(s)$ --- её ориентированная кривизна при~$s$.

Кривая $\sigma_s$ с единичной скоростью постоянной ориентированной кривизны $\skur(s)$, которая начинается в $\gamma(s)$ в направлении $\gamma'(s)$, называется \index{соприкасающаяся!окружность}\emph{соприкасающейся окружностью} кривой $\gamma$ при~$s$.

Центр и радиус соприкасающейся окружности называются \index{центр кривизны}\emph{центром кривизны} и \index{радиус кривизны}\emph{радиусом кривизны} кривой в соответственной точке.
\end{thm}

{

\begin{wrapfigure}{o}{31 mm}
\vskip-0mm
\centering
\includegraphics{mppics/pic-21}
\vskip0mm
\end{wrapfigure}

\textit{Соприкасающейся окружность} может оказаться окружностью или прямой;
в последнем случае центр кривизны не определён, а радиус кривизны можно считать бесконечным.  

Соприкасающаяся окружность $\sigma_s$ также может быть определена как единственная окружность (или прямая), имеющая \index{порядок касания}\emph{второй порядок касания} с $\gamma$ при $s$;
то есть $\rho(\ell)\z=o(\ell^2)$, где $\rho(\ell)$ обозначает расстояние от $\gamma(s+\ell)$ до $\sigma_s$.

Следующее упражнение рекомендуется читателю, знакомому с понятием \index{инверсия}\emph{инверсии}.

}

\begin{thm}{Продвинутое упражнение}\label{ex:inverse}
Пусть $\gamma$ --- гладкая кривая на плоскости, которая не проходит через начало координат.
Рассмотрим инверсию $\hat \gamma$ кривой $\gamma$ относительно окружности с центром в начале координат.
Покажите, что соприкасающаяся к $\hat\gamma$ окружность при $s$ является инверсией соприкасающейся к $\gamma$ окружности  при~$s$.
\end{thm}

\section{Лемма о спирали}
\label{spiral}
\index{лемма о спирали}

Следующая лемма была доказана Питером Тейтом \cite{tait}
и переоткрыта Адольфом Кнезером \cite{kneser}.


\begin{thm}{Лемма}\label{lem:spiral}
Пусть $\gamma$ --- гладкая плоская кривая со строго убывающей положительной кривизной.
Тогда соприкасающиеся к $\gamma$ окружности образуют монотонное семейство;
то есть, если $\sigma_s$ --- соприкасающаяся к $\gamma$ окружность при $s$,
то для любых $s_0<s_1$ окружность $\sigma_{s_0}$ лежит внутри открытого круга, ограниченного $\sigma_{s_1}$.\index{10sigma@$\sigma_s$ (соприкасающаяся окружность)}
\end{thm}

{

\begin{wrapfigure}{r}{31 mm}
\vskip-0mm
\includegraphics{mppics/pic-61}
\end{wrapfigure}

На рисунке изображена кривая с её соприкасающимися окружностями.
Они заполняют кольцо и делают это довольно причудливым образом: если гладкая функция постоянна на каждой окружности, то она постоянна во всём кольце \cite[Лекция 10]{fuchs-tabachnikov}.
Также отметим, что кривая $\gamma$ касается окружности из этого семейства в каждой из своих точек.
Однако, она не идёт ни по одной из этих окружностей.

}

\parbf{Доказательство.}
Пусть $\tan(s),\norm(s)$ --- базис Френе.
Обозначим через $\omega(s)$ и $r(s)$
центр и радиус кривизны~$\gamma$;
согласно \ref{prop:circle},
\[\omega(s)=\gamma(s)+r(s)\cdot \norm(s).\]

Поскольку кривизна положительна, $r(s)\cdot\skur(s)=1$.
Применив формулу Френе \ref{eq:nu'}, получим
\begin{align*}
\omega'(s)&=\gamma'(s)+r'(s)\cdot \norm(s)+r(s)\cdot \norm'(s)=
\\
&=\tan(s)+r'(s)\cdot \norm(s)-r(s)\cdot \skur(s)\cdot \tan(s)=
\\
&=r'(s)\cdot \norm(s).
\end{align*}
Так как $\skur(s)$ убывает, $r(s)$ возрастает;
следовательно $r'\ge 0$.
Из этого вытекает, что $|\omega'(s)|= r'(s)$ и $\omega'(s)$ направлена вдоль $\norm(s)$.

Поскольку $\norm'(s)=-\skur(s)\cdot\tan(s)$, направление $\omega'(s)$ не может оставаться постоянным на любом интервале;
другими словами, кривая $s\mapsto \omega(s)$ не содержит отрезков прямых.

\begin{wrapfigure}[7]{o}{35 mm}
\vskip-0mm
\centering
\includegraphics{mppics/pic-84}
\end{wrapfigure}

В частности, 
\[|\omega(s_1)-\omega(s_0)|\z<\length(\omega|_{[s_0,s_1]})\]
при любых $s_0<s_1$.
Следовательно, 
\begin{align*}
|\omega(s_1)-\omega(s_0)|&<\length(\omega|_{[s_0,s_1]})=
\\
&=\int_{s_0}^{s_1}|\omega'(s)|\cdot ds=
\\
&=\int_{s_0}^{s_1}r'(s)\cdot ds=
\\
&=r(s_1)-r(s_0).
\end{align*}
Другими словами, расстояние между центрами $\sigma_{s_1}$ и $\sigma_{s_0}$
строго меньше разницы между ихними радиусами, отсюда лемма.
\qeds

Кривая $s\mapsto \omega(s)$ называется \index{эволюта}\emph{эволютой} $\gamma$;
она описывает центры кривизны данной кривой и записывается как
\[\omega(t)=\gamma(t)+\tfrac1{\skur(t)}\cdot \norm(t).\]
Из доказательства следует, что $(\tfrac1{\skur})'\cdot\norm$ --- её вектор скорости.

\begin{thm}{Упражнение}\label{ex:evolute}
Пусть $\omega$ --- эволюта гладкой кривой~$\gamma$ на плоскости.
Предположим, что у $\gamma$ положительная кривизна $\skur$ и $\skur ^{\prime} \neq 0$ во всех точках.
%%%%%%% Вершины ещё не определены.
Найдите базис Френе и ориентированную кривизну эволюты $\omega$ через $\skur$ и базис Френе $(\tan,\norm)$ кривой $\gamma$.
\end{thm}

Про следующую теорему можно думать так:
\textit{если ехать по плоскости на машине и всё время поворачивать руль влево,
то невозможно вернуться на исходное место.}

\begin{thm}{Теорема}\label{thm:spiral}
Любая гладкая плоская кривая $\gamma$ с положительной строго монотонной кривизной не имеет самопересечений.
\end{thm}

Теорема остаётся верной и без предположения о положительности кривизны; доказательство требует лишь незначительных изменений.

\parbf{Доказательство.}
Заметим, что $\gamma(s)\in \sigma_s$, где  $\sigma_s$ --- соприкасающейся к $\gamma$ окружность при~$s$.
Из \ref{lem:spiral} следует, что $\sigma_{s_0}$ не пересекает $\sigma_{s_1}$ при $s_1\ne s_0$.
В частности, $\gamma(s_1)\ne \gamma(s_0)$, и теорема следует.
\qeds

{\sloppy

\begin{thm}{Продвинутое упражнение}\label{ex:3D-spiral}
Убедитесь, что трёхмерный аналог теоремы неверен.
То есть постройте гладкую пространственную кривую с пересечением и строго монотонной кривизной.
\end{thm}

}

\begin{thm}{Упражнение}\label{ex:double-tangent}
Предположим, что $\gamma$ --- гладкая плоская кривая с положительной строго монотонной кривизной.

\begin{subthm}{ex:double-tangent:a}
Покажите, что ни одна прямая не касается $\gamma$ в двух различных точках.
\end{subthm}

\begin{subthm}{ex:double-tangent:b}
Покажите, что ни одна окружность не касается $\gamma$ в трёх различных точках.
\end{subthm}

\end{thm}

{

\begin{wrapfigure}{o}{25 mm}
\vskip-4mm
\centering
\includegraphics{mppics/pic-25}
\vskip0mm
\end{wrapfigure}

Часть \ref{SHORT.ex:double-tangent:a} перестаёт быть верной, если разрешить отрицательную кривизну; пример показан на рисунке.

}

\begin{thm}{Продвинутое упражнение}\label{ex:spherical-spiral}
Покажите, что гладкая сферическая кривая с ненулевым кручением не имеет самопересечений.
\end{thm}



