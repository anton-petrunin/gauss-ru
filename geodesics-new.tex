\chapter{Геодезические}
\label{chap:geodesics}

\section{Определение}

Гладкая кривая $\gamma$ на гладкой поверхности называется \index{геодезическая}\emph{геодезической}, если её ускорение $\gamma''(t)$ перпендикулярно касательной плоскости $\T_{\gamma(t)}$ при любом~$t$.

\begin{thm}{Упражнение}\label{ex:helix-geodesic}
Покажите, что на цилиндрической поверхности $x^2+y^2=1$,
винтовая линия $\gamma(t)\z\df(\cos t,\sin t, t)$ является геодезической.
\end{thm}

Геодезические описывают траектории частиц, скользящих по $\Sigma$ без трения и сторонних сил.
Ведь если трения нет, то сила, которая удерживает частицу на $\Sigma$, должна быть перпендикулярна~$\Sigma$, и, по второму закону Ньютона, ускорение $\gamma''$ перпендикулярно $\T_{\gamma(t)}$.

Следующая лемма выводится из закона сохранения энергии.

\begin{thm}{Лемма}\label{lem:constant-speed}
Пусть $\gamma$ --- геодезическая на гладкой поверхности~$\Sigma$. 
Тогда её скорость $|\gamma'|$ постоянна.
Более того, для любого $\lambda\in\mathbb{R}$ кривая 
$\gamma_{\lambda}(t)\df \gamma (\lambda\cdot t)$ также геодезическая. 

\end{thm}

Иными словами, скорость геодезической постоянна, и умножение её параметра на константу оставляет её геодезической.

\parbf{Доказательство.} 
Отметим, что $\gamma''(t)\z\perp\gamma'(t)$,
ибо $\gamma'(t)$ касается поверхности в точке $\gamma(t)$, а $\gamma''(t)$ перпендикуляна к касателной плоскости.
Иначе говоря, $\langle\gamma'',\gamma'\rangle=0$ для любого~$t$.
Отсюда  $\langle\gamma',\gamma'\rangle'\z=2\cdot \langle\gamma'',\gamma'\rangle=0$.
То есть, величина $|\gamma'|^2=\langle\gamma',\gamma'\rangle$ постоянна.

Поскольку $\gamma_{\lambda}''(t) =\lambda^2\cdot \gamma''(\lambda t)$,
получаем вторую часть леммы.
\qeds

{\sloppy

Следующее упражнение описывает \index{соотношение Клеро}\emph{соотношение Клеро};
его можно вывести из закона сохранения углового момента.

}

\begin{thm}{Упражнение}\label{ex:clairaut}
Пусть $\gamma$ --- геодезическая на гладкой поверхности вращения,
$r(t)$ --- расстояние от $\gamma(t)$ до оси вращения
и $\theta(t)$ --- угол между $\gamma'(t)$ и параллелью поверхности, проходящей через $\gamma(t)$. 

Покажите, что величина $r(t)\cdot \cos\theta(t)$ не меняется. 
\end{thm}

Напомним, что {}\emph{асимптотическая линия} --- это кривая на поверхности с нулевой нормальной кривизной.

\begin{thm}{Упражнение}\label{ex:asymptotic-geodesic}
Предположим, что кривая $\gamma$ одновременно и геодезическая и асимптотическая линия гладкой поверхности.
Покажите, что $\gamma$ --- прямолинейный отрезок.
\end{thm}

\begin{thm}{Упражнение}\label{ex:reflection-geodesic}
Пусть гладкая поверхность $\Sigma$ пересекается со своей плоскостью симметрии по гладкой кривой~$\gamma$.
Покажите, что~$\gamma$, параметризованная длиной, является геодезической на~$\Sigma$.
\end{thm}

\section{Существование и единственность}

{\sloppy

Следующее предложение говорит, что движение без трения и внешних сил, гладко описывается через начальные данные.
В формулировке используется гладкость отображения $w$, которое выдаёт точку в $\mathbb{R}^3$ по точке $p$ поверхности $\Sigma$, касательному в ней вектору $\vec{v}$ и вещественному параметру $t$.
Если выбрать карту $s$ на $\Sigma$, то в ней точка $p$ задаётся парой координат $(u,v)$, а вектор $\vec{v}$ можно предсатвить как сумму $a\cdot s_u+b\cdot s_v$.
Поэтому в локальных координатах $w$ задаётся отображением $(u,v,a,b,t)\mapsto w(p,\vec v,t)$ из подмножества $\mathbb{R}^5$ в $\mathbb{R}^3$, а к нему уже применимо обычное определение гладкости.
Само отображение $(p,\vec v, t)\z\mapsto w(p,\vec v, t)$ считается \index{гладкое отображение}\emph{гладким}, если оно описывается гладким отображением $(u,v,a,b,t)\mapsto w(p,\vec v,t)$ в любых локальных координатах на $\Sigma$.

}

\begin{thm}{Предложение}\label{prop:geod-existence} 
Пусть $\Sigma$ --- гладкая поверхность без края.
Для вектора ${\vec v}$ касательного к $\Sigma$ в точке $p$ существует единственная геодезическая $\gamma\:\mathbb{I}\to \Sigma$, выходящая из $p$ со скоростью~${\vec v}$ (то есть $\gamma(0)=p$ и $\gamma'(0)={\vec v}$) и определённая на максимальном открытом интервале $\mathbb{I}\ni 0$.
Более того,
\begin{subthm}{prop:geod-existence:smooth}
Отображение $w\:(p,{\vec v},t)\mapsto \gamma(t)$ гладкое, и его область определения открыта%
\footnote{то есть если $w$ определено для тройки $p\in \Sigma$, ${\vec v}\in \T_p$ и $t\in \mathbb{R}$,
то оно определено для всякой тройки  $q\in \Sigma$, $\vec u\in \T_p$ и $s\in \mathbb{R}$, где $q$, $\vec u$ и $s$ достаточно близки к $p$, ${\vec v}$ и $t$ соответственно.}%
.
\end{subthm}

\begin{subthm}{prop:geod-existence:whole}
Если $\Sigma$ собственная поверхность, то $\mathbb{I}=\mathbb{R}$; то есть $\gamma$ определена на всей вещественной прямой.
\end{subthm}

\end{thm}

Поверхность, в которой все геодезические бесконечно продолжимы в обе стороны, называется \index{геодезически полная}\emph{геодезически полной}.
Таким образом, согласно \ref{SHORT.prop:geod-existence:whole}, \textit{любая собственная поверхность без края геодезически полна}.
Это часть \index{теорема Хопфа --- Ринова}\emph{теоремы Хопфа --- Ринова} \cite{hopf-rinow}.

В доказательстве мы перепишем определение геодезической через дифференциальное уравнение, а потом применим теоремы \ref{thm:ODE-nth-order} и \ref{thm:ODE}.



\begin{thm}{Лемма}\label{lem:geodesic=2nd-order}
Пусть $f$ --- гладкая функция, определённая на открытой области в $\mathbb{R}^2$.
Гладкая кривая $t\mapsto \gamma(t)=(x(t),y(t),z(t))$ является геодезической на графике $z=f(x,y)$ тогда и только тогда, когда $z(t)=f(x(t),y(t))$ для любого $t$, а функции $t\mapsto x(t)$ и $t\mapsto y(t)$ удовлетворяют системе дифференциальных уравнений вида
\[
\begin{cases}
x''=g(x,y,x',y'),
\\
y''=h(x,y,x',y'),
\end{cases}
\]
где $g$ и $h$ --- гладкие функции четырёх переменных, однозначно определяемые функцией~$f$.
\end{thm}

\parbf{Доказательство.}
Первое уравнение $z(t)=f(x(t),y(t))$ означает, что $\gamma(t)$ лежит на графике $z=f(x,y)$.

Далее будем опускать аргументы функций; 
то есть пользоваться сокращениями $x=x(t)$, $f\z=f(x,y)=f(x(t),y(t))$ и так далее.

Сначала выразим $z''$ через $f$, $x$ и $y$.
\[
\begin{aligned}
z''&=f(x,y)''=
\\
&=\left(f_x\cdot x'+ f_y\cdot y'\right)'=
\\
&=
f_{xx}\cdot (x')^2
+
f_x\cdot x''
+ 2\cdot f_{xy}\cdot x'\cdot y'
+
f_{yy}\cdot (y')^2
+
f_y\cdot y''.
\end{aligned}
\eqlbl{eq:def-geod}
\]

Условие
\[\gamma''(t)\perp\T_{\gamma(t)}\] 
означает, что 
вектор $\gamma''$ перпендикулярен двум базисным векторам в $\T_{\gamma(t)}$, то есть
\[
\begin{cases}
\langle \gamma'',s_x\rangle=0,
\\
\langle \gamma'',s_y\rangle=0,
\end{cases}
\]
где $s(x,y)\df (x,y,f(x,y))$, $x=x(t)$ и $y=y(t)$.
Обратите внимание, что 
$s_x=(1,0,f_x)$ 
и 
$s_y=(0,1,f_y)$.
Поскольку $\gamma''\z=(x'',y'',z'')$, наша система переписывается как
\[
\begin{cases}
x''+ f_x\cdot z''=0,
\\
y''+ f_y\cdot z''=0.
\end{cases}
\]
Остаётся выразить $z''$ из \ref{eq:def-geod}, привести подобные и упростить.
\qeds

\parbf{Доказательство \ref{prop:geod-existence}.}
Пусть $\Sigma$ представляется графиком $z=f(x,y)$ в касательно-нормальных координатах при~$p$.
По лемме~\ref{lem:geodesic=2nd-order}, условие $\gamma''(t)\perp\T_{\gamma(t)}$ записывается системой дифференциальных уравнений второго порядка.
Из \ref{thm:ODE-nth-order} и \ref{thm:ODE}, получаем существование и единственность геодезической $\gamma$ на интервале $(-\epsilon,\epsilon)$ для некоторого $\epsilon>0$.

Продолжим геодезическую $\gamma$ до максимального открытого интервала $\mathbb{I}$.
Пусть $\gamma_1$ --- другая геодезическая с теми же начальными данными, определённая на максимальном открытом интервале $\mathbb{I}_1$.
Допустим, что $\gamma_1$ расходится с $\gamma$ при некотором $t_0>0$;
то есть $\gamma_1$ и $\gamma$ совпадают на интервале $[0,t_0)$, но различаются на любом интервале $[0,t_0+\delta)$ при $\delta>0$.
По непрерывности, $\gamma_1(t_0)=\gamma(t_0)$ и $\gamma_1'(t_0)=\gamma'(t_0)$.
Снова применив \ref{thm:ODE-nth-order} и \ref{thm:ODE}, получаем, что $\gamma_1$ совпадает с $\gamma$ в небольшой окрестности $t_0$ --- противоречие.

Аналогичное рассуждение показывает, что $\gamma_1$ не может разойтись с $\gamma$ при $t_0<0$.
Следовательно, $\gamma_1=\gamma$, и в частности, $\mathbb{I}_1=\mathbb{I}$.

Если вся поверхность $\Sigma$ является графиком гладкой функции, то часть \ref{SHORT.prop:geod-existence:smooth} следует из \ref{thm:ODE-nth-order}, \ref{thm:ODE} и леммы.
В этом случае отображение
\[\vec{w}(p,\vec v,t)\df\tfrac{\partial}{\partial t}w(p,\vec v,t)\] также гладкое.
Заметим, что $\vec{w}(p,\vec v,t_0) \in \T_{\gamma(t_0)}$ есть вектор скорости геодезической $\gamma\:t\mapsto w(p,\vec v,t)$ в момент времени $t_0$.

В общем случае пусть $w(p,{\vec v},b)$ определено при $b\ge0$; то есть геодезическая $\gamma\:t\mapsto w(p,{\vec v},t)$ определена в интервале $[0,b]$.
Тогда найдётся такое разбиение $0=t_0<t_1<\dots<t_n=b$ интервала $[0,b]$, что каждая из геодезических $\gamma|_{[t_{i-1},t_i]}$ накрывается картой заданной некоторыми касательно-нормальными координатами.
Пусть $p_i=\gamma(t_i)$ и $\vec v_i=\gamma'(t_i)$ так, что $p_0=p$ и $\vec v_0=\vec v$.
Поскольку $\gamma|_{[t_{i-1},t_i]}$ лежит в графике, по предыдущему рассуждению получаем, что для любого $i$ отображения
$w$ и $\vec w$ определены и гладки в окрестности троек $(p_{i-1},\vec v_{i-1}, t_i-t_{i-1})$.
Заметим, что $p_i=w(p_{i-1},\vec v_{i-1},t_i-t_{i-1})$ и $\vec v_i=\vec w(p_{i-1},\vec v_{i-1},t_i-t_{i-1})$ для любого~$i$.
Поскольку композиция гладких отображений гладкая, отображение $w$ гладко определено в окрестности тройки $(p,\vec v, b)$.

Случай $b\le 0$ аналогичен, и мы получили общий случай в \ref{SHORT.prop:geod-existence:smooth}.

Допустим, что \ref{SHORT.prop:geod-existence:whole} не выполняется;
то есть максимальный интервал $\mathbb{I}$ строго содержится в $\mathbb{R}$.
Не умаляя общности, можно считать, что $b=\sup\mathbb{I}<\infty$.
(Если нет, то обратим параметризацию кривой~$\gamma$.)

{\sloppy

Согласно \ref{lem:constant-speed}, $|\gamma'|$ постоянно; в частности, функция $t\mapsto \gamma(t)$ равномерно непрерывна.
Следовательно, предельная точка
$q\z=\lim_{t\to b}\gamma(t)$
определена, и $q\in \Sigma$, ибо $\Sigma$ собственная.

}

Применив рассуждение выше в касательно-нормальных координатах при $q$, заключаем, что $\gamma$ можно продолжить геодезической за~$q$.
Следовательно, интервал $\mathbb{I}$ не максимален --- противоречие.
\qeds

\begin{thm}{Упражнение}\label{ex:round-torus}
Пусть $\Sigma$ --- гладкий тор вращения; то есть поверхность вращения с гладкой замкнутой образующей кривой.
Докажите, что любая замкнутая геодезическая на $\Sigma$ не стягиваема.

(Другими словами, если $s\:\mathbb{R}^2\to \Sigma$ является естественной би-периодической параметризацией $\Sigma$, то
не существует замкнутой такой кривой $\gamma$ в $\mathbb{R}^2$, что $s\circ\gamma$ является геодезической.)
\end{thm}


\section{Экспоненциальное отображение}\label{sec:exp}

Пусть $p$ --- точка гладкой поверхности $\Sigma$.
Для касательного вектора ${\vec v}\in \T_p$, рассмотрим геодезическую $\gamma_{\vec v}$ на поверхности, выходящую из $p$ с начальной скоростью~$\vec v$;
то есть $\gamma(0)=p$ и $\gamma'(0)={\vec v}$.

Определим \index{экспоненциальное отображение}\emph{экспоненциальное отображение}%
\footnote{Объяснение причины этого термина увело бы нас слишком далеко в сторону.}
в точке $p$ как
\[\exp_p\:\vec v\mapsto \gamma_{\vec v}(1).\]
Согласно \ref{prop:geod-existence}, это отображение гладкое и определено в окрестности нуля касательной плоскости $\T_p$;
более того, если поверхность $\Sigma$ собственная, 
то $\exp_p$ определено на всей плоскости $\T_p$.

Экспоненциальное отображение $\exp_p$ переводит одну поверхность в другую;
первая поверхность это касательная плоскость (или её открытое подмножество), а вторая $\Sigma$.
Плоскость $\T_p$ можно отождествить 
со своей касательной плоскостью $\T_0\T_p$, так что дифференциал $d_0(\exp_p)\:\vec v\mapsto D_{\vec v}\exp_p$ отображает $\T_p$ в себя.
Кроме того, по лемме~\ref{lem:constant-speed}, этот дифференциал является тождественным отображением; то есть $(d_0\exp_p)(\vec v)=
\vec v$ для любого $\vec v\in \T_p$.
Получаем следующее.

\begin{thm}{Наблюдение}\label{obs:d(exp)=1}
Пусть $p$ --- точка гладкой поверхности $\Sigma$.

{\sloppy

\begin{subthm}{}
Экспоненциальное отображение $\exp_p$ гладкое, и его область определения $\Dom(\exp_p)$ содержит окрестность нуля в $\T_p$.
Более того, если $\Sigma$ собственная, то $\Dom(\exp_p)=\T_p$.
\end{subthm}

}

\begin{subthm}{}
Дифференциал $d_0(\exp_p)\:\T_p\to \T_p$ является тождественным отображением.
\end{subthm}

\end{thm}

На самом деле легко проверить, что область $\Dom(\exp_p)$ \index{звёздное множество}\emph{звёздная} в $\T_p$;
то есть $\lambda\cdot\vec v\in \Dom(\exp_p)$ если $\vec v\in \Dom(\exp_p)$ и $0\le \lambda\le 1$.

\pagebreak[4]%???

\section{Радиус инъективности}


\index{радиус инъективности}\emph{Радиусом инъективности} $\inj(p)$ поверхности $\Sigma$ в точке $p$ называется максимальный радус $r_p\ge 0$ такой, что экспоненциальное отображение $\exp_p$ определено на открытом шаре $B_p\z\df B(0,r_p)_{\T_p}$,
и сужение $\exp_p|_{B_p}$ является гладкой регулярной параметризацией окрестности $p$ в~$\Sigma$;


\begin{thm}{Предложение}\label{prop:exp}
Радиус инъективности положителен в любой точке гладкой поверхности $\Sigma$ (без края).
Более того, он {}\emph{локально отделён от нуля};
то есть для любого $p\in \Sigma$ существует такое $\epsilon>0$, что если $\dist{p}{q}\Sigma<\epsilon$ для некоторого $q\in \Sigma$, то $\inj(q)\ge\epsilon$.
\end{thm}

На самом деле верно, что \textit{функция $\inj\:\Sigma\z\to (0,\infty]$ непрерывна} \cite[5.4]{gromoll-klingenberg-meyer}.
Предложение докажется применением \ref{obs:d(exp)=1} и теоремы об обратной функции (\ref{thm:inverse}).

\parbf{Доказательство.}
Пусть $z=f(x,y)$ --- локальное представление $\Sigma$ в касательно-нормальных координатах при~$p$.
В частности, $\T_p$ --- горизонтальная плоскость.

Пусть $h$ --- композиция $\exp_p$ с проекцией $(x,y,z)\z\mapsto (x,y)$.
Согласно \ref{obs:d(exp)=1}, дифференциал $d_0h$ является тождественным;
иными словами, у $h$ единичная матрица Якоби в нуле.
Применив теорему об обратной функции (\ref{thm:inverse}), получим первую часть предложения.

Доказательство второй части аналогично, но более техническое.

Обозначим через $h_q$ композицию $\exp_q$ с ортогональной проекцией $(x,y,z)\mapsto (x,y)$.
Рассмотрим карту $s\:(u,v)\z\mapsto (u,v,f(u,v))$.
Положим 
\[m\:(u,v,a,b)\mapsto h_q(\vec v),\]
где $q=s(u,v)$ и $\vec v=a\cdot s_u+b\cdot s_v$.
По \ref{prop:geod-existence}, $m$ --- гладкое отображение, определённое в окрестности нуля.
Перейдя к меньшей окрестности, можно считать, что первые и вторые частные производные $m$ ограничены.
Из сказанного выше, у отображения $(a,b)\z\mapsto m(0,0,a,b)$ единичная матрица Якоби в нуле.
Значит, при малых $u$ и $v$ матрица Якоби отображения $(a,b)\z\mapsto m(u,v,a,b)$ в нуле близка единичной.
В частности, к отображению $(a,b)\z\mapsto m(u,v,a,b)$ при малых фиксированных $u$ и $v$
применима вторая часть теоремы об обратной функции (\ref{thm:inverse}), что и завершает доказательство.
\qeds

Идея доказательства следующего предложения приводится в \ref{ex:inj-rad}.

\begin{thm}{Предложение}\label{prop:inj-rad}
Пусть $p$ --- точка гладкой поверхности $\Sigma$ (без края).
Если $\exp_p$ инъективно в $B_p=B(0,r)_{\T_p}$, то сужение $\exp_p|_{B_p}$ есть диффеоморфизм из $B_p$ на свой образ в~$\Sigma$.
\end{thm}

Иными словами, $\inj(p)$ можно определить как точную верхнюю грань таких $r$, что сужение $\exp_p|_{B(0,r)_{\T_p}}$ инъективно.

\section{Кратчайшие и геодезические}

\begin{thm}{Предложение}\label{prop:gamma''}
Любая кратчайшая $\gamma$, параметризованная пропорционально длине дуги, на гладкой поверхности $\Sigma$, является геодезической на~$\Sigma$.
В частности, $\gamma$ --- гладкая кривая.

Обратное верно локально; а именно, для любой точки на $\Sigma$ найдётся окрестность $U$, такая что любая геодезическая, целиком лежащая в $U$, является кратчайшей.
\end{thm}

В частности достаточно короткий отрезок любой геодезической является кратчайшей.
Если геодезическая является кратчайшей, то её называют \index{минимизирующая геодезическая}\emph{минимизирующей}.
Сейчас мы увидим, что не все геодезические минимизирующие.

\begin{thm}{Упражнение}\label{ex:helix=geodesic}
Пусть $\Sigma$ --- цилиндрическая поверхность, заданная уравнением $x^2\z+y^2=1$.
Покажите, что винтовая линия $\gamma\:[0,2\cdot\pi]\to \Sigma$, определяемая как $\gamma(t)\z\df(\cos t, \sin t, t)$,
является геодезической, но не кратчайшей на~$\Sigma$.
\end{thm}

Полное доказательство предложения дано в разделе~\ref{sec:proof-of-gamma''}.
Следующее интуитивное объяснение может показаться достаточным.
В предположении гладкости $\gamma$, оно переделывается в строгое доказательство.

\parbf{Физическое объяснение.}
Будем думать, что вдоль кратчайшей $\gamma$ натянута резинка, удерживаемая на поверхности реакцией опоры $\vec n$.
Допустим, что  трение отсутствует, а значит, удельная сила $\vec n=\vec n(t)$ пропорциональна нормальному вектору к поверхности в точке~$\gamma(t)$.

Пусть $\tau$ --- натяжение резинки;
оно должно быть одинаковым во всех точках, иначе бы резинка скользила взад-вперёд.

Можно считать, что $\gamma$ параметризована длиной;
тогда равнодействующая сил натяжения дуги $\gamma_{[t_0,t_1]}$ равна $\tau\cdot(\gamma'(t_1)-\gamma'(t_0))$.
Следовательно, её удельная сила при $t_0$ равна
\begin{align*}
\vec f(t_0)&=\lim_{t_1\to t_0}\tau\cdot\frac{\gamma'(t_1)-\gamma'(t_0)}{t_1-t_0}=
\\
&=\tau\cdot\gamma''(t_0).
\end{align*}
По второму закону Ньютона,  
$\vec f+\vec n=0$,
а значит, $\gamma''(t)\perp\T_{\gamma(t)}\Sigma$.
\qeds

\begin{thm}{Следствие}
Пусть $\Sigma$ --- гладкая поверхность, $p\in\Sigma$ и $r\z\le \inj(p)$.
Тогда экспоненциальное отображение $\exp_p$ определяет диффеоморфизм $B(0,r)_{\T_p}\to B(p,r)_\Sigma$.
\end{thm}

\parbf{Доказательство.}
По \ref{prop:inj-rad}, сужение $\exp_p$ на $B_p={B(0,r)_{\T_p}}$ является диффеоморфизмом на его образ $\exp_p(B_p)\subset \Sigma$.

Очевидно, что $B(p,r)_\Sigma\supset\exp_p(B_p)$.
По \ref{prop:gamma''}, $B(p,r)_\Sigma\subset\exp_p(B_p)$, отсюда результат.
\qeds

{\sloppy

Согласно следствию, сужение $\exp_p|_{B(0,r)_{\T_p}}$ допускает обратное отображение, называемое \index{логарифм}\emph{логарифмом};
оно обозначается как \[\log_p\:B(p,r)_\Sigma\to B(0,r)_{\T_p}.\]

}

По предложению выше, любая кратчайшая, параметризованная длиной гладкая.
Это поможет решить следующую пару упражнений.

\begin{thm}{Упражнение}\label{ex:two-min-geod}
{\sloppy
Покажите, что если у двух кратчайших есть две различные общие точки $p$ и $q$, то либо это концы их обоих, либо же у кратчайших есть общая дуга от $p$ до~$q$.

}

Постройте пример, геодезических без общих дуг которые пересекаются в произвольно большом числе точек.
\end{thm}

\begin{thm}{Упражнение}\label{ex:min-geod+plane}
{\sloppy
Предположим, что гладкая поверхность $\Sigma$ зеркально-симметрична относительно плоскости $\Pi$.
Покажите, что кратчайшая на $\Sigma$ не может \emph{проходить сквозь} $\Pi$ больше раза.

}

Другими словами, если идти вдоль кратчайшей, то стороны $\Pi$ сменятся не больше одного раза. 
\end{thm}


{

\begin{thm}{Продвинутое упражнение}\label{ex:milka}\\
Пусть $\gamma\:[0,\ell]\z\to \Sigma$ --- минимизирующая геодезическая с единичной скоростью на 
гладкой замкнутой строго выпуклой поверхности $\Sigma$.

Положим $p\z=\gamma(0)$, $q=\gamma(\ell)$ и 
\[p^s=\gamma(s)-s\cdot\gamma'(s).\]

\begin{wrapfigure}{r}{40 mm}
\vskip-4mm
\centering
\includegraphics{mppics/pic-250}
\vskip-0mm
\end{wrapfigure}

Покажите, что $q$ не видна из $p^s$ при любом $s\in (0,\ell)$;
то есть отрезок $[p^s,q]$ пересекает $\Sigma$ в точке, отличной от~$q$.

Покажите, что утверждение перестаёт быть верным без предположения, что $\gamma$ минимизирующая.
\end{thm}

}

\begin{wrapfigure}[3]{r}{40 mm}
\end{wrapfigure}

\begin{thm}{Упражнение}\label{ex:round-sphere}
Пусть $\Sigma$ --- гладкая замкнутая поверхность.
Предположим, что для любых $p,q\z\in \Sigma$ расстояние $\dist{p}{q}\Sigma$ зависит только от расстояния $\dist{p}{q}{\mathbb{R}^3}$.
Покажите, что $\Sigma$ есть сфера.
\end{thm}

\begin{thm}{Сильно продвинутое упражнение}\label{ex:rad=2}
Пусть \(\Theta\) — сфера радиуса $2$ с центром в $0\in\mathbb{R}^3$,
и пусть \( \Sigma \) — гладкая замкнутая поверхность, содержащаяся в открытом шаре, ограниченном \(\Theta\).
Предположим, что все нормальные кривизны \( \Sigma \) не превышают~$1$ по абсолютной величине.

\begin{subthm}{ex:rad=2:a}
Докажите, что существует диффеоморфизм \(\rho\:\Theta \to \Sigma \) (назовем его \emph{радиальной проекцией}),
который отображает точку \( p \in \Theta \) в единственную точку пересечения \( \Sigma \cap [0,p]_{\mathbb{R}^3} \).
В частности, \( \Sigma \) ограничивает звёздную область.
\end{subthm}

\begin{subthm}{ex:rad=2:b}
Докажите, что \(\rho\: \Theta \to \Sigma \) не увеличивает длины кривых.
\end{subthm}

\begin{subthm}{ex:rad=2:c}
Пусть \( x\in \Sigma \) — точка на максимальном расстоянии от начала координат.
Докажите, что шар с диаметром \( [0, x] \) лежит в области, ограниченной \( \Sigma \).
Выведите отсюда, что эта область содержит единичный шар.
\end{subthm}

\end{thm}


\section{Лемма Либермана}

Вариант следующей леммы использовался Иосифом Либерманом \cite{liberman}.

\begin{thm}{Лемма}
\label{lem:liberman}
\index{лемма Либермана}
Пусть $f$ --- гладкая локально выпуклая функция, определённая на открытом подмножестве плоскости,
и $t\mapsto \gamma(t)\z=(x(t),y(t),z(t))$ --- геодезическая параметризованная длиной на графике $z=f(x,y)$.
Тогда $t\mapsto z(t)$ --- выпуклая функция; то есть $z''(t)\ge 0$ для любого~$t$.
\end{thm}

\parbf{Доказательство.}
Выберем ориентацию графика так, чтобы нормаль $\Norm$ всегда указывала вверх;
то есть в каждой точке у $\Norm$ положительная координата $z$.
Будем использовать сокращение $\Norm(t)$ для $\Norm(\gamma(t))$.

Поскольку $\gamma$ геодезическая, $\gamma''(t)\perp\T_{\gamma(t)}$;
иначе говоря, ускорение $\gamma''(t)$ пропорционально $\Norm(t)$ при любом~$t$.
Более того,
\[\gamma''=k\cdot\Norm,\]
где $k=k(t)$ --- нормальная кривизна графика в точке $\gamma(t)$ и направлении $\gamma'(t)$.

Следовательно,
\[z''=k\cdot\cos\theta,
\eqlbl{eq:z''}\]
где $\theta=\theta(t)$ --- угол между $\Norm(t)$ и осью $z$.


Поскольку $\Norm$ смотрит вверх, $\theta(t)<\tfrac\pi2$, и, значит, $\cos\theta>0$.

Так как $f$ выпукла, касательная плоскость подпирает график снизу в любой точке;
в частности, $k(t)\ge 0$ для любого~$t$.
Значит правая часть в \ref{eq:z''} неотрицательна.
Лемма доказана.
\qeds


\begin{thm}{Упражнение}\label{ex:closed-liberman}
Пусть $\Sigma$ --- это график локально выпуклой функции, определённой на открытом подмножестве плоскости.
Покажите, что $\Sigma$ не содержит замкнутых геодезических.
\end{thm}

\begin{thm}{Упражнение}\label{ex:rho''}
Пусть $\gamma$ --- геодезическая с единичной скоростью на гладкой выпуклой поверхности $\Sigma$, и точка $p$ лежит внутри выпуклого множества, ограниченного~$\Sigma$.
Рассмотрим функцию $\rho(t)=|p-\gamma(t)|^2$.
Покажите, что $\rho''(t)\le 2$ для любого~$t$.
\end{thm}

\section{Полная кривизна геодезической}

Напомним, что $\tc\gamma$ обозначает полную кривизну кривой~$\gamma$, см. \ref{sec:Total curvature}.

\begin{thm}{Упражнение}\label{ex:tc-spherical-image}
Пусть $\gamma$ --- геодезическая на гладкой поверхности $\Sigma$ 
с полем нормалей $\Norm$.
Покажите, что $\length(\Norm\circ\gamma)\ge \tc\gamma$.
\end{thm}

\begin{thm}{Теорема}\label{thm:usov}
Пусть $\Sigma$ --- график выпуклой $\ell$-липшицевой функции $f$, определённой на открытом множестве в плоскости $(x,y)$.
Тогда полная кривизна любой геодезической на $\Sigma$ не превышает $2\cdot \ell$.
\end{thm}

Теорема доказана Владимиром Усовым \cite{usov}.

\parbf{Доказательство.}
Пусть $t\mapsto\gamma(t)=(x(t),y(t),z(t))$ --- геодезическая с единичной скоростью на~$\Sigma$.
По лемме Либермана (\ref{lem:liberman}), функция $t\mapsto z(t)$ выпукла.

Поскольку угловой коэффициент $f$ не превышает $\ell$, 
$|z'(t)|\le \frac{\ell}{\sqrt{1+\ell^2}}$
при любом $t$.
Можно считать, что $\gamma$ определена на интервале $[a,b]$.
Тогда
\[
\begin{aligned}
\int_a^b z''(t) dt&=z'(b)-z'(a)\le 
 2\cdot \frac{\ell}{\sqrt{1+\ell^2}}.
\end{aligned}
\eqlbl{eq:intz''}
\]

Также отметим, что $z''$ является проекцией $\gamma''$ на ось $z$.
При этом угловой коэффициент касательной плоскости $\T_{\gamma (t)} \Sigma$ не превышает $\ell$, для любого~$t$, и
\[|\gamma'' (t)| \le z''(t)\cdot\sqrt{1+ \ell ^2},\]
ибо $\gamma ''$ перпендикулярна этой плоскости.

Из \ref{eq:intz''}, получаем, что
\begin{align*}
\tc\gamma&=\int_a^b|\gamma'' (t)|\cdot dt\le 
\sqrt{1+ \ell ^2}\cdot \int_a^b z''(t)\cdot dt\le 
2\cdot \ell.
\end{align*}
\qedsf

По следующему упражнению, оценка в теореме оптимальна.

\begin{thm}{Упражнение}\label{ex:usov-exact}
Пусть $\Sigma$ --- график $z=\ell\cdot\sqrt{x^2+y^2}$ с удалённым началом координат.
Покажите, что любая бесконечная в обе стороны геодезическая $\gamma$ на $\Sigma$ имеет полную кривизну ровно $2\cdot \ell$.
\end{thm}

\begin{thm}{Упражнение}\label{ex:ruf-bound-mountain}
Предположим, что $f$ является гладкой выпуклой $\tfrac32$-липшицевой функцией, определённой на плоскости $(x,y)$.
Покажите, что любая геодезическая $\gamma$ на графике $z\z=f(x,y)$ является простой; то есть не имеет самопересечений.

{\sloppy

Постройте выпуклую $2$-липшицеву функцию, определённую на плоскости,
с самопересекающейся геодезической $\gamma$ на её графике.

}

\end{thm}



\begin{thm}{Теорема}\label{thm:tc-of-mingeod}
Пусть гладкая поверхность $\Sigma$ ограничивает выпуклое множество $K$, и при этом $B(0,\epsilon)\subset K\subset B(0,1)$.
Тогда полную кривизну любой кратчайшей на $\Sigma$ можно оценить, зная~$\epsilon$.
\end{thm}

{

\begin{wrapfigure}{r}{48 mm}
\vskip4mm
\centering
\includegraphics{mppics/pic-83}
\vskip-0mm
\end{wrapfigure}

\begin{thm}{Доказательство и упражнение}\label{ex:bound-tc}
Пусть $\Sigma$ как в теореме, и $\gamma$ --- кратчайшая на~$\Sigma$ с единичной скоростью.
Обозначим $\Norm(t)$ вектор нормали к $\Sigma$ в точке $\gamma(t)$, направленный наружу;
пусть $\theta(t)$ --- угол между $\Norm(t)$ и направлением от начала координат до $\gamma(t)$,
и 
$k(t)$ --- кривизна $\gamma$ при~$t$.
Рассмотрим функцию $\rho(t)\z=|\gamma(t)|^2$.


\begin{subthm}{ex:bound-tc:a}
Покажите, что $\cos(\theta(t))\ge \epsilon$ для любого~$t$.
\end{subthm}

\begin{subthm}{ex:bound-tc:b}
Покажите, что $|\rho'(t)|\le 2$ для любого~$t$.
\end{subthm}

\begin{subthm}{ex:bound-tc:c}
Покажите, что
\[\rho''(t)=2-2\cdot k(t)\cdot \cos \theta(t)\cdot |\gamma(t)|\]
для любого~$t$.
\end{subthm}

\begin{subthm}{ex:bound-tc:d}
Воспользуйтесь короткой проекцией из единичной сферы на $\Sigma$, дабы показать, что
\[\length \gamma\le \pi.\]
\end{subthm}

\begin{subthm}{ex:bound-tc:e}
Выведите отсюда, что $\tc\gamma\le 100/\epsilon^2$.
\end{subthm}

\end{thm}

}

\parit{Замечание.}
Полученная оценка идёт к бесконечности при $\epsilon\to 0$,
но известна и оценка, не зависящая от $\epsilon$;
это результат Нины Лебедевой и первого автора \cite{lebedeva-petrunin}.
Алексей Погорелов выдвинул гипотезу, что существует оценка на длину сферического образа кратчайшей \cite{pogorelov}.
Согласно \ref{ex:tc-spherical-image}, эта гипотеза сильнее,
однако к ней и всевозможным её вариантам  нашлись контрпримеры \cite{zalgaller,milka,usov,pach}.
