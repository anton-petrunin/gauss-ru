\chapter{Формула Гаусса --- Бонне}
\label{chap:gauss-bonnet}
\section{Формулировка}

Следующая теорема доказана Карлом Фридрихом Гауссом \cite{gauss}
для геодезических треугольников;
Пьер Бонне и Жак Бине независимо 
обобщили её на произвольные кривые.
\index{формула Гаусса --- Бонне}

\begin{thm}{Теорема}\label{thm:gb}
Пусть $\Delta$ --- топологический диск на гладкой ориентированной поверхности $\Sigma$, 
а его граница $\partial\Delta$ --- кусочно-гладкая кривая так ориентированная, что $\Delta$ лежит слева от неё.
Тогда 
\[\tgc{\partial\Delta}+\iint_\Delta K=2\cdot \pi,\eqlbl{eq:g-b}\]
где $K$ --- гауссова кривизна.
\end{thm}

В этой главе доказывается частный случай этой формулы.
Это доказательство можно обобщить, но полное доказательство, будет строится на другой идее, см.~\ref{sec:gauss--bonnet:formal}.
Прежде чем думать о доказательстве, давайте попрактикуемся в приложениях формулы.

\begin{thm}{Упражнение}\label{ex:1=geodesic-curvature}
Пусть $\gamma$ --- простая замкнутая кривая постоянной геодезической кривизны $1$ на гладкой замкнутой поверхности $\Sigma$ с положительной гауссовой кривизной.
Покажите, что $\length\gamma< 2\cdot\pi$;
то есть, что $\gamma$ короче единичной окружности.  
\end{thm}

\begin{thm}{Упражнение}\label{ex:GB-hat}
Предположим, что диск $\Delta$ лежит на графике $z=f(x,y)$ гладкой функции,
а его граница $\partial\Delta$ лежит в плоскости $(x,y)$, 
и $\Delta$ подходит к плоскости под фиксированным углом $\alpha$.
Покажите, что $\iint_\Delta K=2\cdot\pi\cdot(1-\cos\alpha)$.

\end{thm}

\begin{thm}{Упражнение}\label{ex:geodesic-half}
Пусть $\gamma$ --- простая замкнутая геодезическая на гладкой замкнутой поверхности $\Sigma$ с положительной гауссовой кривизной и сферическим отображением $\Norm\:\Sigma\to\mathbb{S}^2$.
Покажите, что кривая $\alpha=\Norm\circ\gamma$ делит сферу на две области равной площади.

Выведите отсюда, что $\length \alpha\ge 2\cdot\pi.$
\end{thm}

\begin{thm}{Упражнение}\label{ex:closed-geodesic}
Пусть $\gamma$ --- замкнутая геодезическая возможно с самопересечениями на гладкой замкнутой поверхности $\Sigma$ с положительной гауссовой кривизной.
Предположим, что $R$ --- одна из областей, которые $\gamma$ вырезает из~$\Sigma$.
Покажите, что $\iint_R K\le 2\cdot\pi$.

Выведите отсюда, что любые две замкнутые геодезические на $\Sigma$ имеют общую точку.
\end{thm}

\begin{thm}{Упражнение}\label{ex:self-intersections}
Пусть $\Sigma$ --- замкнутая гладкая поверхность с положительной гауссовой кривизной. 
Покажите, что замкнутая геодезическая на $\Sigma$ 
ни в какой из карт не может выглядеть как одна из кривых на следующих рисунках.

\begin{figure}[h]
\begin{minipage}{.32\textwidth}
\centering
\includegraphics{mppics/pic-46}
\end{minipage}
\hfill
\begin{minipage}{.32\textwidth}
\centering
\includegraphics{mppics/pic-47}
\end{minipage}
\hfill
\begin{minipage}{.32\textwidth}
\centering
\includegraphics{mppics/pic-471}
\end{minipage}

\medskip

\begin{minipage}{.32\textwidth}
\centering
\caption*{\textit{(лёгкая)}}
\end{minipage}
\hfill
\begin{minipage}{.32\textwidth}
\centering
\caption*{\textit{(сложная)}}
\end{minipage}
\hfill
\begin{minipage}{.32\textwidth}
\centering
\caption*{\textit{(безнадёжная)}}
\end{minipage}
\vskip-5mm
\end{figure}

\end{thm}



Следующее упражнение уточняет оценку в \ref{ex:ruf-bound-mountain}.

\begin{thm}{Упражнение}\label{ex:sqrt(3)}
Предположим, что $f\:\mathbb{R}^2\to\mathbb{R}$ --- гладкая выпуклая $\sqrt{3}$-липшицева функция.
Покажите, что любая геодезическая на графике $z=f(x,y)$ не имеет самопересечений.
\end{thm}

Поверхность $\Sigma$ называется \index{односвязная поверхность}\emph{односвязной}, если любая простая замкнутая кривая на $\Sigma$ ограничивает диск.
Эквивалентно, любую замкнутую кривую на $\Sigma$ можно продеформировать в \index{тривиальная кривая}\emph{тривиальную кривую}; то есть кривую, стоящую всё время в одной точке.

Плоскости и сферы дают примеры односвязных поверхностей;
торы и цилиндры неодносвязны.

\begin{thm}{Упражнение}\label{ex:unique-geod}
Предположим, что $\Sigma$ --- открытая односвязная поверхность с неположительной гауссовой кривизной.

\begin{subthm}{ex:unique-geod:unique}
Покажите, что любые две точки на $\Sigma$ соединены единственной геодезической.
В частности, геодезические на $\Sigma$ не имеют самопересечений. 
\end{subthm}

\begin{subthm}{ex:unique-geod:diffeomorphism}
Выведите, что для любой точки $p\in \Sigma$,
экспоненциальное отображение $\exp_p$ есть диффеоморфизм из касательной плоскости $\T_p$ в~$\Sigma$.
В частности, поверхность $\Sigma$ диффеоморфна плоскости.
\end{subthm}

\end{thm}


\section{Аддитивность}

Пусть $\Delta$ --- топологический диск на гладкой ориентированной поверхности $\Sigma$, ограниченный простой кусочно-гладкой кривой $\partial \Delta$.
Как и раньше, мы считаем, что $\partial \Delta$ ориентирована и $\Delta$ лежит от неё слева.
Введём обозначение \index{10gb@$\GB$ (формула Гаусса --- Бонне)}
\[\GB(\Delta)
\df
\tgc{\partial\Delta}+\iint_\Delta K-2\cdot \pi,
\eqlbl{eq:GB}\]
где $K$ --- гауссова кривизна.
Обозначение $\GB$ призвано об этом напоминать, что формулу Гаусса --- Бонне можно записать как
\[\GB(\Delta)=0.\]

{

\begin{wrapfigure}{r}{40 mm}
\vskip-14.5mm
\centering
\includegraphics{mppics/pic-1750}
\vskip-0mm
\end{wrapfigure}

\begin{thm}{Лемма}\label{lem:GB-sum}
Пусть диск $\Delta$ разрезан на два диска $\Delta_1$ и $\Delta_2$ кривой $\delta$.
Тогда
\[
\GB(\Delta)=\GB(\Delta_1)+\GB(\Delta_2).
\]
\end{thm}

}

\parbf{Доказательство.}
Разделим $\partial \Delta$ на две кривые $\gamma_1$ и $\gamma_2$, которые имеют общие концы с $\delta$, так что
$\Delta_i$ ограничен дугой $\gamma_i$ и~$\delta$ для $i=1,2$.


Обозначим через $\phi_1$, $\phi_2$, $\psi_1$ и $\psi_2$ углы между $\delta$ и $\gamma_i$ как на рисунке.
Полагая, что дуги $\gamma_1$, $\gamma_2$ и $\delta$ ориентированы как на рисунке, получаем
\begin{align*}
\tgc{\partial \Delta}&= \tgc{\gamma_1}-\tgc{\gamma_2}+(\pi-\phi_1-\phi_2)+(\pi-\psi_1-\psi_2),
\\
\tgc{\partial \Delta_1}&= \tgc{\gamma_1}-\tgc{\delta}+(\pi-\phi_1)+(\pi-\psi_1),
\\
\tgc{\partial \Delta_2}&= \tgc{\delta}-\tgc{\gamma_2}+(\pi-\phi_2)+(\pi-\psi_2),
\\
\iint_\Delta K&=\iint_{\Delta_1} K+\iint_{\Delta_2} K.
\end{align*}
Остаётся подставить результаты в формулы для $\GB(\Delta)$, $\GB(\Delta_1)$ и $\GB(\Delta_2)$.
\qeds

\section{Сферический случай}

Если наша поверхность $\Sigma$ является плоскостью, то её гауссова кривизна равна нулю, и формула Гаусса --- Бонне \ref{eq:g-b} переписывается как 
\[\tgc{\partial\Delta}=2\cdot \pi,\]
а значит, она следует из \ref{prop:total-signed-curvature}.
Другими словами, $\GB(\Delta)=0$ для любого плоского диска~$\Delta$ с кусочно гладкой границей.

Если $\Sigma$ --- это единичная сфера, то $K\equiv1$;
в этом случае \ref{thm:gb} эквивалента следующему.

\begin{thm}{Предложение}\label{prop:area-of-spher-polygon}
{\sloppy
Пусть $P$ --- сферический многоугольник, ограниченный простой замкнутой ломаной геодезической $\partial P$.
Предположим, что $\partial P$ ориентирована так, что $P$ находится слева от $\partial P$.
Тогда 
\[\GB(P)=\tgc{\partial P}+\area P-2\cdot \pi=0.\]

}

Более того, та же формула справедлива для любой сферической области, ограниченной кусочно-гладкой простой замкнутой кривой.
\end{thm}

Это предложение будет использовано в следующем разделе.


\parbf{Набросок доказательства.}
Пусть $\Delta$ --- сферический треугольник с углами 
$\alpha$, $\beta$ и~$\gamma$.
Согласно \ref{lem:area-spher-triangle},
\[\area\Delta=\alpha+\beta+\gamma-\pi.\]

Поскольку граница $\partial\Delta$ ориентирована так, что $\Delta$ лежит от неё слева, 
её ориентированные внешние углы равны $\pi-\alpha$, $\pi-\beta$ и $\pi-\gamma$.
Следовательно,
\[\tgc{\partial\Delta}=3\cdot\pi-\alpha-\beta-\gamma.\]
Отсюда $\tgc{\partial\Delta}+\area \Delta=2\cdot\pi$, что эквивалентно, $\GB(\Delta)=0$.

Далее, любой сферический многоугольник $P$ можно разбить на треугольники, за несколько шагов, разрезая многоугольник на два по ломаной геодезической на каждом шагу.
По аддитивности (\ref{lem:GB-sum}), получаем, что 
\[\GB(P)=0\]
для любого сферического многоугольника~$P$.

Второе утверждение доказывается через приближения.
Нужно показать, что полная геодезическая кривизна кусочно-гладкой простой кривой
приближается полной геодезической кривизной вписанных в неё ломаных.
Рассуждение схоже с решением \ref{ex:total-curvature=}; мы опускаем подробности.
\qeds

\begin{thm}{Упражнение}\label{ex:half-sphere-total-curvature}
Пусть $\gamma$ --- простая кусочно-гладкая петля на единичной сфере $\mathbb{S}^2$.
Предположим, что $\gamma$ делит $\mathbb{S}^2$ на две области равной площади.
Обозначим через $p$ базовую точку~$\gamma$.
Покажите, что параллельный перенос $\iota_\gamma\:\T_p\mathbb{S}^2\to\T_p\mathbb{S}^2$ является тождественным отображением.
\end{thm}

\section{Наглядное полудоказательство}\label{sec:gb-intuitive-proof}

Следующий частный случай формулы Гаусса --- Бонне является ключевым.
Общий случай можно доказать аналогично, используя ориентированную площадь с учётом кратности, но мы пойдём другим путём, см.~\ref{sec:gauss--bonnet:formal}. 

\parbf{Доказательство \ref{thm:gb} для открытых и замкнутых поверхностей с положительной гауссовой кривизной.}
Пусть $\Norm\:\Sigma\to\mathbb{S}^2$ --- сферическое отображение.
Из \ref{cor:intK},
\[\GB(\Delta)=\tgc{\partial\Delta}+\area[\Norm(\Delta)]-2\cdot \pi.
\eqlbl{eq:gb-area}\]

Выберем петлю $\alpha$, которая проходит вдоль $\partial\Delta$ так, чтобы $\Delta$ лежала слева от неё; пусть $p\in \partial\Delta$ --- её базовая точка.
Рассмотрим параллельный перенос $\iota_\alpha\:\T_p\to\T_p$ вдоль $\alpha$.
Согласно \ref{prop:pt+tgc}, $\iota_\alpha$ --- поворот по часовой стрелке на угол $\tgc{\alpha}_\Sigma$.

Пусть $\beta=\Norm\circ\alpha$.
Согласно \ref{obs:parallel=}, $\iota_\alpha=\iota_\beta$, где $\beta$ рассматривается как кривая на единичной сфере.

Далее, $\iota_\beta$ --- поворот по часовой стрелке на угол $\tgc{\beta}_{\mathbb{S}^2}$.
Согласно \ref{prop:area-of-spher-polygon},
\[\GB(\Norm(\Delta))=\tgc{\beta}_{\mathbb{S}^2}+\area[\Norm(\Delta)]-2\cdot \pi=0.\]
Значит, 
$\iota_\beta$ --- поворот против часовой стрелки на угол $\area[\Norm(\Delta)]$.


То есть, поворот по часовой стрелке на угол $\tgc{\alpha}_\Sigma$ идентичен повороту против часовой стрелки на угол $\area[\Norm(\Delta)]$.
Повороты идентичны, если их углы равны по модулю $2\cdot\pi$.
Следовательно, 
\[
\begin{aligned}
\GB(\Delta)&=\tgc{\partial\Delta}_\Sigma+\area[\Norm(\Delta)]-2\cdot \pi=
2\cdot n \cdot \pi
\end{aligned}
\eqlbl{eq:sum=2pin}\]
для некоторого целого числа~$n$.

Остаётся показать, что $n=0$.
Согласно \ref{prop:total-signed-curvature}, это верно для топологического диска на плоскости. 
В общем случае, диск $\Delta$ можно рассматривать как результат непрерывной деформации плоского диска. 
Целое число $n$ не может измениться в процессе деформации, так как левая часть в \ref{eq:sum=2pin} меняется непрерывно,
и раз $n=0$ в начале, то $n=0$ и в конце деформации.
\qeds

\section{Простая геодезическая}

Следующая теорема даёт интересное приложение формулы Гаусса --- Бонне; она доказана Стефаном Кон-Фоссеном \cite[Satz 9]{convossen}.

\begin{thm}{Теорема}\label{thm:cohn-vossen}
Любая открытая гладкая поверхность с положительной гауссовой кривизной имеет простую двустороннe бесконечную геодезическую.
\end{thm}

\parbf{Доказательство.}
Пусть $\Sigma$ --- открытая поверхность с положительной гауссовой кривизной.
Выберем двустороннe бесконечную геодезическую $\gamma$ на~$\Sigma$.

Если у кривой $\gamma$ есть самопересечения, то она содержит простую петлю;
то есть для некоторого интервала $[a,b]$,
сужение $\ell=\gamma|_{[a,b]}$ представляет собой простую петлю.

Согласно \ref{ex:convex-proper-plane}, $\Sigma$ параметризуется открытой выпуклой областью $\Omega$ на плоскости.
По теореме Жордана (\ref{thm:jordan}), $\ell$ ограничивает топологический диск на $\Sigma$; обозначим его через~$\Delta$.
Если $\phi$ --- внутренний угол в базовой точке петли, то по формуле Гаусса --- Бонне
\[\iint_\Delta K=\pi+\phi.\] 

Напомним, что 
\[\iint_\Sigma K\le 2\cdot\pi,
\eqlbl{intK=<2pi+}\]
см. \ref{ex:intK:2pi}.
Следовательно, $0<\phi<\pi$; то есть $\gamma$ не имеет вогнутых простых петель.

Допустим, что у $\gamma$ есть две простые петли, $\ell_1$ и $\ell_2$;
они ограничивают диски скажем $\Delta_1$ и $\Delta_2$.
Тогда диски $\Delta_1$ и $\Delta_2$ должны пересекаться;
иначе кривизна $\Sigma$ превысила бы $2\cdot\pi$, что невозможно по \ref{intK=<2pi+}.

Значит, после выхода из $\Delta_1$ геодезической $\gamma$ придётся войти в него снова, прежде чем образовать новую петлю.
\begin{figure}[h!]
\vskip-0mm
\centering
\includegraphics{mppics/pic-1550}
\end{figure}
Рассмотрим момент, когда $\gamma$ снова входит в $\Delta_1$;
на рисунке показаны два возможных сценария.
На левом рисунке мы получаем два непересекающихся диска, что, как мы знаем, невозможно.
Правый рисунок также невозможен --- в этом случае мы получаем вогнутую простую петлю.

Следовательно, $\gamma$ содержит только одну простую петлю.
Эта петля вырезает из $\Sigma$ диск и обходит его справа или слева.
Так, все самопересекающиеся геодезические 
делятся на два типа: {}\emph{правые} и {}\emph{левые}.

Если геодезическая $t\mapsto \gamma(t)$ правая, то обращение параметризации $t\mapsto \gamma(-t)$ делает её левой.
Выпустим геодезическую в каждом направлении из точки $p=\gamma(0)$.
Это даёт однопараметрическое семейство геодезических $\gamma_s$ для $s\in[0,\pi]$, соединяющее геодезическую $t\mapsto \gamma(t)$ с $t\mapsto \gamma(-t)$; то есть $\gamma_0(t)\z=\gamma(t)$, а $\gamma_\pi(t)=\gamma(-t)$.

Геодезическая $\gamma_s$ является правой (или левой) при $s$ из открытого множества в $[0,\pi]$.
То есть, если $\gamma_s$ правая, то правые и все $\gamma_t$ при $t$, близких к~$s$.%
\footnote{Неформално говоря, это означает, что самопересечение не может исчезнуть вдруг. Попреобуйе в этом убедиться.}


Поскольку интервал $[0,\pi]$ связен, он не разбивается на два открытых множества.
Значит, при каком-то $s$ геодезическая $\gamma_s$ не является ни правой, ни левой;
то есть $\gamma_s$ не имеет самопересечений.
\qeds


{

\begin{wrapfigure}{r}{17 mm}
\vskip-0mm
\centering
\includegraphics{mppics/pic-1575}
\end{wrapfigure}

\begin{thm}{Упражнение}\label{ex:cohn-vossen}
Пусть $\Sigma$ --- открытая гладкая поверхность с положительной гауссовой кривизной,
и $\alpha\:[0,1]\z\to \Sigma$ --- такая гладкая петля, что $\alpha'(0)\z=-\alpha'(1)$.
Покажите, что найдётся простая двухсторонняя бесконечная геодезическая $\gamma$, касаетелная к $\alpha$.
\end{thm}

}


\section{Области общего вида}
\index{формула Гаусса --- Бонне}

Следующее обобщение формулы Гаусса --- Бонне получено Вальтером фон Диком \cite{dyck}.

\begin{thm}{Теорема}\label{thm:GB-generalized}
Пусть $\Lambda$ --- компактная область на гладкой поверхности.
Предположим, что $\Lambda$ ограничена конечным (возможно, пустым) набором замкнутых кусочно-гладких кривых $\gamma_1,\dots,\gamma_n$, и каждая $\gamma_i$ ориентирована так, что $\Lambda$ лежит слева от неё.
Тогда
\[\iint_\Lambda K=2\cdot \pi\cdot \chi-\tgc{\gamma_1}-\dots-\tgc{\gamma_n}\eqlbl{eq:g-b++}\]
для целого числа $\chi=\chi(\Lambda)$.

Более того, если граф с $v$ вершинами и $e$ рёбрами разбивает $\Lambda$ на $f$ дисков и содежит все $\gamma_i$, то $\chi=v-e+f$.
\end{thm}


Число $\chi=\chi(\Lambda)$ называется \index{эйлерова характеристика}\emph{эйлеровой характеристикой} области $\Lambda$. 
Она не зависит от выбора разбиения, ведь остальные члены в формуле \ref{eq:g-b++} от него не зависят.
Формула \ref{eq:g-b++} выводится из стандартной формулы Гаусса --- Бонне (\ref{thm:gb}).
Геометрия та же, что в \ref{lem:GB-sum}, но комбинаторика сложней.

\begin{wrapfigure}{o}{29 mm}
\vskip-8mm
\centering
\includegraphics{mppics/pic-1580}
\end{wrapfigure}

Прежде чем перейти к доказательству, попробуйте вывести формулу для разбиения кольца $A$ на рисунке следуя рассуждениям в \ref{lem:GB-sum}.
У этого графа 4 вершины и 6 рёбер (одно из них --- петля), и он разбивает $A$ на два диска $\Delta_1$ и $\Delta_2$.
Таким образом, $\chi(A)=4-6+2=0$, и
\[\iint_A K=-\tgc{\gamma_1}-\tgc{\gamma_2}.\]


\parbf{Доказательство.}
Пусть граф с $v$ вершинами и $e$ рёбрами разбивает область $\Lambda$ на $f$ дисков $\Delta_1,\dots,\Delta_f$.
Применим формулу Гаусса --- Бонне к каждому диску и сложим результаты:
\[
\begin{aligned}
\iint_\Lambda K&=\iint_{\Delta_1} K+\dots+\iint_{\Delta_f} K=
2\cdot f\cdot \pi-\tgc{\partial\Delta_1}-\dots-\tgc{\partial\Delta_f}.
\end{aligned}
\]
Остаётся показать, что  
\[\tgc{\gamma_1}+\dots+\tgc{\gamma_n}-\tgc{\partial\Delta_1}-\dots-\tgc{\partial\Delta_f}
=
2\cdot\pi\cdot(v-e).
\eqlbl{eq:GB-sum}\]
Для этого, мы вычислим левую часть, суммируя отдельно вклад каждого ребра и каждой вершины.

Пусть $\sigma$ --- ребро графа.
Если $\sigma$ не является частью какой-либо $\gamma_i$,
то оно встречается дважды на границе дисков, скажем, на $\partial \Delta_i$ и $\partial \Delta_j$. 
В этом случае можно считать, что $\Delta_i$ лежит слева от $\sigma$, а $\Delta_j$ --- справа, поэтому 
$\sigma$ добавляет $\tgc\sigma$ в $\tgc{\partial\Delta_i}$ и $-\tgc\sigma$ в $\tgc{\partial\Delta_j}$; следовательно, $\sigma$ ничего не добавляет в левую часть формулы \ref{eq:GB-sum}.
Может случиться, что $i=j$ (как для одного из рёбер на рисунке выше);
в этом случае у нас один и тот же диск с обеих сторон от~$\sigma$, но всё равно оно ничего не добавляет в \ref{eq:GB-sum}.
Если же ребро $\sigma$ идёт по границе, то оно встречается один раз на границе какого-то диска, скажем, на $\partial \Delta_j$.
Можно считать, что и $\Lambda$, и $\Delta_j$ лежат слева от $\sigma$,
поэтому оно вносит $\tgc\sigma$ как в $\tgc{\gamma_i}$, так и в $\tgc{\partial\Delta_j}$, а 
эти вклады компенсируют друг друга в левой части формулы \ref{eq:GB-sum}.


Подытожим сказанное.
Вклады в левую часть формулы \ref{eq:GB-sum}, которые приходят от полной геодезической кривизны рёбер, в точности компенсируют друг друга.

Теперь займёмся с внешними углами при вершинах.
Выберем вершину $p$, пусть $d$ --- её \index{степень вершины}\emph{степень},
то есть число рёбер исходящих из~$p$.

\begin{wrapfigure}{r}{23 mm}
\vskip-3mm
\centering
\includegraphics{mppics/pic-1585}
\end{wrapfigure}

Предположим, что $p$ лежит во внутренней части~$\Lambda$.
Пусть $\delta_1,\dots,\delta_d$ --- внутренние углы при $p$ в дисках, прилегающих к $p$, 
и $\phi_{i}=\pi-\delta_{i}$ --- соответствующие внешние углы.
Тогда вклад $p$ в сумму равен 
$-\phi_1-\dots-\phi_d$.
Поскольку $\delta_1+\dots+\delta_d=2\cdot\pi$, вклад $p$ в левую часть формулы \ref{eq:GB-sum} равен
\[-\phi_1-\dots-\phi_d = (\delta_1+\dots+\delta_d) - d\cdot \pi=(2-d)\cdot \pi.\]

\begin{wrapfigure}{r}{23 mm}
\vskip-0mm
\centering
\includegraphics{mppics/pic-1590}
\end{wrapfigure}

Если же $p$ лежит на границе $\Lambda$, то является вершиной для $d-1$ внутренних углов
$\delta_1,\dots,\delta_{d-1}$,
и $\phi_{i}\z=\pi-\delta_{i}$ --- соответствующие внешние углы.
Заметим, что
\[\delta_1+\dots+\delta_{d-1}\z=\pi-\theta,\]
где $\theta\in(-\pi,\pi)$ --- внешний угол области $\Lambda$ при $p$.
А значит, вклад $p$ в левую часть формулы \ref{eq:GB-sum} опять равен
\[\theta-\sum\phi_{i}=(2-d)\cdot \pi.\]


Ещё раз подытожим сказанное.
Если $p_1,\dots,p_v$ --- вершины графа, а $d_1,\dots,d_v$ --- их степени,
то общий вклад от внешних углов в левую часть формулы \ref{eq:GB-sum} равен
\[2\cdot v\cdot \pi-(d_1+\dots+d_v)\cdot\pi.
\eqlbl{eq:GB-sum-d}\]
Так как рёбра не внесли ничего, левая часть формулы \ref{eq:GB-sum} равна \ref{eq:GB-sum-d}.

Остаётся проверить, что $d_1+\dots+d_v=2\cdot e$.
И действительно, $d_1+\z\dots+d_v$ --- это число концов всех рёбер в графе, а их $2\cdot e$,
ведь у каждого ребра ровно два конца.
Отсюда получаем \ref{eq:GB-sum} и \ref{eq:g-b++}.
\qeds

\begin{thm}{Упражнение}\label{ex:g-b-chi}
{\sloppy
Найдите интеграл гауссовой кривизны по каждой из следующих поверхностей:

}

\setlength{\columnseprule}{0.4pt}
\begin{multicols}{2}

\begin{subthm}{ex:g-b-chi:torus}
Тор.
\end{subthm}

\begin{Figure}
\vskip-0mm
\centering
\includegraphics{mppics/pic-1595}
\end{Figure}

\begin{subthm}{ex:g-b-chi:moebius}
Лента Мёбиуса с геодезической границей.
\end{subthm}

\begin{Figure}
\vskip-0mm
\centering
\includegraphics{mppics/pic-1605}
\end{Figure}

\begin{subthm}{ex:g-b-chi:pair-of-pants}
Пара штанов с геодезическими компонентами границы.
\end{subthm}
\begin{Figure}
\vskip-0mm
\centering
\includegraphics{mppics/pic-1600}
\end{Figure}

\begin{subthm}{ex:g-b-chi:two-handles}
Сфера с двумя ручками.
\end{subthm}

\begin{Figure}
\vskip-0mm
\centering
\includegraphics{mppics/pic-1610}
\end{Figure}

\end{multicols}

\begin{subthm}{ex:g-b-chi:cylinder}
Цилиндр, окрестности краёв которого лежат в плоскостях.
\begin{Figure}
\vskip-0mm
\centering
\includegraphics{mppics/pic-1620}
\end{Figure}
\end{subthm}

\end{thm}
