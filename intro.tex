Художественное оформление Аны Кристины Чавес Калис.
\null
\vfill
\noindent{\includegraphics[scale=0.5]{pics/by-sa}
\vspace*{1mm}
\\
\hbox{\parbox{.8\textwidth}
{Это произведение распространяется на условиях лицензии CC BY-SA 4.0, с ней можно ознакомиться по ссылке
\texttt{https://creativecommons.org/licenses/by-sa/4.0}}}}


\thispagestyle{empty}
\newpage
\addtocontents{toc}{\cftpagenumbersoff{part}}
{
\clearpage
\phantomsection
\pdfbookmark[1]{\contentsname}{bm:toc}
\sloppy
\footnotesize
\tableofcontents

}
\vfill

\begin{figure}[h!]
\centering
\begin{tikzpicture}[->,>=stealth',shorten >=1pt,auto,scale=.22,
  thick,main node/.style={circle,draw,font=\sffamily\bfseries,minimum size=8mm}]

  \node[main node] (1) at (0,0) {\ref{chap:curves-def}};
  \node[main node] (2) at (3,-5) {\ref{chap:length}};
  \node[main node] (3) at (6,-10) {\ref{chap:curve-curvature}};
  \node[main node] (4) at (9,-5) {\ref{chap:poly}};
  \node[main node] (5) at (9,-15) {\ref{chap:torsion}};
  \node[main node] (6) at (12,-10) {\ref{chap:signed-curvature}};
  \node[main node] (7) at (15,-15) {\ref{chap:supporting-curves}};
  \node[main node] (8) at (18,0) {\ref{chap:surfaces-def}};
  \node[main node] (9) at (15,-5) {\ref{chap:first-order}};
  \node[main node] (10) at (18,-10) {\ref{chap:surface-curvature}};
  \node[main node] (11) at (21,-15) {\ref{chap:Curves in a surface}};
  \node[main node] (12) at (18,-20) {\ref{chap:surface-support}};
  \node[main node] (13) at (21,-5) {\ref{chap:shortest}};
  \node[main node] (14) at (24,-10) {\ref{chap:geodesics}};
  \node[main node] (15) at (27,-15) {\ref{chap:parallel-transport}};
  \node[main node] (16) at (33,-15) {\ref{chap:gauss-bonnet}};
  \node[main node] (17) at (30,-10) {\ref{chap:semigeodesic}};
  \node[main node] (18) at (36,-10) {\ref{chap:comparison}};

  \path[every node/.style={font=\sffamily\small}]
   (1) edge node{}(2)
   (2) edge node{}(3)
   (3) edge node{}(5)
   (3) edge node{}(4)
   (3) edge node{}(6)
   (4) edge[dashed] node{}(6)
   (6) edge node{}(7)
   (6) edge node{}(10)
   (7) edge node{}(12)
   (8) edge node{}(9)
   (8) edge node{}(13)
   (9) edge node{}(10)
   (10) edge node{}(11)
   (11) edge node{}(12)
   (10) edge node{}(14)
   (13) edge node{}(14)
   (14) edge node{}(15)
   (14) edge[bend left= 15] node{}(17)
(17) edge[dashed, bend left=15] node{}(14)
   (15) edge node{}(16)
   (16) edge[dashed] node{}(18)
   (17) edge[dashed] node{}(15)
   (17) edge[dashed] node{}(16)
   (17) edge node{}(18);
\end{tikzpicture}
\end{figure}

\vfill

\newpage

%\phantomsection
\chapter*{Предисловие}
\addcontentsline{toc}{chapter}{Предисловие}
\thispagestyle{myheadings}
\markboth{ПРЕДИСЛОВИЕ}{ПРЕДИСЛОВИЕ}

Этот учебник рассчитан на тех, кто решил заниматься дифференциальной геометрией или же хочет найти вескую причину этого не делать.
Материала хватит на один семестр, и ещё останется.

Дифференциальная геометрия опирается на несколько разделов математики, включая
вещественный анализ,
теорию меры,
вариационное исчисление,
дифференциальные уравнения, топологию, элементарную и выпуклую геометрию.
Кроме того, физическая интуиция помогает разобраться во многих её аспектах.
В эту науку уйма входов, поэтому её и интересно, и трудно и преподавать, и изучать.

Гладкие кривые и поверхности дают важнейший источник примеров и идей дифференциальной геометрии.
Разумно хорошо разобраться в этой области, прежде чем идти дальше --- не стоит спешить.

В книжке делается упор на задачи,
доказательства элементарны, наглядны и почти строги (иногда пропускается кое-что из других разделов, в основном тех, что обсуждаются в приложении).
Мы сосредоточились на нескольких идеях, которые точно пригодятся в дальнейшем.
Поэтому обошли вниманием ряд тем, традиционно включаемых в вводные тексты;
например, мы почти не касаемся минимальных поверхностей и формул Петерсона --- Кодацци.

В то же время включены теоремы, которые обычно не обсуждаются в вводных курсах.

{\sloppy

Первый пример --- теорема о луне в луже Владимира Ионина и Германа Пестова (\ref{thm:moon-orginal}).
Это простейший значимый пример так называемых теорем от локального к глобальному, которые лежат в основе всей дифференциальной геометрии;
он даёт хороший ответ на главный вопрос книги --- «Что такое дифференциальная геометрия?».
Другие примеры включают теорему о седловых графиках Сергея Бернштейна (\ref{thm:bernshtein}) и теорему о бесконечной двусторонней геодезической Стефана Кон-Фоссена (\ref{thm:cohn-vossen}).

}

Учебник основан на наших лекциях, прочитанных осенью 2018 года на MASS-программе Университета штата Пенсильвания.
Многие из этих тем использовались Юрием Бураго в его лекциях, читаемых в Ленинградском университете, когда первый автор был его студентом.
При написании мы подглядывали в учебники
Вильгельма Бляшке~\cite{blaschke},
Виктора Топоногова~\cite{toponogov-book}
и Алексея Чернавского \cite{chernavsky}, а также в лекции Сергея Иванова \cite{ivanov};
многие продвинутые упражнения взяты из \cite{petrunin2020}.
Последняя глава основана на вводном материале из книги Стефани Александер, Виталия Каповича и первого автора~\cite{alexander-kapovitch-petrunin2027}.
Мы хотим поблагодарить
Стефани Александер,
Юрия Бураго,
Берка Джейлана,
Нину Лебедеву,
Александра Лычака,
Бенджамина Маккея
и студентов нашего класса
за помощь.

Работа частично поддержана грантом NSF DMS-2005279 и грантом Фонда Саймонса  № 584781.

\begin{flushright}
Антон Петрунин и
\\
Серхио Замора Баррера.
\end{flushright}




