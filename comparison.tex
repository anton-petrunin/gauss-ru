\chapter{Теоремы сравнения}
\label{chap:comparison}

Теоремы сравнения --- мощный инструмент.
Они дают возможность применять евклидову интуицию в дифференциальной геометрии.

\section{Треугольники и створки}

Напомним, что $[x,y]$ и $[x,y]_\Sigma$ обозначают кратчайшую между точками $x$ и $y$ на поверхности $\Sigma$, а
$\dist{x}{y}\Sigma$ --- \index{внутренняя метрика}\emph{внутреннее расстояние} от $x$ до $y$ на~$\Sigma$.

\index{геодезический треугольник}\emph{Геодезический треугольник} на поверхности $\Sigma$ определяется как тройка точек $x,y,z\z\in \Sigma$ с выбранными кратчайшими $[x,y]_\Sigma$, $[y,z]_\Sigma$ и $[z,x]_\Sigma$.
Точки $x,y,z$ называются {}\emph{вершинами} треугольника,
а кратчайшие $[x,y]$, $[y,z]$ и $[z,x]$ --- его {}\emph{сторонами};
сам треугольник обозначается как $[xyz]$ или $[xyz]_\Sigma$;
как всегда, последнее обозначение используется, если хочется подчеркнуть, что треугольник лежит на поверхности~$\Sigma$.\index{10aad@$[xyz]$, $[xyz]_\Sigma$ (геодезический треугольник)}

Треугольник $[\tilde x\tilde y\tilde z]$ на плоскости $\mathbb{R}^2$ называется \index{модельный!треугольник}\emph{модельным треугольником} треугольника $[xyz]$,
если их соответствующие стороны равны;
то есть
\[\dist{\tilde x}{\tilde y}{\mathbb{R}^2}=\dist{x}{y}\Sigma,
\quad
\dist{\tilde y}{\tilde z}{\mathbb{R}^2}=\dist{y}{z}\Sigma,
\quad
\dist{\tilde z}{\tilde x}{\mathbb{R}^2}=\dist{z}{x}\Sigma.
\]
В этом случае пишут $[\tilde x\tilde y\tilde z]=\tilde\triangle xyz$.
\index{10aae@$\tilde\triangle$ (модельный треугольник)}


Пара кратчайших $[x,y]$ и $[x,z]$, исходящих из одной точки $x$, называется \index{створка}\emph{створкой} и обозначается как $\hinge xyz$.\index{10aac@$\hinge yxz$ (створка)}
Угол между этими геодезическими в точке $x$ обозначается как $\measuredangle\hinge xyz$.
Величина угла $\measuredangle\hinge {\tilde x}{\tilde y}{\tilde z}$ модельного треугольника $[\tilde x\tilde y\tilde z]=\tilde\triangle xyz$ обозначается через $\modangle xyz$,
\index{10aab@$\modangle yxz$ (модельный угол)}
и называется \index{модельный!треугольник}\emph{модельным углом} треугольника $[xyz]$ при~$x$.

Согласно признаку равенства треугольников по трём сторонам,
модельный треугольник $[\tilde x\tilde y\tilde z]$ определён с точностью до конгруэнтности.
Поэтому модельный угол $\tilde\theta=\modangle xyz$ однозначно определён.
По теореме косинусов, 
\[\cos \tilde\theta=\frac{a^2+b^2-c^2}{2\cdot a \cdot b},\]
где $a=\dist{x}{y}{\Sigma}$, $b=\dist{x}{z}{\Sigma}$, и $c=\dist{y}{z}{\Sigma}$.

\begin{thm}{Упражнение}\label{ex:wide-hinges}
Пусть $[x_ny_nz_n]$ --- последовательность треугольников на гладкой поверхности~$\Sigma$ 
со сторонами $a_n=\dist{x_n}{y_n}{\Sigma}$,
$b_n=\dist{x_n}{z_n}{\Sigma}$,
$c_n=\dist{y_n}{z_n}{\Sigma}$, и пусть $\tilde\theta_n=\modangle {x_n}{y_n}{z_n}$.
Предположим, что последовательности $a_n$ и $b_n$ отделена от нуля;
то есть $a_n>\epsilon$ и $b_n>\epsilon$ для фиксированного $\epsilon>0$ и любого~$n$.
Покажите, что
\[(a_n+b_n-c_n)\to 0\qquad\iff\qquad \tilde\theta_n\to \pi\]
при $n\to\infty$.
\end{thm}

\section{Формулировки}

Часть \ref{SHORT.thm:comp:cat} следующей теоремы называется {}\emph{теоремой Картана--Адамара};
она доказана 
Гансом фон Мангольдтом \cite{mangoldt} и обобщена 
Эли Картаном \cite{cartan} и
Жаком Адамаром \cite{hadamard}.
Часть \ref{SHORT.thm:comp:toponogov} называется {}\emph{теоремой сравнения Топоногова} и иногда {}\emph{теоремой сравнения Александрова};
она доказана Паоло Пиццетти \cite{pizzetti}, переоткрыта Александром Александровым \cite{aleksandrov}, и 
обобщена Виктором Топоноговым \cite{toponogov1957}.%, Михаилом Громовым, Юрием Бураго и Григорием Перельманом~\cite{burago-gromov-perelman}.

Напомним, что поверхность называется \index{односвязная поверхность}\emph{односвязной}, если любая простая замкнутая кривая на ней ограничивает диск.

\begin{thm}{Теоремы сравнения}
\label{thm:comp}
\index{теорема сравнения}
\ 

\begin{subthm}{thm:comp:cat}
Пусть $\Sigma$ --- открытая односвязная гладкая поверхность с неположительной гауссовой кривизной.
Тогда 
\[\measuredangle\hinge {x}{y}{z}\le\modangle xyz\]
для любого геодезического треугольника $[xyz]$ на $\Sigma$.
\end{subthm}

\begin{subthm}{thm:comp:toponogov}
Пусть $\Sigma$ --- замкнутая (или открытая) гладкая поверхность с неотрицательной гауссовой кривизной.
Тогда 
 \[\measuredangle\hinge {x}{y}{z}\ge\modangle xyz\]
для любого геодезического треугольника $[xyz]$ на $\Sigma$.
\end{subthm}

\end{thm}

Доказательства частей \ref{SHORT.thm:comp:cat} и \ref{SHORT.thm:comp:toponogov} даны в разделах~\ref{sec:nonpos-comp} и~\ref{sec:nonneg-comp} соответственно.

\begin{thm}{Упражнение}\label{ex:thm:comp:cat:nsc}
Покажите, что неравенство в \ref{thm:comp:cat} не выполняется для гиперболоида $\set{(x,y,z)\in\mathbb{R}^3}{x^2+y^2-z^2=1}$.
В частности, \ref{thm:comp:cat} не верно без предположения об односвязности.
\end{thm}

Давайте сравним формулу Гаусса --- Бонне с теоремами сравнения.
Рассмотрим диск $\Delta$ ограниченный геодезическим треугольником $[xyz]$ с внутренними углами $\alpha$, $\beta$ и~$\gamma$.
По формуле Гаусса --- Бонне, 
\[\alpha+\beta+\gamma-\pi=\iint_\Delta K;\]
в частности, у обеих сторон уравнения тот же знак.
Значит,
\begin{itemize}
\item если $K_\Sigma\ge 0$, то $\alpha+\beta+\gamma\ge\pi$, и
\item если $K_\Sigma\le 0$, то $\alpha+\beta+\gamma\le\pi$.
\end{itemize}

\begin{wrapfigure}{r}{35mm}
\centering
\vskip-10mm
\includegraphics{mppics/pic-2307}
\end{wrapfigure}

Теперь пусть 
$\hat\alpha\z=\measuredangle\hinge {x}{y}{z}$,
$\hat\beta\z=\measuredangle\hinge {y}{z}{x}$,
и $\hat\gamma\z=\measuredangle\hinge {z}{x}{y}$.
Заметим, что $\hat\alpha,\hat\beta,\hat\gamma\in[0,\pi]$.
Так как сумма углов любого плоского треугольника равна $\pi$, то из теорем сравнения вытекает, что
\begin{itemize}
\item если $K_\Sigma\ge 0$, то $\hat\alpha+\hat\beta+\hat\gamma\ge\pi$, и
\item если $K_\Sigma\le 0$, то $\hat\alpha+\hat\beta+\hat\gamma\le\pi$.
\end{itemize}

Получается, что формула Гаусса --- Бонне и теоремы сравнения тесно связаны,
однако эта связь не столь прямолинейна.

Например, допустим, что $K\ge 0$.
Тогда формула Гаусса --- Бонне не запрещает внутренним углам $\alpha$, $\beta$ и $\gamma$ одновременно быть близкими к $2\cdot\pi$.
Однако $\hat \alpha=\alpha$ если $\alpha\le \pi$, а если нет, то $\hat \alpha=\pi-\alpha$;
то есть
\begin{align*}
\hat \alpha&=\min\{\,\alpha,2\cdot\pi-\alpha\,\},
&
\hat\beta &=\min\{\,\beta,2\cdot\pi-\beta\,\},
&
\hat\gamma&=\min\{\,\gamma,2\cdot\pi-\gamma\,\}.
\end{align*}
Значит, если $\alpha$, $\beta$ и $\gamma$ близки к $2\cdot\pi$, то $\hat\alpha$, $\hat\beta$ и $\hat\gamma$ будут близки к $0$,
ну а последнее невозможно по теореме сравнения.






\begin{thm}{Упражнение}\label{ex:diam-angle}
Пусть точки $p$ и $q$ максимизируют расстояние на гладкой замкнутой выпуклой поверхности $\Sigma$;
то есть $\dist{p}{q}\Sigma\z\ge\dist{x}{y}\Sigma$ для любых $x,y\in \Sigma$.
Покажите, что $\measuredangle\hinge xpq\ge \tfrac\pi3$
для любой точки $x\z\in \Sigma\setminus\{p,q\}$.
\end{thm}

\begin{thm}{Упражнение}\label{ex:sum=<2pi}
Пусть $\Sigma$ --- замкнутая (или открытая) поверхность неотрицательной гауссовой кривизной.
Покажите, что 
\[\modangle pxy+\modangle pyz+\modangle pzx\le2\cdot \pi\]
для любых четырёх различных точек $p,x,y,z$ на~$\Sigma$.
\end{thm}


\section{Локальные теоремы}\label{sec:loc-comp}

Следующая теорема --- первый шаг в доказательстве теорем сравнения (\ref{thm:comp});
она будет выведена из теоремы сравнения Рауха (\ref{prop:rauch}).

\begin{thm}{Теорема}\label{thm:loc-comp}
Теоремы сравнения (\ref{thm:comp}) выполняются в малой окрестности любой точки.

Более того, если $\Sigma$ --- гладкая поверхность без края,
то для любой точки $p\in \Sigma$ существует такое $r>0$, что если $\dist{p}{x}\Sigma<r$, то $\inj(x)_\Sigma>r$, и справедливы следующие утверждения:

\begin{subthm}{thm:loc-comp:cba}
Если $\Sigma$ имеет неположительную гауссову кривизну, то
\[\measuredangle\hinge {x}{y}{z}\le\modangle xyz\]
для любого геодезического треугольника $[xyz]$ в $B(p,\tfrac r4)_\Sigma$.
\end{subthm}

\begin{subthm}{thm:loc-comp:cbb}
Если $\Sigma$ имеет неотрицательную гауссову кривизну, то 
\[\measuredangle\hinge {x}{y}{z}\ge\modangle xyz\]
для любого геодезического треугольника $[xyz]$ в $B(p,\tfrac r4)_\Sigma$.
\end{subthm}

\end{thm}

\parbf{Доказательство.}
Существование $r>0$ вытекает из \ref{prop:exp}.
Пусть $[xyz]$ --- геодезический треугольник в $B(p,\tfrac{r}4)$.

Поскольку $r<\inj(x)_\Sigma$, найдутся такие векторы $\vec v,\vec w\in\T_x$, что 
\begin{align*}
y&=\exp_x\vec v,
& 
z&=\exp_x\vec w,
\\
|\vec v|_{\T_x}&=\dist{x}{y}\Sigma,
&
|\vec w|_{\T_x}&=\dist{x}{z}\Sigma,
&
\measuredangle\hinge 0{\vec v}{\vec w}_{\T_x}&=\measuredangle\hinge xyz_\Sigma;
\end{align*}
в частности, $|\vec v|, |\vec w|< \tfrac r2$.

\parit{\ref{SHORT.thm:loc-comp:cba}.}
Пусть $\gamma$ --- минимизирующая геодезическая из $y$ в $z$.
Поскольку $\dist{x}{y}{\Sigma},\dist{x}{z}{\Sigma}\z<\tfrac r2$, по неравенству треугольника, $\gamma$ лежит в $r$-окрестности~$x$.
В частности, кривая
$\tilde \gamma\df\log_x\circ\gamma$ определена и лежит в $r$-окрестности нуля в $\T_x$.
Заметим, что $\tilde\gamma$ соединяет $\vec v$ и $\vec w$ в~$\T_x$.

По теореме сравнения Рауха (\ref{prop:rauch:K=<0}),
\[\length \tilde \gamma\le \length\gamma.\]
Так как $\dist{\vec v}{\vec w}{\T_x}\le\length\tilde \gamma$ и $\length\gamma=\dist{y}{z}{\Sigma}$, получаем 
\[\dist{\vec v}{\vec w}{\T_x}\le \dist{y}{z}\Sigma.\]
По монотонности угла (\ref{lem:angle-monotonicity}), 
\[\measuredangle\hinge 0{\vec v}{\vec w}_{\T_x}\le \modangle xzy,\]
и результат следует.

\parit{\ref{SHORT.thm:loc-comp:cbb}.}
Рассмотрим отрезок прямой $\tilde \gamma$, соединяющий $\vec v$ и $\vec w$ в касательной плоскости $\T_x$, и пусть $\gamma\df\exp_x\circ\tilde \gamma$.
По теореме Рауха (\ref{prop:rauch:K>=0}), 
\[\length \tilde \gamma\ge\length\gamma.\]
Поскольку $\dist{\vec v}{\vec w}{\T_x}=\length\tilde \gamma$ и $\length\gamma\ge\dist{y}{z}\Sigma$, 
\[\dist{\vec v}{\vec w}{\T_x}\ge \dist{y}{z}\Sigma.\]
По монотонности угла (\ref{lem:angle-monotonicity}), 
\[\measuredangle\hinge 0{\vec v}{\vec w}_{\T_x}\ge\modangle xzy,\]
отсюда результат.
\qeds


\section{Неположительная кривизна}\label{sec:nonpos-comp}

\parbf{Доказательство \ref{thm:comp:cat}.}
Так как $\Sigma$  односвязна, из \ref{ex:unique-geod} получаем, что 
\[\inj(p)_\Sigma=\infty\]
для любой точки $p\in\Sigma$.
Следовательно, \ref{thm:loc-comp:cba} влечёт \ref{thm:comp:cat}.
\qeds

\section{Неотрицательная кривизна}\label{sec:nonneg-comp}

Сейчас мы докажем \ref{thm:comp:toponogov} для компактных поверхностей.
Общий случай требует лишь незначительных изменений; они приводятся в упражнении \ref{ex:open-comparison} в конце раздела.
Доказательство взято из \cite{alexander-kapovitch-petrunin2027}, похожее доказательство независимо нашли Урс Лэнг и Виктор Шрёдер \cite{lang-schroeder}.

\parbf{Доказательство \ref{thm:comp:toponogov} в компактном случае.}\label{proof(thm:comp:toponogov)}
Предположим, что $\Sigma$ компактна.
По \ref{thm:loc-comp} найдётся такое $\epsilon>0$, что неравенство 
\[\measuredangle\hinge {x}{p}{q}\ge\modangle xpq\]
выполняется если  
$\dist{x}{p}\Sigma+\dist{x}{q}\Sigma<\epsilon$.
Следующая лемма утверждает, что в этом случае то же верно и если $\dist{x}{p}\Sigma+\dist{x}{q}\Sigma<\tfrac32\cdot\epsilon$.
Применив эту лемму несколько раз, получаем, сравнение для любой створки, что и доказывает \mbox{\ref{thm:comp:toponogov}}.
\qeds

\begin{thm}{Основная лемма}\label{key-lem:globalization}
Пусть $\Sigma$ --- гладкая открытая или замкнутая поверхность.
Предположим, что на $\Sigma$ выполняется сравнение
\[\measuredangle\hinge x y z
\ge\modangle x y z\eqlbl{eq:key-lem:globalization}\]
если 
$\dist{x}{y}\Sigma+\dist{x}{z}\Sigma
<
\frac{2}{3}\cdot\ell$.
Тогда \ref{eq:key-lem:globalization} выполняется и если $\dist{x}{y}\Sigma+\dist{x}{z}\Sigma<\ell$.
\end{thm}


{

\begin{wrapfigure}{r}{35mm}
\centering
\vskip-0mm
\includegraphics{mppics/pic-2308}
\end{wrapfigure}

Для данной створки $\hinge x p q$,
рассмотрим треугольник на плоскости с углом $\measuredangle\hinge x p q$ и двумя прилежащими сторонами $\dist{x}{p}\Sigma$ и $\dist{x}{q}\Sigma$.
Обозначим через $\side \hinge x p q$ третью сторону этого треугольника; она будет называться \index{модельная сторона}\emph{модельной стороной} створки.

Следующее вычислительное упражнение сыграет роль в доказательстве леммы.


}

\begin{thm}{Упражнение}\label{ex:s-r}
Предположим, что створки $\hinge xpq$ и $\hinge xpy$ имеют общую сторону $[x,p]$, и $[x,y]\subset [x,q]$.
Покажите, что
\[\frac{\dist{x}{p}{}+\dist{x}{q}{}-\side\hinge xpq}{\dist{x}{q}{}}
\le
\frac{\dist{x}{p}{}+\dist{x}{y}{}-\side\hinge xpy}{\dist{x}{y}{}}.\]
\end{thm}


\parbf{Доказательство.} 
По монотонности угла (\ref{lem:angle-monotonicity})
\[\measuredangle\hinge x p q\ge \modangle x p q\quad\iff\quad\side \hinge x p q
\ge\dist{p}{q}\Sigma.\]
Следовательно, достаточно доказать, что
\[\side \hinge x p q
\ge\dist{p}{q}\Sigma
\eqlbl{eq:thm:=def-loc}\]
при условии, что $\dist{x}{p}\Sigma+\dist{x}{q}\Sigma<\ell$.

Опишем построение новой створки $\hinge{x'}p q$ по данной створке $\hinge x p q$, для которой 
\[\tfrac{2}{3}\cdot\ell \le\dist{p}{x}\Sigma+\dist{x}{q}\Sigma< \ell.\]

\begin{wrapfigure}{r}{32mm}
\vskip-6mm
\centering
\includegraphics{mppics/pic-2310}
\end{wrapfigure}

Будем считать, что $\dist{x}{q}\Sigma\ge\dist{x}{p}\Sigma$; иначе поменяем $p$ и $q$ ролями.
Выберем $x'\in [x, q]$ так, что 
\[\dist{p}{x}\Sigma+3\cdot\dist{x}{x'}\Sigma
=\tfrac{2}{3}\cdot\ell
\eqlbl{3|xx'|}\]
Выберите геодезическую $[x', p]$ и рассмотрите створку $\hinge{x'}p q$, образованную $[x',p]$ и $[x',q]\z\subset [x,q]$.

По неравенству треугольника,
\[
\dist{p}{x}\Sigma+\dist{x}{q}\Sigma\ge\dist{p}{x'}\Sigma+\dist{x'}{q}\Sigma.
\eqlbl{eq:thm:=def-loc-fourstar}\]
Покажем, что
\[\side \hinge x p q
\ge\side \hinge{x'}p q
\eqlbl{eq:thm:=def-loc-fivestar}\]


Из \ref{3|xx'|},
\[
\begin{aligned}
\dist{p}{x}{\Sigma}\z+\dist{x}{x'}{\Sigma}&<\tfrac{2}{3}\cdot\ell,
\\
\dist{p}{x'}{\Sigma}\z+\dist{x'}{x}{\Sigma}&<\tfrac{2}{3}\cdot\ell.
\end{aligned}
\]
Значит, по предположению 
\[\begin{aligned}
\measuredangle\hinge x p{x'}
\ge\modangle x p{x'}
\quad\text{и}\quad
\measuredangle\hinge {x'}p x
\ge\modangle {x'}p x.
  \end{aligned}
\eqlbl{eq:thm:=def-loc-threestar}
\]

Рассмотрим модельный треугольник $[\tilde x\tilde x'\tilde p]\z=\modtrig xx'p$. 
Выберем такую точку $\tilde q$ на продолжении отрезка $[\tilde x,\tilde x']$ за $x'$, что $\dist{\tilde x}{\tilde q}\Sigma=\dist{x}{q}\Sigma$, и, значит, $\dist{\tilde x'}{\tilde q}\Sigma\z=\dist{x'}{q}\Sigma$.

Из \ref{eq:thm:=def-loc-threestar}
\[\measuredangle\hinge x p q
=\measuredangle\hinge x p{x'}\ge\modangle x p{x'}.\]
Следовательно,
\[
\side\hinge x q p
\ge
\dist{\tilde p}{\tilde q}{\mathbb{R}^2}.
\]
Далее, так как $\measuredangle\hinge{x'}p x+\measuredangle\hinge{x'}p q= \pi$,
неравенства в \ref{eq:thm:=def-loc-threestar} означают, что
\[
\pi
-\modangle{x'}p x
\ge
\pi-\measuredangle\hinge{x'}p x
\ge
\measuredangle\hinge{x'}p q.
\]
Следовательно,
\[\dist{\tilde p}{\tilde q}{\mathbb{R}^2}\ge\side \hinge{x'}q p \]
и из этого следует \ref{eq:thm:=def-loc-fivestar}.

Пусть $x_0=x$; применяя построение выше, получаем последовательность створок $\hinge{x_n}p q$, где $x_{n+1}=x_n'$.
Согласно \ref{eq:thm:=def-loc-fivestar} и \ref{eq:thm:=def-loc-fourstar}, обе последовательности
\[s_n=\side \hinge{x_n}pq\quad\text{и}\quad r_n=\dist{p}{x_n}\Sigma+\dist{x_n}{q}\Sigma\]
невозрастающие.

Последовательность может закончиться на $x_n$ только если $r_n\z< \tfrac{2}{3}\cdot\ell$.
Тогда, по предположению леммы, 
\[s_n=\side \hinge{x_n}p q\ge\dist{p}{q}\Sigma.\]
А так как последовательность $s_n$ не возрастает, получаем, что
\[\side \hinge{x}p q=s_0\ge s_n\ge\dist{p}{q}\Sigma;\]
откуда следует \ref{eq:thm:=def-loc}.

\begin{figure}[!ht]
\centering
\includegraphics{mppics/pic-2315}
\end{figure}

Осталось доказать \ref{eq:thm:=def-loc} в случае если последовательностью $x_n$ бесконечна.
Из \ref{3|xx'|}
\[
\dist{x_n}{x_{n-1}}\Sigma
\ge 
\tfrac1{100}\cdot \ell.
\eqlbl{eq:|x-x|><l}
\]
Согласно \ref{eq:thm:=def-loc-fourstar}, $\dist{x_n}{p}{},\dist{x_n}{q}{}<
\ell$ для любого~$n$.
В случае, если $x_{n+1}\z\in [x_n,q]$, применим \ref{ex:s-r} к створкам $\hinge{x_n}pq$ и $\hinge{x_n}p{x_{n+1}}$.
Согласно \ref{eq:thm:=def-loc-threestar}, $\dist{p}{x_{n+1}}{}\le \side \hinge{x_n}{x_{n+1}}{p}$.
Следовательно,
\[r_n-s_n\le 100\cdot (r_n-r_{n+1})\eqlbl{eq:r-s<100(r-r)}\]
В случае, если $x_{n+1}\in [x_n,p]$, неравенство \ref{eq:r-s<100(r-r)} следует, если применить \ref{ex:s-r} к створкам $\hinge{x_n}pq$ и $\hinge{x_n}{x_{n+1}}q$.

Последовательности $r_n$ и $s_n$ являются невозрастающими и неотрицательными;
поэтому они обязаны сходиться.
В частности, $(r_n\z-r_{n+1})\to0$ при $n\to \infty$.
Следовательно, из \ref{eq:r-s<100(r-r)} следует, что
\[\lim_{n\to\infty}s_n=\lim_{n\to\infty}r_n.\]
По неравенству треугольника, $r_n\ge \dist{p}{q}\Sigma$ для любого~$n$.
Поскольку последовательность $s_n$ невозрастающая, получаем
\[\side \hinge{x}p q=s_0\ge \lim_{n\to\infty}s_n=\lim_{n\to\infty}r_n\ge \dist{p}{q}\Sigma,\]
что завершает доказательство \ref{eq:thm:=def-loc}.
\qeds

\begin{thm}{Упражнение}\label{ex:open-comparison}
Пусть $\Sigma$ --- открытая поверхность с неотрицательной гауссовой кривизной.
Для данного $p\in\Sigma$ обозначим $R_p$ 
(\emph{радиус сравнения} в точке $p$) 
максимальное значение (возможно, $\infty$), при котором сравнение 
\[\measuredangle\hinge x p y
\ge\modangle x p y\]
выполняется для любой створки $\hinge x p y$ если $\dist{p}{x}\Sigma+\dist{x}{y}\Sigma<R_p$.

\begin{subthm}{ex:open-comparison:positive}
Покажите, что для любого компактного подмножества $K\z\subset \Sigma$ существует такое $\epsilon>0$, что $R_p>\epsilon$ для любой точки $p\in K$.
\end{subthm}

\begin{subthm}{ex:open-comparison:almost-min}
Используя часть \ref{SHORT.ex:open-comparison:positive}, покажите, что 
существует точка $p\z\in\Sigma$, такая что 
\[R_q>(1-\tfrac1{100})\cdot R_p,\]
для любой точки $q\in B(p,100\cdot R_p)_\Sigma$.
\end{subthm}

\begin{subthm}{ex:open-comparison:proof}
Объясните как, используя \ref{SHORT.ex:open-comparison:almost-min}, можно распространить доказательство \ref{thm:comp:toponogov} (страница \pageref{proof(thm:comp:toponogov)}) на открытые поверхности. 
(То есть доказать, что $R_p=\infty$ для любой точки $p\in\Sigma$.) 
\end{subthm}

\end{thm}

\section{Лемма Александрова}
\index{лемма Александрова}

Следующая лемма (\ref{lem:alex-reformulation}) позволит получить несколько полезных переформулировок теорем сравнения.

\begin{thm}{Лемма}
\label{lem:alex}
Пусть $pxyz$ и $p'x'y'z'$ --- два четырехугольника на евклидовой плоскости с равными соответствующими сторонами.
Предположим, что стороны $[x',y']$ и $[y',z']$ являются продолжениями друг друга; то есть точка $y'$ лежит на отрезке $[x',z']$.
Тогда следующие выражения имеют одинаковые знаки:
\begin{enumerate}[(i)]
 \item $|p-y|-|p'-y'|$;
 \item $\measuredangle\hinge xpy-\measuredangle\hinge {x'}{p'}{y'}$;
 \item $\pi-\measuredangle\hinge ypx-\measuredangle\hinge ypz$.
\end{enumerate}
\end{thm}

\parbf{Доказательство.} 
Рассмотрим точку $\bar z$ на продолжении отрезка 
$[x,y]$ за точкой $y$ так, что $\dist{y}{\bar z}{}=\dist{y}{z}{}$ (и, следовательно, $\dist{x}{\bar z}{}=\dist{x'}{z'}{}$).

\begin{figure}[!ht]
\vskip-0mm
\centering
\includegraphics{mppics/pic-50}
\vskip-0mm
\end{figure}

По монотонности угла (\ref{lem:angle-monotonicity}), следующие выражения одного знака:
\begin{enumerate}[(i)]
\item $|p-y|-|p'-y'|$;
\item $\measuredangle\hinge{x}{y}{p}-\measuredangle\hinge{x'}{y'}{p'}=\measuredangle\hinge{x}{\bar z}{p}-\measuredangle\hinge{x'}{z'}{p'}$;
\item $|p-\bar z|-|p'-z'| = | p - \bar z | - | p-z | $;
\item $\measuredangle\hinge{y}{\bar z}{p}-\measuredangle\hinge{y}{z}{p}$.
\end{enumerate}
Поскольку
\[\measuredangle\hinge{y'}{z'}{p'}+\measuredangle\hinge{y'}{x'}{p'}=\pi
\quad\text{и}\quad
\measuredangle\hinge{y}{\bar z}{p}+\measuredangle\hinge{y}{x}{p}=\pi,\]
утверждение следует.
\qeds

\section{Переформулировки}

Для любого треугольника $[xyz]$ на поверхности $\Sigma$ и его модельного треугольника $[\tilde x \tilde y \tilde z]$, существует естественное отображение $p\mapsto \tilde p$, которое изометрически отображает геодезические $[x,y]$, $[y,z]$, $[z,x]$ на отрезки $[\tilde x,\tilde y]$, $[\tilde y, \tilde z]$, $[\tilde z, \tilde x]$ соответственно.
Треугольник $[xyz]$ называется \index{толстый/тонкий треугольник}\emph{толстым} (или {}\emph{тонким}), если неравенство
\[\dist{p}{q}{\Sigma}\ge |\tilde p- \tilde q|_{\mathbb{R}^2}\qquad \text{(или, соответственно,}\quad \dist{p}{q}{\Sigma}\le |\tilde p- \tilde q|_{\mathbb{R}^2})\]
выполняется для любых двух точек $p$ и $q$ на сторонах треугольника $[xyz]$.

\begin{thm}{Предложение}\label{prop:comp-reformulations}
Пусть $\Sigma$ --- открытая или замкнутая гладкая поверхность.
Тогда следующие три условия эквивалентны:

\begin{subthm}{mang>angk}
Для любого геодезического треугольника $[xyz]$ на $\Sigma$ выполняется неравенство
 \[\measuredangle\hinge{x}{y}{z}\ge\modangle xyz.\]
\end{subthm}

\begin{subthm}{angk>angk}
Для любого геодезического треугольника $[pxz]$ на $\Sigma$ и точки $y$ на стороне $[x,z]$ выполняется неравенство
 \[\modangle xpy \ge \modangle xpz.\]
 
\end{subthm}

\begin{subthm}{fat}
Любой геодезический треугольник на $\Sigma$ является толстым.
\end{subthm}

\medskip

Аналогично, следующие три условия эквивалентны:

\begin{subthmA}{mang<angk}
Для любого геодезического треугольника $[xyz]$ на $\Sigma$ выполняется неравенство
 \[\measuredangle\hinge{x}{y}{z}\le\modangle xyz.\]
\end{subthmA}

\begin{subthmA}{angk<angk}
Для любого геодезического треугольника $[pxz]$ на $\Sigma$ и точки $y$ на стороне $[x,z]$ выполняется неравенство
 \[\modangle xpy \le \modangle xpz.\]
\end{subthmA}

\begin{subthmA}{thin}
Любой геодезический треугольник на $\Sigma$ является тонким.
\end{subthmA}

\end{thm}

Давайте перепишем лемму Александрова (\ref{lem:alex}) на языке модельных треугольников и углов.

\begin{thm}{Переформулировка леммы Александрова}\label{lem:alex-reformulation}
Пусть $[pxz]$ --- треугольник на поверхности $\Sigma$, 
и $y\in[x,z]$. 
Рассмотрим его модельный треугольник $[\tilde p\tilde x\tilde z]\z=\tilde\triangle pxz$, и пусть $\tilde y$ будет соответствующей точкой на стороне $[\tilde x,\tilde z]$.

\begin{wrapfigure}{r}{25mm}
\vskip-2mm
\centering
\includegraphics{mppics/pic-2305}
\end{wrapfigure}

Тогда следующие выражения имеют одинаковые знаки:
\begin{enumerate}[(i)]
 \item $\dist{p}{y}\Sigma-\dist{\tilde p}{\tilde y}{\mathbb{R}^2}$;
 \item $\modangle xpy-\modangle {x}{p}{z}$;
 \item $\pi-\modangle ypx-\modangle ypz$;
\end{enumerate}
\end{thm}

\parbf{Доказательство \ref{prop:comp-reformulations}.}
Мы докажем последовательность импликаций \ref{SHORT.mang>angk}$\Rightarrow$\ref{SHORT.angk>angk}$\Rightarrow$\ref{SHORT.fat}$\Rightarrow$\ref{SHORT.mang>angk}.
Импликации \ref{SHORT.mang<angk}$\Rightarrow$\ref{SHORT.angk<angk}$\Rightarrow$\ref{SHORT.thin}$\Rightarrow$\ref{SHORT.mang<angk} доказываются также, заменяя знаки в неравенствах.

\parit{\ref{SHORT.mang>angk}$\Rightarrow$\ref{SHORT.angk>angk}.}
Заметим, что $\measuredangle\hinge ypx+\measuredangle\hinge ypz=\pi$.
По \ref{SHORT.mang>angk}, 
\[\modangle ypx+\modangle ypz\le \pi.\]
Остаётся применить лемму Александрова (\ref{lem:alex-reformulation}).

\parit{\ref{SHORT.angk>angk}$\Rightarrow$\ref{SHORT.fat}.}
Применив \ref{SHORT.angk>angk} дважды, сначала для $y\in [x,z]$, а затем для $w\in [p,x]$, получим
\[\modangle xwy \ge \modangle xpy \ge \modangle xpz.\]
Следовательно,
\[\dist{w}{y}\Sigma\ge \dist{\tilde w}{\tilde y}{\mathbb{R}^2},\]
где $\tilde w$ и $\tilde y$ --- точки, соответствующие $w$ и $y$ на сторонах модельного треугольника. 

\parit{\ref{SHORT.fat}$\Rightarrow$\ref{SHORT.mang>angk}.}
Поскольку треугольник толстый, имеем
\[\modangle xwy \ge \modangle xpz\]
для любого $w\in [x,p]\setminus \{x\}$ и $y\in [x,z]\setminus \{x\}$.
Заметим, что $\modangle xwy\z\to \measuredangle\hinge xpz$ при $w,y\to x$,
и импликация следует.
\qeds

\begin{thm}{Упражнение}\label{ex:geod-convexity}
Пусть $\gamma$ --- геодезическая с единичной скоростью на гладкой открытой поверхности
$\Sigma$, и $p\in\Sigma$.

Рассмотрим функцию
\[h(t)=\dist{p}{\gamma(t)}\Sigma^2-t^2.\]

\begin{subthm}{}
Покажите, что если $\Sigma$ односвязна и её гауссова кривизна неположительна, то функция $h$ является вогнутой.
\end{subthm}

\begin{subthm}{}
Покажите, что если гауссова кривизна $\Sigma$ неотрицательна, то функция $h$ является выпуклой.
\end{subthm}

\end{thm}

\begin{thm}{Упражнение}\label{ex:midpoints}
Пусть $\bar x$ и $\bar y$ --- середины кратчайших $[p,x]$ и $[p,y]$ на гладкой открытой поверхности~$\Sigma$.

\begin{subthm}{}
Покажите, что если $\Sigma$ является односвязной и её гауссова кривизна неположительна, то 
\[2\cdot \dist{\bar x}{\bar y}\Sigma\le \dist{x}{y}\Sigma.\]
\end{subthm}

\begin{subthm}{}
Покажите, что если гауссова кривизна $\Sigma$ неотрицательна, то 
 \[2\cdot \dist{\bar x}{\bar y}\Sigma\ge \dist{x}{y}\Sigma.\]
\end{subthm}

\end{thm}

\begin{thm}{Упражнение}\label{ex:convex-dist}
Пусть $\gamma_1$ и $\gamma_2$ --- две геодезические на односвязной открытой гладкой поверхности $\Sigma$ с неположительной гауссовой кривизной.
Покажите, что функция $h(t)\df\dist{\gamma_1(t)}{\gamma_2(t)}\Sigma$
является выпуклой.
\end{thm}

\begin{thm}{Упражнение}\label{ex:disc+}
Пусть $\Sigma$ --- открытая или замкнутая гладкая поверхность с неотрицательной гауссовой кривизной.
Докажите, что площадь любого $R$-круга во внутренней метрике $\Sigma$ не превышает $\pi\cdot R^2$.
\end{thm}

\begin{thm}{Упражнение}\label{ex:disc-}
Пусть $\Delta$ --- $R$-шар во внутренней метрике открытой, односвязной, гладкой поверхности $\Sigma$ неположительной гауссовой кривизной.

\begin{subthm}{ex:disc-:kg}
Докажите, что граница $\Delta$ является гладкой кривой с геодезической кривизной не менее $\tfrac1R$.
\end{subthm}

\begin{subthm}{ex:disc-:area}
Докажите, что площадь $\Delta$ не меньше $\pi\cdot R^2$.
\end{subthm}

\end{thm}

Следующее упражнение обобщает задачу о луне в луже (\ref{thm:moon-orginal}).

\begin{thm}{Продвинутое упражнение}\label{ex:moon-}
Пусть $\Delta$ --- диск на гладкой поверхности $\Sigma$ с неположительной гауссовой кривизной.
Предположим, что $\Delta$ ограничен гладкой кривой $\gamma$ с геодезической кривизной, не превышающей 1 по абсолютному значению.
Докажите, что $\Delta$ содержит единичный круг во внутренней метрике $\Sigma$.

Выведите отсюда, что площадь $\Delta$ не меньше $\pi$.
\end{thm}
