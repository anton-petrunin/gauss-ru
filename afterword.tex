\chapter{Послесловие}

Если после этой книжки вы перейдёте к изучению римановой геометрии, то половина материала в этой области окажется вам знакомой.
Но прежде придётся разобраться в тензорномом исчислении; например, по книге Ричарда Бишопа и Самуэля Гольдберга \cite{bishop-goldberg}.

Перечислим несколько вводных текстов, которые мы знаем и любим
от простых и подробных до сжатых и сложных:
\begin{itemize}
\item «Риманова геометрия»  Манфредо до Кармо \cite{carmo1992riemannian}.
\item «Введение в риманову геометрию» Юрия Бураго и Виктора Залгаллера \cite{burago-zalgaller}.
\item «Риманова геометрия в целом»  Дэтлефа Громолла, Вильгельма Клингенберга и Вернера Мейера \cite{gromoll-klingenberg-meyer,gromoll-klingenberg-meyer-ru}.
\item «Comparison geometry»  Джеффа Чигера и Дэвида Эбина \cite{cheeger-ebin}.
\item «Знак и геометрический смысл кривизны» Михаила Громова \cite{gromov-1991}.
\end{itemize}
Удачи.

\begin{flushright}
Антон Петрунин и Серхио Замора Баррера,\\
Стейт-Колледж, Пенсильвания, 11 мая 2021 года.
\end{flushright}

